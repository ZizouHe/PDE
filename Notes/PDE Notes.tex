% @Author: semquant
% @Date:   2018-02-10 18:30:10
% @Last Modified by:   Zizou
% @Last Modified time: 2018-06-19 18:19:52
\documentclass[11pt, a4paper]{article}

\usepackage{algorithm}
\usepackage{amstext}
\usepackage{amsmath,amsfonts,amssymb,amsthm,epsfig,epstopdf,extpfeil,titling,url,array,}
\usepackage{geometry}
\usepackage{algorithmic}
\usepackage{color}
\usepackage{enumerate}
\usepackage[colorlinks, linkcolor=blue, anchorcolor=blue, citecolor=green]{hyperref}
\usepackage{ctex}
\usepackage{bm}
\usepackage{esint}

\geometry{left = 2.54cm, right = 2.54cm, top = 3cm, bottom = 3cm}

\newtheoremstyle{theorem}
  {11pt}
  {}
  {}
  {}
  {\bfseries}
  {.}
  {.5em}
  {}
\theoremstyle{theorem}

\newtheorem{thm}{定理}[section]
\newtheorem{definition}[thm]{定义}
\newtheorem{lemma}[thm]{引理}
\newtheorem{claim}[thm]{声明}
\newtheorem{cor}[thm]{推论}
\newtheorem*{note}{注}
\newtheorem{eg}[thm]{例}
\newtheorem*{refer}{参考书籍}
\newtheorem*{CI}{课程信息}
\newtheorem*{slv}{解}
\newtheorem{hw}{作业}

\newcommand{\intd}[1]{\,\mathrm{d}{#1}}

\makeatletter
\@addtoreset{equation}{subsection}
\makeatother
\renewcommand{\theequation}{\arabic{section}.\arabic{subsection}.\arabic{equation}}

\begin{document}
\title{偏微分方程课堂笔记}
\date{\emph{2018春季}}
\author{李赫}
\maketitle

\tableofcontents

\newpage

\section{课程简介}

\begin{refer}
Textbook: 数学物理方程讲义. 姜礼尚,陈亚浙. 高教第三版,2007.

References:
\begin{enumerate}[(1)]
    \item L.C.Evans, Partial Differential Equations (Part I)
    \item O.A.Olenik: 偏微分方程讲义,高教,2007.
\end{enumerate}
\end{refer}

\begin{CI}
简怀玉,理科楼A412. Tel: 62772864. Email: \emph{hjian@math.tsinghua.edu.cn}

助教:
\begin{enumerate}[(1)]
    \item 涂绪山,18500325351. Email: \emph{1347167157@qq.com}
    \item 朱晓鹏,15201519542. Email: \emph{1303698364@qq.com}
\end{enumerate}

期末考试:60\%,作业:20\%,期中(一次或两次):20\%.
\end{CI}

\begin{definition}
PDE: 含有未知多元函数及该函数的某些阶偏导函数。
\end{definition}

\begin{eg}
线性PDE:关于未知函数及其偏导数均是线性。设$\Omega \subset \mathbb{R}^n$ 为开集,$u$是定义在$\Omega$上的函数,证明:
\begin{align*}
u_{x_i} = \frac{\partial u}{\partial x_i}, u_{x_ix_j} = \frac{\partial^2 u}{\partial x_i \partial x_j}, Du = \triangledown u = (u_{x_1}, ..., u_{x_n}) \\
D^2 u = [u_{x_ix_j}]_{n\times n} (\text{Hessian}), \triangle u = \sum_{i=1}^n u_{x_ix_i} (\text{Laplace})
\end{align*}
\begin{enumerate}
    \item 波动方程:$u_{tt} - \triangle u = f(x,t)$
    \item 热传导方程:$u_{t} - \triangle u = f(x,t)$
    \item 平衡态:$\triangle u = f(x)$ (\text{poisson})
    \item 输运方程:$b^i(x,t), b(x,t), f(x,t)$已知,求$u(x,t)$:
    $$u_t + \sum_{i=1}^n b^i(x,t) u_{x_i} + b(x,t) = f(x,t)$$
\end{enumerate}

设 $[a^{ij}(x)]_{n\times n}$ 是$\Omega$上的对称正定矩阵,$b^i(x), \zeta, f$已知,求
$$
Lu = - \sum_{i,j = 1}^n a^{ij}(x) u_{x_ix_j} + \sum_{i=1}^n b^i(x) u_{x_i} + \zeta(x) u
$$
\begin{enumerate}
    \item 一般波动方程:$u_{tt} + Lu = f$
    \item 一般热传导:$u_t + Lu = f$ (Kolmogrov)
    \item 一般poisson:$Lu = f$
\end{enumerate}
\end{eg}

\begin{eg}
非线性方程:非线性PDE
\begin{enumerate}
    \item 极小曲面方程:$\text{div} \left(\frac{\triangledown u}{\sqrt{1 + \|\triangle u\|^2}}\right) = 0$
    \item 平均曲率流方程:$u_t - \sqrt{1 + \|\triangle u\|^2} \; \text{div} \left(\frac{\triangledown u}{\sqrt{1 + \|\triangle u\|^2}}\right) = 0$
    \item Monge-Ampere方程:$\text{det}(D^2 u) = f(x,u,Du)$
    \item Gauss曲率流: $u_t - \left[\frac{det D^2u}{(1 + \|\triangledown u\|^2)^{\frac{n+t}{2}}}\right] = 0$, ($\alpha > 0$ 已知)
    \item Hamilton-Jacobi: $u_t + H(Du,x) = 0$, $H: \mathbb{R}^n \times \mathbb{R}^n \rightarrow \mathbb{R}$已知
    \item 渗流方程:$u_t - \triangle u^m = 0$
\end{enumerate}
\end{eg}

\begin{eg}
非线性方程组:
\begin{enumerate}
  \item 守恒律方程:$u_t + div F(u) = 0$,其中$u: \Omega \times (0,+ \infty) \rightarrow \mathbb{R}^n$未知,$F: \mathbb{R} \rightarrow M_{n \times n}$已知
  \item Navier-Stoke:
  \begin{align*}
  \begin{cases}
  u_t - \triangle u + u Du + Dp = 0 \\
  div u = 0
  \end{cases}
  \end{align*}
  其中$u: \Omega \times (0, + \infty) \rightarrow \mathbb{R}^n, P: \Omega \times (0, + \infty) \rightarrow \mathbb{R}$未知,$$uDu = \left(\sum_{i=1}^n u^i \frac{\partial u}{\partial x_i}^1, \sum_{i=1}^n u^i \frac{\partial u}{\partial x_i}^2, ..., \sum_{i=1}^n u^i \frac{\partial u}{\partial x_i}^n \right)$$
  \item Schrodinger方程:$i u_t + \triangle u = u \| u\|^p$, $p >0$已知,$u = u_1 + i u_2: \Omega \times (0, +\infty) \rightarrow \mathbb{C}$未知
  \item Ricci流方程:在已知流形M上求一族度量$q_{ij}(x,t)$,
  $$
  \partial_t q_{ij} = - 2 R_{ij} + \frac{2}{n} \frac{\int_M R d\mu}{\int_M d\mu} q_{ij}
  $$
  其中$R_{ij} = F(D^2 q_{ij}), R = q^{ij} R_{ij}, \mu = \sqrt{det(q_{ij})}$
  \item Clay-Problem: 当$n=3$时,$u_0 \in \mathbb{C}^{\infty} (\mathbb{R}^3)$,求一个$u(x,t)$在$\mathbb{R}^n \times (0, + \infty)$中满足Navier-Stoke,且$u|_{t=0} = u_0(x)$,求这样的解是否唯一?
  \item Fields-Medal Problem: $f(x)$的值变号时,方程$det D^2 u = f(x)$是否有解?
  $$
  u_{xx}u_{yy} - u_{xy}^2 = f(x,y)
  $$
\end{enumerate}
\end{eg}

\newpage

\section{方程的建立与定解条件}

\subsection{质量守恒与连续性方程}

以流体为例,$\forall D \subset \Omega, t_2 > t_1$,
\begin{align}
\fbox{\begin{minipage}{5 em}
D在$t_2$时刻的质量
\end{minipage}} - \fbox{\begin{minipage}{5 em}
D在$t_1$时刻的质量
\end{minipage}} = \fbox{\begin{minipage}{8 em}
通过 $\partial D$ 流入的质量 ($t_1 \leq t \leq t_2$)
\end{minipage}} + \fbox{\begin{minipage}{8 em}
D中的源产生的质量 ($t_1 \leq t \leq t_2$)
\end{minipage}}
\end{align}


设$V(x,t): \Omega \times (0, + \infty) \rightarrow \mathbb{R}^3$是流体在x处时刻t的流速(已知)。$\rho(x,t): \Omega \times (0, + \infty) \rightarrow \mathbb{R}$是流体在x处时刻t的密度函数(未知)。$f(x,t): \Omega \times (0, + \infty) \rightarrow \mathbb{R}^3$是流体在x处时刻t源的密度函数

\begin{align}
\int_D \rho(x,t_2) \intd x - \int_D \rho(x,t_1) \intd x = \int_{t_1}^{t_2} \oint_{\partial D} \rho(x,t) \vec{V}(- \vec{n}) \intd x \; \intd t + \int_{t_1}^{t_2} \int_D f(x,t) \intd x \; \intd t
\end{align}

\begin{lemma}
\label{lemma2-1}
设$f, g \in \mathbb{C}(\Omega)$,且任意方体或球均有
\begin{align*}
\int_D f \intd x = \int_D g \intd x
\end{align*}
则$f = g \; \text{in} \; \Omega$
\end{lemma}

\begin{proof}
$\forall x \in \Omega$
\begin{align}
f(x) = \lim_{\text{diam} D \rightarrow 0} \frac{1}{|D|} \int_D f(y) \intd y
\end{align}

\end{proof}

\begin{lemma}
\label{lemma2-2}
(O-G公式):设$\Omega \subset \mathbb{R}^n$为开集,且$\partial \Omega$是分片$C^1$的,$\vec{V} \in C^1(\overline{\Omega}; \mathbb{R}^n)$,则
\begin{align}
\int_\Omega div \vec{V} \intd x = \oint_{\partial \Omega} \vec{V} \vec{n} \intd S
\end{align}
\end{lemma}

\begin{cor}
设$\Omega$同引理\ref{lemma2-2},$u,v \in C^1(\overline{\Omega})$,则
\begin{align}
\int_\Omega u v_{x_i} \intd x = \oint_{\partial \Omega} u v(\vec{n}\cdot \vec{e_i}) \intd S - \int_\Omega v u_{x_i} \intd x
\end{align}
\end{cor}
\begin{proof}
在引理\ref{lemma2-2}中,令$\vec{V} = u \cdot v \cdot e_i$

\end{proof}

回到上面的问题,
\begin{align*}
\int_{t_1}^{t_2} \int_D \frac{\partial \rho}{\partial t} \intd x \; \intd t = - \int_{t_1}^{t_2} \int_D div (\rho \vec{V}) \intd x \; \intd t + \int_{t_1}^{t_2} \int_D f(x,t) \intd x \; \intd t
\end{align*}
因为$D$任意,$t_2 > t_1$任意,所以由引理\ref{lemma2-1},得到连续性方程:
\begin{align}
\frac{\partial \rho}{\partial t} + div(\rho \vec{V}) = f(x,t)
\end{align}
其中$\vec{V} = (V^1, ..., V^3)$。再变换
\begin{align}
\frac{\partial \rho}{\partial t} + div(\vec{V}) \rho + \sum_{i=1}^n V^i(x,t) \rho_{x_i} = f(x,t)
\end{align}
当$b^i = V^i$, $b = div(\vec{V})$时,为输运方程。

\subsection{动量守恒与弦振动方程}

牛顿第二定律:
$$
F = ma \approx \frac{u(t + \triangle t) - u(t)}{\triangle t}, \quad \quad |\triangle t| < < 1
$$
$$
\text{冲量: } F \triangle t \approx m u(t + \triangle t) - m u(t)
$$
对于一个系统,$\forall t_2 > t_1$, 有
\begin{align}
\label{func2-2-1}
\fbox{\begin{minipage}{5 em}
$t_2$ 时刻的动量
\end{minipage}} - \fbox{\begin{minipage}{5em}
$t_1$ 时刻的动量
\end{minipage}} = \fbox{\begin{minipage}{10em}
在 $[t_1, t_2]$ 时间区间内外力产生的冲量
\end{minipage}}
\end{align}

(1) \textbf{Taylor 弦振动方程(1715)}

对$\forall [a.b] \subset [0, l]$ 和 $\forall t_2 > t_1$,式(\ref{func2-2-1})左边为:
\begin{align*}
    = \int_a^b \frac{\partial }{\partial t} u(x, t_2) \rho(x) \intd x - \int_a^b \frac{\partial }{\partial t} u(x, t_1) \rho(x) \intd x
\end{align*}
由外力产生的冲量为:
\begin{align*}
    \int_{t_1}^{t_2} \int_a^b f(x, t) \intd x \intd t
\end{align*}
由近似得
\begin{align}
    &\text{a 处张力: } - |T_a| \sin \alpha_a \approx - |T_a| \tan \alpha_a = - |T_a| \; \frac{\partial }{\partial x} u |_{x = a} \notag \\
    &\text{b 处张力: } - |T_b| \sin \alpha_b \approx - |T_b| \tan \alpha_b = - |T_b| \; \frac{\partial }{\partial x} u |_{x = b}
\end{align}
因为振动是微小的,所以$\alpha_a, \alpha_b \approx 0$。张力产生的冲量为
\begin{align*}
    \int_{t_1}^{t_2} \left[|T_b| \; \frac{\partial }{\partial x} u(b, t)- |T_a| \; \frac{\partial }{\partial x} u(a, t)\right] \intd t
\end{align*}
设$|T_a| = |T_b| = T_0$,则
\begin{align*}
    \int_{t_1}^{t_2} \left(\int_a^b \rho(x)\frac{\partial^2 u}{\partial t^2} \intd x\right) \intd t = \int_{t_1}^{t_2} \int_a^b \left(f(x,t) + T_0 \frac{\partial^2 u}{\partial x^2}\right) \intd x \intd t
\end{align*}
因此
\begin{align*}
    \rho(x) \frac{\partial^2 u}{\partial t^2} - T_0 \frac{\partial^2 u}{\partial x^2} = f(x,t), \quad \quad (x,t) \in (0, l) \times (0, +\infty)
\end{align*}
进一步假设$\rho(x) = \rho_0$为常数,令$a = \sqrt{\frac{T_0}{\rho_0}} > 0$, $F(x,t) = \frac{f(x,t)}{\rho_0}$,则Taylor弦振动方程为
\begin{align}
    u_{tt} - a^2 u_{xx} = F(x,t), \quad \quad x \in (0, l), t > 0
\end{align}
\textcolor{red}{注:如果不为均匀细绳则为一般振动方程。}

(2) \textbf{定解条件}

(i) 初始条件
\begin{align*}
    \begin{cases}
    u(x,0) = \varphi(x) \quad \quad \text{已知} \\
    \frac{\partial }{\partial t} u(x, 0) = \psi(x) \quad \quad \text{已知}
    \end{cases}
\end{align*}

(ii) 边界条件
\begin{enumerate}[(a)]
  \item Dirichlet条件
  \begin{align*}
      \begin{cases}
      u(0, t) = q_1(t) \\
      u(l, t) = q_2(t)
      \end{cases}
  \end{align*}
  \item Neuman条件
    \begin{align*}
      \begin{cases}
      \frac{\partial }{\partial x}u(0, t) = q_1(t) \\
      \frac{\partial }{\partial x}u(l, t) = q_2(t)
      \end{cases}
  \end{align*}
  \item 混合边界
    \begin{align*}
      \begin{cases}
      \frac{\partial u}{\partial x} + \alpha u |_{x = 0}= q_1(t) \\
      \frac{\partial u}{\partial x} + \alpha u |_{x = l}= q_2(t)
      \end{cases}
  \end{align*}
\end{enumerate}

(3) \textbf{Euler波方程}

二维情况:设$\Omega \subset R^2$, $u = u(x, y;t)$ 满足
\begin{align}
    u_{tt} - a^2 \left(\frac{\partial^2 u}{\partial x^2} + \frac{\partial^2 u}{\partial y^2} \right) = f(x,y; t), \quad \quad (x,y; t) \in \Omega \times (0, +\infty)
\end{align}

三维情况:设$\Omega \subset R^3$, $u = u(x, y, z;t)$ 满足
\begin{align}
    u_{tt} - a^2 \left(\frac{\partial^2 u}{\partial x^2} + \frac{\partial^2 u}{\partial y^2} + \frac{\partial^2 u}{\partial z^2}\right) = f(x,y,z; t), \quad \quad (x,y,z; t) \in \Omega \times (0, +\infty)
\end{align}

事实上,
\begin{align}
    T_0 \frac{\partial u}{\partial \vec{n}} = T_0 \triangledown u \vec{n}
\end{align}

(4) \textbf{一般维数的波方程}

设$\Omega \subset R^n$为开集,$n \geq 1$,$a > 0$为常数,$0 < T \leq + \infty$,$f(x,t)$是定义在$\Omega \times (0, T)$上的函数。
\begin{align}
    \text{方程:} u_{tt} - a^2 \triangle u = f(x,t), \quad \quad (x,t) \in \Omega \times (0, T)
\end{align}

\begin{enumerate}[(a)]
  \item 初始条件
  \begin{align*}
    \begin{cases}
    u(x,0) = \varphi(x) \\
    \frac{\partial }{\partial t} u(x, 0) = \psi(x) \quad \quad x \in \Omega
    \end{cases}
  \end{align*}
  \item Dirichlet条件
  \begin{align*}
      u|_{\partial \Omega} = q_1(x,t), \quad \quad x \in \partial \Omega, 0 \leq t < T
  \end{align*}
  \item Neuman条件
    \begin{align*}
      \frac{\partial u}{\partial \vec{n}}|_{\partial \Omega} = q_2(x,t)
  \end{align*}
  \item 混合边界
    \begin{align*}
      \left(\frac{\partial u}{\partial \vec{n}} + \alpha u\right)|_{\partial \Omega} = q_3(x,t)
  \end{align*}
\end{enumerate}

(5) \textbf{定解问题:带有定解条件的方程}

\begin{enumerate}
  \item 初边值问题:$\partial \Omega$非空,对应三类边界条件分别有Dirichlet初边值问题,Newman初边值问题和混合初边值问题。
  \item Cauchy问题:$\Omega = R^n$($\partial \Omega$为空集)叫做初值问题。
\end{enumerate}

(6) \textbf{齐次性:}

当$f \equiv 0$时,为齐次方程;当$q_i \equiv 0$时,为齐次边界条件。 任何两个解的线性组合仍旧满足同一个方程(同一个边界条件),那么这样的方程(边界条件)是齐次的。

\subsection{能量守恒与热传导方程}

设$\Omega \subset R^n \; (n = 1, 2, \cdots)$表示有热传导现象的物体。考虑其温度的变化情况。$u(x,t): \Omega \times (0, T) \rightarrow R$未知,已知物体的密度函数$\rho(x)$,热源强度函数$f(x,t)$,比热系数$c$,热传导系数$k$。$\forall D \subset \Omega, \forall t_2 > t_1$,
\begin{align}
\label{func2-3-1}
\fbox{\begin{minipage}{5 em}
D在$t_2$时刻的热量
\end{minipage}} - \fbox{\begin{minipage}{5 em}
D在$t_1$时刻的热量
\end{minipage}} = \fbox{\begin{minipage}{9 em}
$[t_1 t_2]$时间区间内热交换产生的热量(出现在边界$\partial \Omega$上)
\end{minipage}} + \fbox{\begin{minipage}{6.5 em}
$[t_1 t_2]$时间区间内热源产生的热量
\end{minipage}}
\end{align}
方程(\ref{func2-3-1})的左端为
$$
\int_D c u(x, t_2) \rho(x) \intd x - \int_D c u(x, t_1) \rho(x) \intd x
$$
方程(\ref{func2-3-1})的右端第二项为
$$
\fbox{\begin{minipage}{8 em}
$[t_1 t_2]$时间区间内热源产生的热量
\end{minipage}} = \int_{t_1}^{t_2} \int_D f(x,t) \intd x \intd t
$$
方程(\ref{func2-3-1})的右端第一项为
$$
\int_{t_1}^{t_2} \oint_{\partial D} k \frac{\partial u}{\partial \vec{n}} \intd s \intd t
$$
进行整理:
$$
\int_{t_1}^{t_2} \int_D \frac{\partial }{\partial t} (\rho c u) \intd x \intd t = \int_{t_1}^{t_2} \int_D \left[f + div(\triangledown(ku))\right] \intd x \intd t
$$
因为$div(\triangledown(ku)) = k \triangle u$,
\begin{align}
    &\frac{\partial }{\partial t} (\rho c u) = k \triangle u + f \notag \\
    \Longrightarrow &u_t - a^2 \triangle u = f
\end{align}
where $a = \sqrt{\frac{k}{\rho x}}, \hat{f} = \frac{f}{\rho c}$

\begin{enumerate}[(1)]
  \item 方程$u_t - a^2 \triangle = f(x,t)$ in $\Omega \times (0.T)$
  \item 定解条件
  \begin{enumerate}[(i)]
    \item 边值条件
    \begin{enumerate}[(a)]
      \item $u | _{\partial \Omega \times (0,T)} = q_1(x,t)$
      \item $\frac{\partial u}{\partial \vec{n}} = q_2(x,t)$ on $\partial \Omega \times (0,T)$
      \item $\frac{\partial u}{\partial \vec{n}} + \alpha u = q_3(x,t)$
    \end{enumerate}
    \item 初始条件 $u(x,0) = \varphi(x), x \in \Omega$,$\Longrightarrow$ 定解问题(初边值问题,Cauchy问题)
  \end{enumerate}
  \item 齐次性
  \item 平衡态 $f(x,t) = f(x)$,$q_i$与$t$无关。$t \rightarrow 0$, $u(x,t) \rightarrow u(x)$,则Poission方程
  \begin{align}
      - \triangle u = f(x)  \quad \text{in} \; \Omega
  \end{align}
  其中,边界条件为:
  \begin{enumerate}[(i)]
    \item $u|_{\partial \Omega} = q_1(x)$ (Dirichlet);
    \item $\frac{\partial u}{\partial \vec{n}} |_{\partial \Omega} = q_2(x)$ (Neuman);
    \item $\frac{\partial u}{\partial \vec{n}} + \alpha u = q_3(x)$ (混合)。
  \end{enumerate}
\end{enumerate}

\begin{eg}
通过变换$u = v + w$,将下面关于$u$的问题转化为关于$v$的齐次方程和齐次边界条件。
\begin{align*}
\begin{cases}
u_{tt} - u_{xx} = f(x), \quad \quad 0 < x < l, t >0 \\
u(0,t) = 1, u_x(l,t) = 2 \\
u(x,0) = \varphi(x), u_t(x,0) = \psi(x)
\end{cases}
\end{align*}

解:用$u = v+w$代入,由题意,只需要$w$满足
\begin{align*}
\begin{cases}
w_{tt} - w_{xx} = f(x), \quad \quad 0 < x < l, t >0 \\
w(0,t) = 1, w_x(l,t) = 2 \\
\end{cases}
\end{align*}
由于$f$与$t$无关,可以设$w = w(x)$,则
\begin{align*}
\begin{cases}
- w_{xx} = f(x), \quad \quad 0 < x < l \\
w(0) = 1 \\
w(1) = 2
\end{cases}
\end{align*}
解出:
$$
w(x) = 2 x + 1 + \int_0^x \int_t^l f(y) \intd y \intd t
$$
\end{eg}

\subsection{极小曲面及变分问题}

在$\mathbb{R}^3$中给一条闭曲线,
\begin{align*}
\begin{cases}
x = x(s) \\
y = y(s) \\
z = \psi(x(s), y(s))
\end{cases}
\end{align*}

设闭曲线在$x \circ y$平面上围成的区域为$D$,求一个$\mathbb{C}^1$曲面$\Sigma$,满足:(a) $\Sigma$以$l$为周界; (b) $\Sigma$的面积最小。记
\begin{align}
    M_\psi := \left\{v \in \mathbb{C}^1 (\overline{D}): v |_{\partial D} = \psi\right\}
\end{align}
因此面积:
\begin{align}
    J(u) = \text{Area}(\Sigma) = \iint\limits_D  \sqrt{1 + |\triangledown u|^2} \intd x \intd y, \quad \quad \forall u \in M_\psi
\end{align}

任取$v \in M_\psi$,$\forall \varpi \in \mathbb{C}^1(\overline{D})$,$\varphi |_{\partial D} = 0$,令$j(t) = J(v + t \varphi), t \in R$。

\textbf{(1) 必要性}:如果$v = u$是最小曲面,则$j(t)$在$t = 0$处一定取到最小值,i.e. $j'(0) = 0$
\begin{align}
    j'(t) &= \frac{d}{d t} \iint\limits_D \sqrt{1 + |\triangledown v + t \triangledown \varphi|^2} \intd x \intd y \notag \\
    &= \iint\limits_D \frac{(\triangledown v + \triangledown t \varphi) \triangledown \varphi}{\sqrt{1 + |\triangledown v + t \triangledown \varphi|^2}} \intd x \intd y
\end{align}
所以
\begin{align*}
\iint\limits_D \frac{\triangledown v \triangledown \varphi}{\sqrt{1 + |\triangledown v|^2}} \intd x \intd y = 0
\end{align*}
又因为\footnote{根据求导法则,$\text{div}(v F) = \text{div}(F)v + \triangledown v \cdot F$,其中$F$是向量场,$v$是标量函数。}
\begin{align*}
    \text{div}\left(\frac{\triangledown v}{\sqrt{1 + |\triangledown v|^2}} \varphi\right) = \text{div}\left(\frac{\triangledown v}{\sqrt{1 + |\triangledown v|^2}}\right) \varphi + \frac{\triangledown v \triangledown \varphi}{\sqrt{1 + |\triangledown v|^2}}
\end{align*}
和Gauss公式:\footnote{$\iiint_D \text{div}(F) \intd V = \oiint_{\partial D} F \cdot \vec{n} \intd S = \oiint_{\partial D} \frac{\partial F}{\partial \vec{n}} \intd S$,其中$F$是向量场。}
\begin{align*}
    \iint_D\text{div}\left(\frac{\triangledown v}{\sqrt{1 + |\triangledown v|^2}} \varphi\right) \intd x \intd y
    &= \oint_{\partial D} \left(\frac{\triangledown v}{\sqrt{1 + |\triangledown v|^2}} \varphi\right) \cdot \vec{n} \intd S \\
    &= \oint_{\partial D} \left(\frac{\varphi}{\sqrt{1 + |\triangledown v|^2}} \right) \frac{\partial v}{\partial \vec{n}} \intd S \\
    &= 0,\ \quad \left(\because \varphi\bigg|_{\partial D} = 0\right)
\end{align*}
\begin{align}
    \therefore \; &\iint\limits_D \text{div}\left(\frac{\triangledown v}{\sqrt{1 + |\triangledown v|^2}}\right) \varphi \intd x \intd y = 0 \quad\quad \forall v \in \mathbb{C}^2
\end{align}

根据$\varphi$的任意性,我们得到\textbf{MSE}: $\text{div}\left(\frac{\triangledown v}{\sqrt{1 + |\triangledown v|^2}}\right) = 0 \; \text{in} \; D$,$v|_{\partial D} = \psi$。


\begin{lemma}
如果$f \in \mathbb{C}(\Omega)$, $\varphi \in \mathbb{C}_0^\infty (\Omega) := \left\{\varphi \in \mathbb{C}^\infty(\Omega) \; \text{且} \; \{x: \varphi(x) \neq 0\} \subset \subset \Omega\right\}$,有$\int_\Omega f \varphi \intd x = 0$,则$f \equiv 0$ in $\Omega$。
\end{lemma}

\begin{proof}
反证:设$f \not\equiv 0$,不妨设$x_0 \in \Omega, f(x_0) > 0$。\\
由$f$的连续性,$\exists \varepsilon > 0$, s.t. $f(x) \geq \frac{f(x_0}{2} = \delta$,$B_\varepsilon (x_0) \subset \subset \Omega$,令
\begin{align*}
\varphi(x) = \begin{cases}
\exp\{\frac{1}{|x|^2 - \varepsilon^2}\} \quad \quad &|x| < \varepsilon \\
0 \quad \quad &|x| \geq \varepsilon
\end{cases}
\end{align*}
则$\varphi \in \mathbb{C}_0^\infty (\Omega)$,但是
$$
\int_\Omega f \varphi \intd x = \int_{B_\varepsilon(x_0)} f \varphi \intd x \geq \delta \int_{B_\varepsilon(x_0)} \varphi \intd x> 0  \quad \text{矛盾!}
$$
\end{proof}

\textbf{(2) 充分性}:设$v= u$是MSE的解,\\
只需验证:$j''(t) \geq 0, \forall t \in R$,其中$j(t) = J(u + t \varphi)$,
$$
\because j''(t) = \int\limits_D \frac{|\triangledown \varphi|^2 (1 + |\triangledown u + t \triangledown \varphi|^2) - |(\triangledown u + t \triangledown \varphi) \triangledown \varphi|^2}{(1 + (\triangledown u + t \triangledown \varphi)^2)^{3/2}} \geq 0
$$
(因为$|ab| \leq |a||b|$),那么我们可以得到$\forall \varphi \in \mathbb{C}^1(\overline{D})$,$\varphi = 0$ on $\partial \Omega$,有$J(u + t \varphi)$在$t = 0$处取得最小值
$$\Longrightarrow J(u + \varphi) \geq J(u), \quad \forall v \in M_\varphi$$
令$\varphi = v - u$,
$$
J(v) = J(u + (v - u)) \geq J(u)
$$

\begin{eg}
\textbf{Chapter 1-12}. 求解変分问题,求$u \in M := \left\{y(x) | y \in \mathbb{C}^1[0,1] , y(1) = 0\right\}$ s.t.
$$
J(u) = \min\limits_{y \in M} J(y)
$$
其中
\begin{align}
    J(y) = \frac{1}{2} \int_0^1 (y'(x))^2 \intd x - 2 \int_0^1 y(x)\intd x - y(0)
\end{align}

解:令$j(t) = J(u + t \varphi)$,其中$u \in M$(是最小的),$\varphi \in \mathbb{C}^1([0,1]), \varphi(1) = 0$是任意的。
\begin{align*}
\therefore j'(t) &= \int_0^1 (u' + t \varphi') \varphi' - 2\int_0^1 \varphi \intd x - \varphi(0) \\
j''(t) &= \int_0^1 \varphi^{'2} \geq 0
\end{align*}
所以只要求$u$ s.t. $j'(0) = 0$即可。
\begin{align*}
0 = \int_0^1 u' \varphi' \intd x - 2 \int_0^1 \varphi \intd x - \varphi(0), \quad \quad \forall \varphi \in \mathbb{C}^1([0,1]), \varphi(1) = 0
\end{align*}
\begin{align*}
\therefore 0 = \left(u'(0) +1\right) \varphi(0) + \int_0^1 (u'' + 2) \varphi \intd x
\end{align*}
所以我们可以得到
\begin{align*}
    \begin{cases}
    u'' + 2 = 0 \; \text{in} \; [0,1]\\
    u(1) = 0 \\
    u'(0) = -1
    \end{cases}
\end{align*}
\end{eg}

\newpage

\section{波动方程}

考虑
\begin{align}
    u_{tt} - a^2 u_{xx} = f(x,t)
\end{align}
令$v = \partial_t u + a \partial_x u$,则
\begin{align*}
    \begin{cases}
    \partial_t v - a\partial_x v = f(x,t) \\
    \partial_t u + a \partial_x u = v(x,t)
    \end{cases}
\end{align*}

\subsection{一阶偏微分方程的特征线方法}

以线性方程为例,
\begin{align*}
    a u_t + B_1(x,t) \triangledown_x u + b_1(x, t) u = 0, \quad \quad x \in \mathbb{R}^n
\end{align*}
标准形式为
\begin{align}
\label{func3-1}
    u_t  + B(x,t) \triangledown_x u + b(x,t) u = 0, \quad \quad x \in \mathbb{R}^n
\end{align}
\textbf{idea}:$u$在曲线$x = x(t)$上,$\overline{u} = u(x(t), t)$
$$
\frac{d}{d t} \overline{u}(t) = u_t + \dot{x}(t) \triangledown u (x,t)
$$
\textbf{Step 1}:先求特征线
\begin{align*}
    \begin{cases}
    \dot{x}(t) = B(x,t) \\
    x(0) = c, \quad c \in \mathbb{R}^n
    \end{cases}
\end{align*}
解得
\begin{align}
\label{func3-2}
    x = x(t,c)
\end{align}
\textbf{Step 2}:沿特征线方程(\ref{func3-1})变为
\begin{align}
\frac{d}{d t} u(x(t),t) + b(x(t,c),t) u(x(t,c),t) = 0
\end{align}
解得
\begin{align}
\label{func3-3}
    u= u(c, t)
\end{align}
\textbf{Step 3}:从(\ref{func3-2})$c = c(x,t)$代入(\ref{func3-3}),得
$$
u = u(x(x,t), t) = u(x,t)
$$
即为所求。

\begin{eg}
解一维的输运方程,其中$a$为常数,$b, \rho_0$为$x$的已知函数。
\begin{align}
    \begin{cases}
    \rho_t + a \rho_x + b(x) \rho = 0, \quad x \in \mathbb{R}, t > 0 \\
    \rho(x,0) = \rho_0(x), \quad x \in \mathbb{R}
    \end{cases}
\end{align}
\end{eg}

\begin{slv}
特征线
\begin{align*}
\begin{cases}
\dot{x}(t) = a \\
x(0) = c
\end{cases}
\Longrightarrow x(t) = at +c, \quad c \in \mathbb{R}
\end{align*}
沿特征线问题转化为
\begin{align*}
\begin{cases}
\frac{d}{d t} \rho(at + c, t) + b(at + c) \; \rho(at + c, t) = 0 \\
\rho(c,0) = \rho_0(c)
\end{cases}
\end{align*}
解得
$$
\ln \rho(at + c, t) - \ln \rho_0(c) = - \int_0^t b(as + c) \intd s
$$
$$
\therefore \rho(at+c,t) = \rho_0(c) \exp\{- \int_0^t b(as + c) \intd s\}
$$
由特征线方程$c = x - at$代入
\begin{align}
    \rho(x,t) = \rho_0(x-at) \exp\{- \int_0^t b(as + x - at) \intd s\}
\end{align}
\end{slv}

\begin{lemma}
\label{lemma3-1}
问题
\begin{align}
    \begin{cases}
    \rho_t + a\rho_x = 0, \quad \quad x \in \mathbb{R}, t > 0 \\
    \rho(x,0) = \rho_0(x)
    \end{cases}
\end{align}
的解为$\rho(x,t) = \rho_0(x - at)$,其中$\rho_0 \in \mathbb{C}^1(\mathbb{R})$。
\end{lemma}

\begin{lemma}
方程$\rho_t + a \rho_x = 0,x \in \mathbb{R}, t > 0$的通解为$\rho(x,t) = f(x - at)$,其中$f$为任一可微函数。
\end{lemma}

\subsection{非齐次方程的解法 - Duhamel原理}
\begin{align}
  \begin{cases}
  \frac{\intd v}{\intd t} = f(t) \\
  v(0) = 0
  \end{cases} \Longrightarrow
  v(t) = \int_0^t f(s) \intd s = \int_0^t w(t,s) \intd s
\end{align}
\begin{align*}
\forall s \geq 0, \quad \quad \begin{cases}
\frac{\intd v}{\intd t} = 0 \\
v|_{t=s} f(s)
\end{cases} \Longrightarrow
v = w(t,s) = f(s)
\end{align*}
$$
\therefore v(t) = \int_0^t w(t,s) \intd s
$$
\begin{lemma}
$\forall s \geq 0$,设$u = w(x,t,s)$是问题
\begin{align}
\begin{cases}
u_t + P(D_x) u = 0, \quad \quad x \in \mathbb{R}^m, t > 0 \\
u|_{t=s} = f(x,s)
\end{cases}
\end{align}
的解,则$\overline{u}(x,t) = \int_0^t w(x,t,s) \intd s$是非齐次问题
\begin{align}
    \begin{cases}
    u_t + P(D_x)u = f(x,t), \quad \quad x \in \mathbb{R}^n, t > 0 \\
    u|_{t = 0} = 0
    \end{cases}
\end{align}
的解,其中$P(D_x)$是关于$x$的一个线性连续微分算子。
\end{lemma}

\begin{proof}
$u|_{t = 0} = 0$显然,又因为
\begin{align*}
    \overline{u}_t &= \frac{\partial }{\partial t} \int_0^t w(x,t,s) \intd s \\
    &= w(x,t,t) + \int_0^t \frac{\partial w}{\partial t} \intd s \\
    &= f(x,t) - \int_0^t P(D_x) w(x,t,s) \intd s \\
    &= f(x,t) - P(D_x) \int_0^t w(x,t,s) \intd s
\end{align*}
\end{proof}

\begin{lemma}
$\forall s \geq 0$,设$u = w(x,t,s)$是问题
\begin{align}
\begin{cases}
u_{tt} + P(D_x) u = 0, \quad \quad x \in \mathbb{R}^m, t > 0 \\
u|_{t=s} = 0, \quad u_t|_{t=s} = f(x,s)
\end{cases}
\end{align}
的解,则$\overline{u}(x,t) = \int_0^t w(x,t,s) \intd s$是非齐次问题
\begin{align}
    \begin{cases}
    u_{tt} + P(D_x) u = f(x,t), \quad \quad x \in \mathbb{R}^m, t > 0 \\
    u|_{t=0} = 0, \quad u_t|_{t=0} = 0
    \end{cases}
\end{align}
的解,其中$P(D_x)$是关于$x$的一个线性连续微分算子。
\end{lemma}

\begin{proof}
$u_t|_{t = 0} = 0$显然,
$$
\overline{u}_t = \frac{\partial }{\partial t} \int_0^t w(x,t,s) \intd s = w(x,t,t) + \int_0^t \frac{\partial }{\partial t} w(x,t,s) \intd s
$$
$$
\therefore \overline{u}_t|_{t = 0} = 0 + 0 = 0
$$
又因为
\begin{align*}
\overline{u}_{tt} &= 0 + \frac{\partial }{\partial t} w(x,t,s) |_{s = t} + \int_0^t \frac{\partial^2 }{\partial t^2} w(x,t,s) \intd s \\
&= f(x,t) + \cdots
\end{align*}
\end{proof}

\subsection{一般维数波动方程Cauchy问题的解法}

\begin{align}
\label{func3-4}
    \begin{cases}
    u_{tt} - a^2 \triangle u = f(x,t), \quad \quad &x \in \mathbb{R}^n, t > 0 \\
    u|_{t = 0} = \varphi(x), u_t|_{t = 0} = \psi(x), \quad \quad &x \in \mathbb{R}^n
    \end{cases}
\end{align}
从物理上,$u$分解为三部分,$V \leftarrow f$,$W \leftarrow \varphi$,$Q \leftarrow \psi$。如果$V,W,Q$是下面三个问题的解
\begin{align}
\label{func3-5}
    &\begin{cases}
    V_{tt} - a^2 \triangle V = f(x,t), \quad \quad x \in \mathbb{R}^n, t > 0 \\
    V|_{t = 0} = V_t|_{t = 0} = 0, \quad \quad x \in \mathbb{R}^n
    \end{cases} \\
    \label{func3-6}
    &\begin{cases}
    W_{tt} - a^2 \triangle W = 0, \quad \quad x \in \mathbb{R}^n, t > 0 \\
    W|_{t = 0} = \varphi(x), W_t|_{t = 0} = 0, \quad \quad x \in \mathbb{R}^n
    \end{cases}\\
    \label{func3-7}
    &\begin{cases}
    Q_{tt} - a^2 \triangle Q = 0, \quad \quad x \in \mathbb{R}^n, t > 0 \\
    Q|_{t = 0} = 0, Q_t|_{t = 0} = \psi(x),  \quad \quad x \in \mathbb{R}^n
    \end{cases}
\end{align}
则(\ref{func3-4})的解为$u = V+W+Q$。设(\ref{func3-7})的解为$Q(x,t) = M_\psi(x,t)$,则$\forall s \geq 0$,$f_s(x) = f(x.s)$。
\begin{align*}
    \begin{cases}
    \overline{Q}_{tt} - a^2 \triangle \overline{Q} = 0 \\
    \overline{Q}|_{t = s} = 0, \overline{Q}_t |_{t = s} = f_s(x)
    \end{cases}
\end{align*}
的解$\overline{Q}(x,t) = Q(x,t-s) = M_{f_s}(x,t - s)$,则由Duhamel原理得,
\begin{align*}
    V(x,t) = \int_0^t M_{f_s}(x,t-s) \intd s
\end{align*}

令$W(x,t) = \frac{\partial}{\partial t} M_\varphi(x,t)$,则$W|_{t = 0} = \frac{\partial}{\partial t} M_\varphi(x,t) |_{t = 0} = \varphi$

\begin{align*}
W_t|_{t = 0} &= \frac{\partial^2 }{\partial t^2} M_\varphi(x,t) |_{t = 0} = a^2 \triangle M_\varphi(x,t) |_{t = 0} \\
&= a^2 \triangle M_\varphi(x,0) = a^2 \triangle(0) = 0
\end{align*}
方程满足条件。

\begin{thm}
\label{thm3-4}
若$M_\psi, M_\varphi$在$\mathbb{R}^n \times [0, + \infty)$上分别有连续二阶和三阶偏导数,而$M_{f_s}$在$\{x \in \mathbb{R}^n, 0 \leq s \leq t < + \infty\}$上有连续的二阶偏导,则(\ref{func3-4})的解为
\begin{align}
    u(x,t) = M_\psi(x,t) + \frac{\partial}{\partial t} M_\varphi(x,t) + \int_0^t M_{f_s}(x,t-s) \intd s
\end{align}
\end{thm}

\begin{thm}
若
\begin{align}
    \begin{cases}
    y'' + a(t) y' + b(t) y = 0 \\
    y|_{t = 0} = y'|_{t = o} = 0
    \end{cases}
\end{align}
的解为$M_c(t)$,且$M_{c_1}(t), M_{c_2}(t)$在$[0,+\infty)$上分别有连续二阶和三阶导数,$M_{f^s}(x,t-s)$在$\{t \geq 0, 0 \geq s \leq t < + \infty\}$上有连续二阶偏导数,则$\overline{y}(t) = \frac{d}{d t} M_{c_1}(t) + M_{c_2}(t) + \int_0^t M_{f^s} (x, t-s) \intd s$是
\begin{align}
    \begin{cases}
    \overline{y}'' + a \overline{y}' + b \overline{y} = 0 \\
    \overline{y}|_{t = 0} = c_1, \overline{y}'|_{t = o} = c_2
    \end{cases}
\end{align}
的一个解。
\end{thm}

\subsection{Taylor弦振动方程的解法}

\begin{align}
\label{func3-12}
    \begin{cases}
    u_{tt} - a^2 u_{xx} = f(x,t), \quad \quad &x \in \mathbb{R}^n, t > 0 \\
    u|_{t = 0} = \varphi(x), u_t|_{t = 0} = \psi(x), \quad \quad &x \in \mathbb{R}^n
    \end{cases}
\end{align}
只需要求解
\begin{align*}
    \begin{cases}
    w_{tt} - a^2 w_{xx} = 0, \quad \quad &x \in \mathbb{R}^n, t > 0 \\
    w|_{t = 0} = 0, w_t|_{t = 0} = \psi(x),\quad \quad &x \in \mathbb{R}^n
    \end{cases}
\end{align*}
$$
\therefore w_{tt} - a^2 w_{xx} = (\partial_t - a \partial_x)(\partial_t w + a \partial_w)
$$
令$v(x,t) = \partial_t w +  a \partial_x w$,则
\begin{align*}
    \begin{cases}
    \partial_t v- a \partial_x v= 0 \\
    v|_{t = 0} = \partial_t w|_{t = 0} +  a \partial_x w|_{t = 0} = \psi(x)
    \end{cases}
\end{align*}
由lemma(\ref{lemma3-1}),特征线为$x(t) = -at + c$,因此原问题化为
\begin{align*}
    \begin{cases}
    \frac{d}{d t} v(x(t),t)= 0 \\
    v(x(0),0) = \psi(x)
    \end{cases}
\end{align*}
由第一个式子知,$v(x(t),t)$与$t$无关,因此$v(x(t),t) = v(x(0),0) = v(c,0) = \psi(x(0))$,
$v(x,t) = \psi(x + at)$。则,
\begin{align}
\label{func3-10}
    \begin{cases}
    \partial_t w + a \partial_x w = \psi(x + at), \quad \quad x \in \mathbb{R}^n, t > 0 \\
    w|_{t = 0} = 0
    \end{cases}
\end{align}
我们先解$\forall s \geq 0$,
\begin{align}
\label{func3-11}
    \begin{cases}
    \overline{w}_t + a \overline{w}_x = 0, \quad \quad &x \in \mathbb{R}^n \\
    \overline{w}|_{t=s} = \psi(x + as), & t \geq s
    \end{cases}
\end{align}
将$t \rightarrow t - s$,则(\ref{func3-11})的解为
$$
\overline{w}(x,t,s) = \psi(x+as - a(t-s)) = \psi(x - a(t - 2s))
$$
则(\ref{func3-10})的解为
\begin{align*}
w(x,t) &= \int_0^t \psi(x-a(t-2s)) \intd s \\
&= \frac{1}{2a} \int_{x-at}^{x+at} \psi(s) \intd s
\end{align*}
由定理(\ref{thm3-4})可以得到(\ref{func3-12})的解,
\begin{align}
    u(x,t) &= \frac{\partial}{\partial t} \left(\frac{1}{2a} \int_{x-at}^{x+at} \varphi(\xi) \intd \xi \right) + \frac{1}{2a} \int_{x-at}^{x+at} \psi(\xi) \intd \xi + \int_0^t \frac{1}{2a} \int_{x-a(t-s)}^{x+a(t-s)} f(\xi, s) \intd \xi \intd s \notag \\
    \label{func3-13}
    &= \frac{\varphi(x + at) + \varphi(x - at)}{2} + \frac{1}{2a} \int_{x-at}^{x+at} \psi(\xi) \intd \xi+ \frac{1}{2a} \int_0^t \int_{x-a(t-s)}^{x+a(t-s)} f(\xi, s) \intd \xi \intd s
\end{align}

\begin{thm}
令$Q = \mathbb{R} \times (0, +\infty)$,$\overline{Q} = \mathbb{R} \times [0, + \infty)$,如果$\varphi, \psi \in \mathcal{C}^2(Q) \cap \mathcal{C}^1(\overline{Q}), f \in \mathcal{C}^2(Q)$有界,则(\ref{func3-13})给出的$u(x,t) \in \mathcal{C}^2(Q) \cap \mathcal{C}^1(\overline{Q})$是(\ref{func3-12})的解。式(\ref{func3-13})又被称为\emph{达朗贝尔(D'Alembert)公式}。
\end{thm}

\begin{cor}
方程
\begin{align}
    u_{tt} - a^2 u_{xx} = f(x,t), \quad \quad x \in \mathbb{R}^n, t > 0
\end{align}
的解为
\begin{align}
    u(x,t) = F(x + at) + G(x-at) + \frac{1}{2a} \int_0^t \int_{x - a(t-s)}^{x + a(t-s)} f(\xi, s) \intd \xi \intd s
\end{align}
其中$F, G$是$\mathcal{C}^2(\mathbb{R})$上的任意函数。
\end{cor}

\begin{cor}
若$\varphi, \psi, f(\cdot, t)$关于$x$是以$l$为周期(或奇偶)函数,则(\ref{func3-12})的解也是关于$x$是以$l$为周期(或奇偶)函数。
\end{cor}

\subsection{波的传播}

设$f \equiv 0$,
\begin{enumerate}[(1)]
    \item $(x_0, t_0) \in Q$的决定区间$[x_0 - a t_0, x_0 + at_0]$
    \item 直线$x = c \pm at$在$x_0$的波是沿着斜率为$\pm \frac{1}{2a}$的直线传播,在沿直线上所有点的传播与$\varphi$一致,称$x = c \pm at$为Taylor弦振动的特征线。
    \item 任给一个区间$[c,d] \subset \mathbb{R}$,过$(c,0)$做斜率为$-\frac{1}{a}$的直线,过$(d,0)$做斜率为$\frac{1}{a}$的直线。 $\Longrightarrow$ 无穷梯形,该梯形中任意一点的波都会受到初值在$[c,d]$中值的影响。称该梯形为$[c,d]$的影响区域。
\end{enumerate}

\begin{eg}
\textbf{Chaper 2-6}. 试求解初值问题,
\begin{align}
    \begin{cases}
    u_{tt} - u_{xx} = 0, \quad \quad x \in \mathbb{R}, t >ax \\
    u|_{t=0} = u_0(x), u|_{t = ax} = u_1(x)
    \end{cases}
\end{align}
其中$a = \pm 1$。

解:设
\begin{align*}
    \begin{cases}
    \eta = t - ax \\
    \xi = \lambda_1 t + \lambda_2 x
    \end{cases}
\end{align*}
使得$u_{tt} - u_{xx} = u_{\xi\xi} - u_{\eta\eta}$。解得$\lambda_1 = a, \lambda_2 = -1$。

令$v(\eta, \xi) = u(x,y) = u(\frac{-\eta + a\xi}{1 - a^2}, \frac{\xi - a \eta}{1 - a^2})$,
\begin{align*}
    v_\xi &= u_x \frac{a}{1-a^2} + u_t \frac{1}{1-a^2} \\
    v_{\xi\xi} &= u_{xx}(\frac{a}{1-a^2})^2 + 2 u_{xt} \frac{a}{1-a^2} + u_{tt} (\frac{1}{1-a^2})^2 \\
    v_{\eta\eta} &= u_{xx} (\frac{-1}{1 - a^2})^2  + 2 u_{xt} \frac{a}{1 - a^2} + u_{tt} (\frac{-a}{1-a^2})^2
\end{align*}
$$
\Longrightarrow u_{tt} - u_{xx} = v_{\xi\xi} - v_{\eta\eta} = 0
$$
所以原问题化为
\begin{align}
    \begin{cases}
    v_{\xi\xi} - v_{\eta\eta} = 0 \\
    v(\eta, 0) = u_0(-\frac{a}{1-a^2}) \equiv \varphi(\eta)\\
    v_\xi |_{\xi = 0} = \frac{a}{1-a^2} u_0'(-\frac{a}{1-a^2}) + \frac{1}{1-a^2} u_1(-\frac{a}{1-a^2}) \equiv \psi(\xi)
    \end{cases}
\end{align}
由D'Alembert公式
$$
v(\eta, \xi) = \frac{\varphi(\eta - \xi) + \varphi(\eta + \xi)}{2} + \frac{1}{2} \int_{\eta - \xi}^{\eta + \xi} \psi(s) \intd s
$$
即
\begin{align}
    &u(x,t) = v(at - x, t - ax) \notag \\
    = &\frac{1}{2} \left[\varphi((a+1)(t-x)) + \varphi((a-1)(t+x))\right] + \frac{1}{2} \int_{(a-1)(t+x)}^{(a+1)(t-x)} \psi(s) \intd s
\end{align}
\end{eg}

\subsection{波方程的唯一性和稳定性}

\begin{align}
    \label{func3-21}
    \begin{cases}
    u_{tt} - a^2 \triangle u = f(x,t). \quad \quad x \in \mathbb{Q}:= \mathbb{R}^n \times (0, +\infty) \\
    u|_{t = 0} = \varphi(\Omega), u_t|_{t = 0} = \psi(\Omega)
    \end{cases}
\end{align}

\subsubsection{Gronwall不等式}.

\begin{align*}
    \frac{\intd g}{\intd t} \leq C g(t) + F(t), \quad \quad t \in [0, +\infty)
\end{align*}

\begin{align*}
    \Longrightarrow \quad \frac{d}{d t} (e^{-ct} g(t)) &\leq e^{-ct } F(t) \\
    e^{-ct} g(t) &\leq g(0) + \int_0^t e^{-cs} F(s) \intd s \\
    \Longrightarrow \quad g(t) &\leq e^{ct} g(0)+\int_0^t e^{c(t - s)} F(s) \intd s \\
\end{align*}
若$F$非减函数,
\begin{align*}
    g(t) &\leq e^{ct} g(0) + (e^{ct} -1) \frac{F(t)}{c} \\
    \frac{\intd g}{\intd t} &\leq c e^{ct} g(0) + e^{ct} F(t)
\end{align*}

\begin{thm} \textbf{Gronwall 不等式}.
若$g$在$[0, +\infty)$上连续且满足
\begin{align}
    \frac{\intd g}{\intd t} \leq C g(t) + F(t), \quad \quad t \in [0, +\infty)
\end{align}
且$F$非减函数,则
\begin{align}
    g(t) &\leq e^{ct} g(0) + (e^{ct} -1) \frac{F(t)}{c} \\
    \frac{\intd g}{\intd t} &\leq c e^{ct} g(0) + e^{ct} F(t)
\end{align}
\end{thm}

若$f$在$[x_0 - r, x_0 + r]$中连续,则
$$
\frac{d}{d r} \int_{x_0 - r}^{x_0 + r} f(t) \intd t = f(x_0 + r) - f(x_0 - r)
$$
设$n \geq 2$,$f \in \mathcal{C}(\overline{B(x_0, r)})$,
$$
\frac{d}{d r} \int_{B(x_0, r)} f(x) \intd x = \int_{\partial B(x_0 ,r)} f(s) \intd S_x
$$
由此得
\begin{align}
\frac{d}{d r}\int_{B(x_0, t - ar)} f(x) \intd x = -a \int_{\partial B(x_0, t - ar)} f(x) d S_x
\end{align}

\subsubsection{能量方法}

$$
\mathcal{C}(x_0, t_0) = \left\{(x, t) \in \overline{Q}, |x - x_0| < a(t_0 - t)\right\}
$$
瞬时能量:
$$
E(t) = \int_{B(x_0, a(t_0 - t))} u_t^2 + a^2 (|\triangledown u|^2) \intd x
$$
记
$$
K_\tau = \left\{(x,t) \in \mathcal{C}(x_0, t_0), 0 \leq t \leq \tau \right\}, \quad \quad (0 < \tau < t_0)
$$
$$
\hat{E}(\tau) = \int_0^\tau E(t) \intd t
$$
显然$\frac{d \hat{E}}{d \tau} = E(\tau)$。

\begin{thm}
\label{thm-power}
设$u \in \mathcal{C}^1(\overline{Q}) \cap \mathcal{C}^2(Q)$满足方程(\ref{func3-21}),则$\forall (x_0, t_0) \in Q, \forall 0 < \tau < t_0$,均有
\begin{align}
    \max \left\{E(\tau), \hat{E}(\tau)\right\} \leq e^\tau \left[\int_{B(x_0, at_0)} (\psi^2 + a^2 |\triangledown u|^2) \intd x + \int_{K_\tau} f^2(x,t) \intd x \intd t\right]
\end{align}
\end{thm}

\begin{proof}
在(\ref{func3-21})两边同乘$u_t$,$\forall 0 < \tau <t_0$,
$$
u_{tt} u_t - a^2 \triangle u \; u_t = u_t f,   \quad \quad \text{in} \; K_\tau
$$
两边积分
$$
\int_{K_\tau} u_{tt} u_t - \int_{K_\tau}  a^2 \triangle u \; u_t = \int_{K_\tau}  u_t f
$$
左边第一项:
\begin{align*}
    \int_{K_\tau} u_{tt} u_t &= \int_0^\tau \left( \int_{B(x_0, a(t_0 - t))}\frac{d}{d t} \frac{u_t^2}{2} \intd x\right) \intd t \\
    &= \int_0^\tau \left(\frac{d}{d t} \int_{B(x_0, a(t_0 - t))} \frac{u_t^2}{2} \intd x\right) \intd t +
    a \int_0^\tau \int_{\partial B(x_0, a(t_0 - t))} \frac{u_t^2}{2} \intd S_x \intd t \\
    &= \int_{B(x_0, a(t_0 - \tau))}\frac{u_t^2}{2} \intd x - \int_{B(x_0, at_0)} \frac{\psi^2}{2} \intd x +  a \int_0^\tau \int_{\partial B(x_0, a(t_0 - t))} \frac{u_t^2}{2} \intd S_x \intd t
\end{align*}
因为
$$
\triangle u = div(\triangledown u)
$$
$$
\triangledown u \triangledown u_t + u_t \triangle u = div(\triangledown u \; u_t)
$$
因此左边第二项
\begin{align*}
    - \int_{K_\tau}  a^2 \triangle u \; u_t &= - a^2 \int_0^\tau  \left(\int_{B(x_0, a(t_0 - t)} \triangle u \; u_t \intd x\right) \intd t \\
    &= -a^2 \int_0^\tau \intd t \left(\int_{B(x_0, a(t_0 - t))} div(\triangledown u \; u_t) \intd x- \int_{B(x_0, a(t_0 - t))} \triangledown u \triangledown u_t \intd x\right) \quad \text{(变限积分求导)}\\
    &= -a^2 \int_0^\tau \intd t \int_{\partial B(x_0, a(t_0 - t))} \frac{\partial u}{\partial \vec{n}} u_t \intd S_x + a^2 \int_0^\tau \intd t \int_{B(x_0, a(t_0 - t)} \frac{d}{d t} \frac{|\triangledown u|^2}{2} \intd x \\
    &= -a^2 \int_0^\tau \intd t \int_{\partial B(x_0, a(t_0 - t))}  \triangledown u \; \vec{n} \; u_t \intd S_x + \int_{B(x_0, a(t_0 - \tau))} a^2 \frac{|\triangledown u|^2}{2} \intd x  \\
    &- \int_{B(x_0, at_0)} a^2 \frac{|\triangledown \varphi|^2}{2} \intd x  +  a^3 \int_0^\tau  \int_{\partial B(x_0, a(t_0 - t))} \frac{|\triangledown u|^2}{2} \intd S_x
\end{align*}
因为
$$
|\triangledown u \; \vec{n} \; u_t| \leq \frac{a}{2} |\triangledown u|^2 + \frac{1}{2a} |u_t \vec{n}|^2
$$
代入原式,根据$2ab \leq a^2 + b^2$
\begin{align*}
    E(\tau) &\leq E(0) + \int_{K_\tau}f u_t \intd x \intd t \leq E(0) + \int_{K_\tau} \frac{f^2}{2} \intd x \intd t + \int_{K_\tau} \frac{u_t^2}{2} \intd x \intd t \\
    \therefore \; \frac{d \hat{E}(\tau)}{d t} &\leq E(0) + \int_{K_\tau} \frac{f^2}{2} \intd x \intd t + \hat{E}(\tau)
\end{align*}
令$F = E(0) + \int_{K_\tau} \frac{f^2}{2} \intd x \intd t$, 由Gronwall不等式
\begin{align}
    \hat{E}(\tau) &\leq (e^\tau - 1)F(\tau) \\
    E(\tau) &\leq e^\tau F(\tau)
\end{align}
\end{proof}

\begin{cor}
问题(\ref{func3-21})的解在$\mathcal{C}^1(\overline{Q}) \cap \mathcal{C}^2(Q)$中只有一个解,且解是稳定的。
\end{cor}

\subsection{半无界问题的解法}

$Q_h = (0, +\infty) \times (0, +\infty), \overline{Q_h} = [0, +\infty) \times [0, +\infty)$,

\begin{align}
\label{func3-30}
    \begin{cases}
    u_{tt} - a^2 u_{xx} = f(x,t), \quad \quad &(x,t) \in Q_h \\
    u|_{t = 0} = \varphi(x), u_t|_{t = 0} = \psi(x), \quad \quad &x > 0 \\
    u_x|_{x = 0} = g(t), t > 0 \quad &\left(\text{or} \; u|_{x=0} = g(t)\right)
    \end{cases}
\end{align}

\textcolor{red}{若 $u|_{x = 0} = 0$,做奇延拓;若$u_x|_{x=0} = 0$,做偶延拓。}

\textbf{解法}
\begin{enumerate}[(1)]
  \item 化为齐次边界条件

令$u(x,t) = v(x,t) + w(x,t)$, s.t.
$$
v_x|_{x = 0} = 0 \Longleftrightarrow w_x|_{x = 0} = g(t)
$$
选择$w(x,t) = + x g(t) ( + f(t))$。

  \item 问题化为
\begin{align}
\label{func3-31}
\begin{cases}
    v_{tt} - a^2 v_{xx} = f(x,t) - x g''(t) := f_1(x,t) \\
    v|_{t = 0} = \varphi(x) - xg(0) := \varphi_1(x), v_t|_{t = 0} = \psi(x) - xg'(x) := \psi_1(x) \\
    v_x|_{x = 0} = 0
\end{cases}
\end{align}

  \item 将$f_1, \varphi_1, \psi_1$关于$x$做偶延拓。

  \item 解出(\ref{func3-31}),利用D'Alebermt公式。

  $$
  v(x,t) = \frac{1}{2} \left(\hat{\varphi}_1(x+at) + \hat{\varphi}_1(x-at)\right) + \frac{1}{2a} \int_{x-at}^{x+at} \hat{\psi}_1(\xi) \intd \xi + \frac{1}{2} \int_0^t \int_{x - a(t - \xi)}^{x + a(t-\xi)} \hat{f}_1(s,\xi) \intd \xi \intd s
  $$

  \item 将$u(x,t) = v(x,t) + xg(t)$限制在$Q_h$上,就得到(\ref{func3-30})的解。
  \begin{enumerate}[a)]
    \item 当$x \geq at$时,
    $$
    u(x,t) = xg(t) + \frac{1}{2} \left(\varphi_1(x+at) + \varphi_1(x-at)\right) + \frac{1}{2a} \int_{x-at}^{x+at} \psi_1(\xi) \intd \xi + \cdots
    $$
    \item $x < at$时,$\varphi_1(x-at) \longrightarrow \varphi_1(at - x)$,积分变成两部分。
  \end{enumerate}

  \item 相容性条件(在0点处有两个方向和条件,解的过程中无体现):要求$u \in \mathcal{C}^2(\overline{Q_h}), \varphi'(0) = g(0), g'(0) = \psi'(0)$,
  \begin{align}
  \label{func3-32}
      \begin{cases}
      u_{tx} = \psi'(x), u_{xt} = g'(t), u_x = \varphi'(x) \\
      u_{xtt} - a^2 u_{xxx} = \frac{\partial f}{\partial x} (x,t) \\
      g''(0) - a^2 \varphi'''(0) = \frac{\partial f}{\partial x}(0,0), \quad \quad u \in \mathcal{C}^3
      \end{cases}
  \end{align}

\end{enumerate}

\begin{thm}
\label{thm-power2}
设$g,\psi \in \mathcal{C}^2 [0, + \infty), \varphi \in \mathcal{C}^3[0, +\infty), f \in \mathcal{C}^1(\overline{Q_h})$,且满足相容性条件(\ref{func3-32}),则问题(\ref{func3-30})存在唯一解$\in \mathcal{C}^2(\overline{Q_h})$,$u(x,t) = v(x,t) + xg(t)$。
\end{thm}

\begin{proof}
\begin{enumerate}[(1)]
  \item 验证
  \item 唯一性,Cauchy问题有唯一性,但因为中间有奇偶延拓,不能说明延拓前解的唯一,完全类似定理(\ref{thm-power})。
\end{enumerate}
\end{proof}

\subsection{高维波动方程的解法}

$\forall x \in \mathbb{R}^n, u \in \mathcal{C}(\mathbb{R}^n)$,
  \begin{align*}
      u(x) &= \lim_{r \rightarrow 0^+} \int_{B(x,r)} u(y) \intd y = \lim_{r \rightarrow 0^+} \int_{B(x,r)} u(y) \intd y \frac{1}{|B(x,r)|} \\
      &= \lim_{r \rightarrow 0^+} \frac{1}{|\partial B(x,r|} \int_{\partial B(x,r)} u(y) \intd S_y := I(x,r,u)
  \end{align*}
  先求$f(r) = I(x,r,u)$的表达式,再令$r \rightarrow 0^+$,得到$u(x)$。

\subsubsection{球面平均法,$n=3$为例}(适合所有$n \geq 1$)

  以$n = 3$为例子,任取$x \in \mathbb{R}^3, F \in \mathcal{C}^2(R^3)$,记$I(r) = \frac{1}{4\pi r^2} \int_{\partial B(x,r)} F(y) \intd S_y$,则有
  $$
  \triangle_x(r I(r)) = \frac{\partial^2}{\partial r^2} (r I(r)), \forall r > 0
  $$
  下面我们来证明这个结论:
  \begin{align*}
      &r^2 I(r) = \frac{1}{4\pi} \int_{\partial B(0,r)} F(x+z) \intd S_z \quad \quad (z = y - x) \\
      \xrightarrow[]{\text{对}r\text{积分}} &\int_0^r S^2 I(S) \intd S = \frac{1}{4\pi} \int_{B(0,r)} F(x+z) \intd z \\
      \xrightarrow[]{\text{对}x\text{求Laplace}} &\triangle_x \int_0^r S^2 I(S) \intd S = \frac{1}{4 \pi} \int_{B(0,r)} \triangle_x F(x+z) \intd z
  \end{align*}
  \begin{align*}
      \triangle_x \int_0^r S^2 I(S) \intd S = &\frac{1}{4\pi} \int_{B(0,r)} \triangle_z F(x+z) \intd z \\
      = &\frac{1}{4\pi} \int_{B(0,r)} \text{div}_z \triangledown_z F(x+z) \intd z \\
      = &\frac{1}{4 \pi} \int_{\partial B(0,r)} \triangledown_z F(x+z) \vec{n} \intd S_z \\
      = &\frac{1}{4\pi} \int_{\partial B(0,r)} \triangledown_zF(x+z) \frac{\vec{z}}{|z|} \intd S_z \\
      \xlongequal[]{z=ry} &\frac{r^2}{4\pi} \int_{\partial B(0,1)}\triangledown_z F(x+ry) \frac{\vec{y}}{|1|} \intd S_y, \quad (dS_z = r^2 d S_y) \\
      = &\frac{r^2}{4\pi} \int_{\partial B(0,1)} \frac{\partial }{\partial r} F(x + ry) \intd S_y \\
      = &r^2 \frac{\partial }{\partial r} \frac{1}{4 \pi} \int_{\partial B(0,1)} F(x+ry) \intd S_y \\
      \xlongequal[]{z=x+ry} &r^2 \frac{\partial }{\partial r} \left(\frac{1}{4 \pi r^2} \int_{\partial B(x,r)} F(z) \intd S_z\right) \\
      = &r^2 \frac{\partial }{\partial r} I(r)
  \end{align*}
$$
\therefore \; \triangle_x r^2 I(r) \xlongequal[]{\frac{\partial}{\partial r}} \frac{\partial}{\partial r}\left(r^2 \frac{\partial}{\partial r}I(r)\right) = 2 r \frac{\partial}{\partial r} I(r) + r^2 \frac{\partial^2 }{\partial r^2} I(r)
$$
$$
\triangle_x r(I(r)) = 2\frac{\partial}{\partial r} I(r) + r \frac{\partial^2 }{\partial r^2} I(r) = \frac{\partial^2 }{\partial r^2} (I(r)r)
$$
则有
$$
\triangle_x (rI(r)) = \frac{\partial^2 }{\partial r^2} (I(r)r), \quad \quad \forall r > 0
$$

考虑
\begin{align}
\begin{cases}
    u_{tt} - a^2 \triangle u = f(x,t), \quad \quad &x \in \mathbb{R}^3 \\
    u|_{t = 0} = \varphi(x), u_t|_{t = 0} = \psi(x), \quad &t > 0
\end{cases}
\end{align}

由Duhamel,可以设$f, \varphi \equiv 0$,记$I(r,t,x) = \frac{1}{4\pi r^2} \int_{\partial B(x,r)} u(y,t) \intd y$,令$w(x,t,r) = rI(x,r,t)$,

\begin{align*}
    a^2 \frac{\partial^2 }{\partial r^2} w(x,t,r) = &a^2 \triangle_x w(x,r,t) \\
    = &a^2 r \triangle_x I(x,t,r) \\
    = &\frac{a^2 r}{4\pi r^2} \triangle_x \int_{\partial B(x,r)} u(y,t) \intd S_y \\
    \xlongequal[]{y=x+z} &\frac{a^2 r}{4 \pi r^2} \triangle_x \int_{\partial B(0,r)} u(x+z, t) \intd S_z \\
\end{align*}
\begin{align*}
\therefore \;
    a^2 \frac{\partial^2 }{\partial r^2} w(x,t,r)= &\frac{a^2 r}{4 \pi r^2} \int_{\partial B(0,r)} \triangle_x u(x+z, t) \intd S_z \\
    = &\frac{1}{4 \pi r} \int_{\partial B(0,r)} u_{tt} \intd S_z \\
    = &\left[r I(x,t,r)\right]_{tt} = w_{tt}
\end{align*}
$$
\Rightarrow w_{tt} - a^2 w_{rr} = 0
$$

对问题:
\begin{align}
\label{Q3-1}
    \begin{cases}
    u_{tt} - a^2 \triangle u = f,   \quad \quad \text{in} \; \mathbb{R}^n \times^+ (0, +\infty) \\
    u|_{t = 0} = \varphi, u_t|_{t = 0} = \psi
    \end{cases}
\end{align}
只要求
\begin{align}
    \begin{cases}
    v_{tt} - a^2 \triangle v = 0 \\
    v|_{t = 0} = 0, v_t|_{t = 0} = \psi
    \end{cases}
\end{align}
令$w(x,r,t) = r I(x,r,v(x,t))$,则$t > 0, r > 0$,
\begin{align}
    \begin{cases}
    w_{tt} - a^2 w_{rr} = 0 \\
    w|_{t = 0} = 0 \\
    w_t|_{t = 0} = r I(x,r,\psi) \\
    w_{r = 0} = \lim\limits_{r \rightarrow 0+} u(x,r,v(\cdot,t)) = \lim\limits_{r \rightarrow 0+} \frac{1}{4 \pi r} \iint\limits_{\partial B(x,r)} v(r,t) \intd S_y = 0
    \end{cases}
\end{align}
因此
\begin{align}
\label{Q-high}
    \begin{cases}
    w_{tt} - a^2 w_{rr} = 0, \quad r > 0, t > 0 \\
    w|_{t = 0} = 0, w_t|_{t = 0} = rI(x,r,\psi) \\
    w|_{r = 0} = 0
    \end{cases}
\end{align}
由定理(\ref{thm-power2}),对于半无界问题(\ref{Q-high}),因为$r \rightarrow 0^+$,可以认为$0 \leq r \leq at$:
\begin{align*}
    w(x,r,t) &= \frac{1}{2a} \int_{at-r}^{at+r} \xi I(x,\xi, \psi) \intd \xi \\
    \therefore \quad v(x,t) &= \lim\limits_{r \rightarrow 0^+} I(x,r,v(\cdot,t)) \\
    &= \lim\limits_{r \rightarrow 0^+} \frac{1}{a} \frac{1}{2r} \int_{at-r}^{at+r} \xi I(x,\xi, \psi) \intd \xi \\
    &= \frac{1}{a} \xi I(x,\xi, \psi)\bigg|_{\xi = at} = t I(x,at,\psi) \\
    &= \frac{1}{4\pi a^2 t} \iint\limits_{\partial B(x,at)} \psi(y) \intd S_y
\end{align*}
由Duhamel原理,问题(\ref{Q3-1})的解为
\begin{align}
\label{A3-1}
    u(x,t) = &\frac{\partial}{\partial t} \left(\frac{1}{4 \pi a^2 t} \iint\limits_{\partial B(x,at)} \varphi(y) \intd S_y\right) + \frac{1}{4 \pi a^2 t} \iint\limits_{\partial B(x,at)} \psi(y) \intd S_y \notag \\
    + &\int_0^t \left(\frac{1}{4\pi a^2(t-s)}\iint\limits_{\partial B(x,a(t-s))}f(y,s) \intd S_y \right) \intd s
\end{align}


\subsubsection{降维法,以$n=2$为例}

\begin{align}
\label{Q3-2}
    \begin{cases}
    u_{tt} - a^2 (u_{xx} + u_{yy}) = f(x,y,t), \quad \quad (x,y) \in \mathbb{R}^2, t > 0\\
    u|_{t = 0} = \varphi(x,y), \; \; u_t|_{t = 0} = \psi(x,y)
    \end{cases}
\end{align}
令$\overline{u}(x,y,z,t) = u(x,y,t)$,得到
\begin{align}
    \begin{cases}
    \overline{u}_{tt} - a^2 (\overline{u}_{xx} + \overline{u}_{yy}+\overline{u}_{zz}) = \overline{f} \\
    \overline{u}|_{t = 0} = \overline{\varphi}, \; \; \overline{u}_t|_{t = 0} = \overline{\psi}
    \end{cases}
\end{align}
由(\ref{A3-1})得到
\begin{align}
    \overline{u}(x,y,z,t) = \frac{\partial}{\partial t} \left(\frac{1}{4 \pi a^2 t} \iint\limits_{\partial B(x,,y,z,at)} \overline{\varphi}(x_1,y_1,z_1) \intd S_\xi\right) + \cdots
\end{align}
其中在上半球面上
$$
\intd S = \sqrt{1 + z_x^2 + z_y^2} \intd x \intd y = \frac{r}{\sqrt{r^2 - (x-x_1)^2) - (y - y_1)^2}} \intd x \intd y
$$
因此
\begin{align*}
\iint\limits_{\partial B(x,y,z,r)} F(x,y) \intd S_{xy} = 2 \iint\limits_{\Sigma(x,y,r)} F(x,y) \intd S_{xy} = 2 \iint\limits_{\Sigma(x,y,r)} \frac{F(x,y)r}{\sqrt{r^2 - (x-x_1)^2 - (y - y_1)^2}} \intd x \intd y \\
\left(\text{对}\; z = \sqrt{r^2 - (x-x_1)^2 - (y - y_1)^2} \; \text{求导}\right)
\end{align*}
因此问题(\ref{Q3-2})的解为
\begin{align}
\label{A3-2}
    u(x,y,t) = &\frac{1}{2\pi a}\frac{\partial}{\partial t} \left( \iint\limits_{\Sigma(at)} \frac{\varphi(x_1,y_1)}{\sqrt{(at)^2 - (x-x_1)^2 - (y - y_1)^2}}\intd x_1 \intd y_1\right) \notag \\
    + &\frac{1}{2\pi a} \iint\limits_{\Sigma(at)} \frac{\psi(x_1,y_1)}{\sqrt{(at)^2 - (x-x_1)^2 - (y - y_1)^2}}\intd x_1 \intd y_1 \notag \\
    + &\frac{1}{2\pi a} \int_0^t \left(\iint\limits_{\Sigma(a(t-s)} \frac{f(x_1,y_1,s)}{\sqrt{a^2(t-s)^2 - (x-x_1)^2 - (y - y_1)^2}}\intd x_1 \intd y_1\right) \intd s
\end{align}
where
\begin{align*}
\Sigma(at) = \left\{(x_1,y_1) \in \mathbb{R}^2 \Bigg| |x_1 - x|^2 + |y_1 - y|^2 < (at)^2 \right\}
\end{align*}

\begin{thm}
设$n=3$或$n=2$,$\varphi \in \mathcal{C}^3(\mathbb{R^n}),\psi \in \mathcal{C}^2(\mathbb{R^n}), f \in \mathcal{C}^2(\mathbb{R}^n \times [0,+\infty))$,则由(\ref{A3-1})和(\ref{A3-2})给出的$u \in \mathcal{C}^2(\mathbb{R}^n \times (0,+\infty)) \cap \mathcal{C}^2(\mathbb{R}^n \times [0,+\infty))$,且分别满足(\ref{Q3-1})和(\ref{Q3-2})。
\end{thm}

\subsection{Sturm-Liouville问题}

\subsubsection{问题的提出}

设$k,p > 0, q \geq 0$,$k \in \mathcal{C}^1[0, +\infty), p,q \in \mathcal{C}[a,b]$,求函数$y(x)$和常数$\lambda$,使得
\begin{align}
\label{Q3-3}
    \begin{cases}
    \frac{d}{d x} \left(k(x) \frac{d y}{d x}\right) + \lambda p(x) y - q(x) y = 0 \\
    \left(-\alpha_1 y' + \alpha_2y \right) \bigg|_{x=a} = 0, \; \; \left(\beta_1 y' + \beta_2 y \right) \bigg|_{x=b} = 0
    \end{cases}
\end{align}
其中$\alpha, \beta$是非负常数,$\alpha_1^2 + \alpha_2^2 \neq 0,\beta_1^2 + \beta_2^2 \neq 0$。\\

\subsubsection{有解$\lambda$的必要条件}

\begin{definition}
若$(y,\lambda)$满足(\ref{Q3-3}),且$y$不恒为0,则称$\lambda$为(\ref{Q3-3})的一个\textbf{特征值},$y$为与$\lambda$对应的\textbf{特征函数}。
\end{definition}

\subsubsection{一些结论}

\begin{enumerate}[(a)]
    \item $\lambda \geq 0$,且若$q(x) > 0$,或者$\alpha_2 + \beta_2 >0$,则$\lambda > 0$。\textcolor{red}{特征值非负}

将(\ref{Q3-3})同乘$y$,再在$[a,b]$上积分,由分部积分得
\begin{align*}
    &k y'y \bigg|_{x=a}^{x=b} - \int_a^b ky'^2 \intd x + \int_a^b \left(\lambda p(x) - q(x)\right) y^2 = 0 \\
    \Longrightarrow & \lambda = \frac{1}{\int_a^b py^2 \intd x} \left(\int_a^b (ky'^2 + qy^2) \intd x - k y' y \bigg|_{x=b} + ky'y\bigg|_{x=a} \right)
\end{align*}
\begin{align*}
\because \quad &\beta_1 y'(b) - \beta_2 y(b) = 0 \\
&\begin{cases}
\beta_1 y'(b) y(b) - \beta_2 y^2(b) = 0 \\
\beta_2 y'(b) y(b) + \beta_1 y'^2(b) = 0
\end{cases} \\
\Rightarrow \quad & y'(b) y(b) = - \frac{\beta_2 y^2(b) + \beta_1 y'^2(b)}{\beta_1 + \beta_2} \leq 0
\end{align*}
且当$q(x) > 0$或者$\beta_2 + \alpha_2 > 0$ \; $\Longrightarrow$ \; $\lambda > 0$

\item 若$(y_i, \lambda_i)$是问题(\ref{Q3-3})的解,($i = 1,2$),且$\lambda_1 \neq \lambda_2$,\textcolor{red}{特征函数正交}:
\begin{align}
    \frac{d}{dx} \left(k(x) \frac{dy_1}{d x}\right) + (\lambda_1 p(x) - q(x))y_1 = 0 \\
    \frac{d}{dx} \left(k(x) \frac{dy_1}{d x}\right)y_2 + (\lambda_1 p(x) - q(x))y_1y_2 = 0
\end{align}
Therefore,
\begin{align*}
    &k(x) y_1' y_2 \bigg|_{x=a}^{x=b} - \int_a^b \left(k(x) y_1' y_2' + (\lambda_1 p - q)y_1y_2\right) \intd x = 0 \\
    \Rightarrow \quad &k y_1'y_2 |_{x=b} - k y_1' y_2 |_{x=a} + \lambda_1 \int_a^b p(x)y_1 y_2 \intd x - \int_a^b q(x) y_1 y_2 \intd x - \int_a^b k(x) y_1' y_2' = 0
\end{align*}
对$y_1,y_2$互换位置,得到
$$
k y_2'y_1 |_{x=b} - k y_2' y_1 |_{x=a} + \lambda_2 \int_a^b p(x)y_1 y_2 \intd x - \int_a^b q(x) y_1 y_2 \intd x - \int_a^b k(x) y_1' y_2' = 0
$$
上两式相减得到
$$
\left(\lambda_1 - \lambda_2\right) \int_a^b p(x)y_1(x) y_2(x) \intd x = 0
$$
因为$\lambda_1 \neq \lambda_2$,所以
$$
\int_a^b p(x)y_1(x) y_2(x) \intd x = 0   \quad \quad \text{正交}
$$

\end{enumerate}

\subsubsection{定理(\ref{thm-character})}

\begin{thm}
\label{thm-character}
问题(\ref{Q3-3})有如下性质:
\begin{enumerate}[(i)]
    \item 所有特征值$\lambda \geq 0$,且当$q(x) = 0$或者$\alpha_2 + \beta_2 > 0$时,$\lambda > 0$;
    \item 不同特征值$\lambda_i, i = 1,2$对应的特征函数$y_i(x), i = 1,2$正交,i.e. 若$\lambda_1 \neq \lambda_2$,则
    \begin{align}
        \int_a^b p(x)y_1(x)y_2(x) \intd x = 0
    \end{align}

    \item 所有特征值构成可数集,按照大小排列为$0 \leq \lambda_1 < \lambda_2 < \cdots < \lambda_{n+1}$;

    \item 若$f \in L^2[a,b] = \left\{f: \int_a^b f^2 \intd x < + \infty\right\}$,
    $$
    f_n(x) = \sum_{i=1}^n C_n X_n(x)
    $$
    where
    $$
    C_k = \frac{\int_a^b p(x) f(x) X_k(x) \intd x}{\int_a^b p(x) X_k^2(x) \intd x}
    $$
    $X_n(x)$是$\lambda_n$对应的特征子空间的正交基,则
    \begin{align}
        \lim\limits_{n \rightarrow + \infty} \int_a^b (f(x) - f_n(x))^2 \intd x = 0, \quad \quad \text{平方收敛}
     \end{align}

     \item 若$f(x)$在$[a,b]$上连续可微,且$f^-(b) = f^+(a)$,则级数$f(x) = \sum_{n=1}^\infty C_n X_n (x)$绝对一致收敛。
\end{enumerate}
\end{thm}

\begin{proof}
(iii)(v)的证明,见彼得罗夫斯基《偏微分方程》,萧树铁等译。
\end{proof}

\textbf{注:}如果问题(\ref{Q3-3})中边界条件换位$y \bigg|_{x=a} = y \bigg|_{x = b} $,或者$- \infty < y \bigg|_{x = a}, y \bigg|_{x=b} < + \infty$,以上定理也正确。

\begin{eg} \textbf{习题2-22(5).} 求解特征值问题$y'' + \lambda y = 0, 0 < x < l$,$y'(0) = y'(l) + h y(l) = 0 (h > 0)$.

\textbf{解:} 由定理(\ref{thm-character}),我们知道$\lambda > 0$。再由方程ODE对应的特征方程,$t^2 + \lambda = 0$。不妨设$\lambda = \beta^2, \beta > 0, \; \Rightarrow \; t = \pm \beta i$。

ODE的解,$A \cos \beta x + B \sin \beta x$,再由边界条件$B \cos \beta 0 = 0$得,$B = 0$。因此,
\begin{align*}
-A \beta \sin \beta l + h A \cos \beta l = 0 \\
\because A \neq 0, \tan \beta l = \frac{h}{\beta}
\end{align*}
$\beta$是方程$\tan x l = \frac{h}{x}$的解,设$\tan \beta l = \frac{h}{\beta}$的正解为$\beta_1, ..., \beta_n$,则原问题解为($\beta_i, \cos \beta_i x$)$i = 1,2,...$。
\end{eg}

\subsection{初边值问题的求解}

\subsubsection{初边值问题}

Fourier方法,分离变量

\textbf{(1) 齐次问题}
    \begin{align}
    \label{Q3-10-1}
        \begin{cases}
        u_{tt} - a^2 u_{xx} + cu = 0, \quad &0 < x < l, c \geq 0 \\
        u(0,t) = u_x(l, t) = 0, \quad &t \geq 0 \\
        u(x,0) = \varphi(x), u_t \bigg|_{t = 0} = \psi(x), \quad &0 \leq x \leq l
        \end{cases}
    \end{align}

    Step 1: 变量分离。

    令$u(x,t) = X(x)T(t)$,代入(\ref{Q3-10-1}),得到
    \begin{align*}
        T''X - a^2 X'' T + c TX = 0 \\
        \frac{T''}{T}(t) = \frac{a^2 X'' - cX}{X}(x) = - \lambda
    \end{align*}
    \begin{align}
    \label{Q3-10-2}
        \begin{cases}
        T'' + \lambda a^2 T = 0, \quad &t > 0 \\
        X'' + (\lambda - \frac{c}{a^2})X = 0, \quad &0 < x < l
        \end{cases}
    \end{align}
    $$
    \Rightarrow \; X(0) = 0 = \mathbb{X}'(l)
    $$
    因此,
    \begin{align}
    \label{Q3-10-3}
        \begin{cases}
        \mathbb{X}'' + (\lambda - \frac{c}{a^2}) X  = 0, \quad 0 < x < l \\
        X(0) = X'(l) = 0
        \end{cases}
    \end{align}

    Step 2: 解S-L问题(\ref{Q3-10-3})

    由定理(\ref{thm-character}),$\lambda - \frac{c}{a^2} > 0$,设$\lambda - \frac{c}{a^2} = \beta^2, \beta > 0$
    $$
   t^2 + \beta^2 = 0 \Rightarrow t = \pm \beta i
    $$
    因此
    $$
    X(x) = A \cos \beta x + B \sin \beta x
    $$
    \begin{align*}
        \begin{cases}
            A = 0 \\
            B \beta \cos \beta l = 0
        \end{cases} \Rightarrow
        \begin{cases}
        \beta l = k \pi + \frac{\pi}{2} \\
        \beta_k = \frac{(2k+1)\pi}{2l}
        \end{cases}
    \end{align*}
    $$
    \because B \neq 0, X_k(x) = \sin \beta_k x, \quad k = 0,1,2,...
    $$

    Step 3: 代入(\ref{Q3-10-2})

    $$
    T_k(l) = A_k \cos \lambda_k t + B_k \sin \lambda_k t
    $$
    求得
    $$
    u_k(x,t) = \mathbb{X}(x) T_k(t), \quad k = 0,1,2,...
    $$
    此时已经满足(\ref{Q3-10-1})的前三个式子。

    Step 4: (叠加)

    求形如
    \begin{align}
    \label{equa3-1}
        u(x,t) = \sum_{k=0}^\infty X_k(x)T_k(x) = \sum_{k=0}^\infty \left(A_k \cos \lambda_k t + B_k \sin \lambda_k t \right) \sin \beta_k(x)
    \end{align}
    的方程,满足初始条件,只需要
    \begin{align*}
        \begin{cases}
        \sum_{k=0}^\infty A_k \sin \beta_k x = \varphi(x)\\
        \sum_{k=0}^\infty \lambda_k B_k \sin \beta_k x = \psi(x)
        \end{cases}
    \end{align*}
    只需要
    \begin{align}
    \label{equa3-2}
        A_k &= \frac{2}{l} \int_0^l \varphi(x) \sin \beta_k x \intd x \notag\\
        B_k &= \frac{2}{\lambda_k l} \int_0^l \psi(x) \sin \beta_k x \intd x
    \end{align}
    因此,(\ref{Q3-10-1})的解由(\ref{equa3-1})和(\ref{equa3-2})给出。

    Step 5: 检验是解。

\begin{thm}
设$\psi \in \mathcal{C}^3[0,l]$,$\varphi \in \mathcal{C}^2[0,l]$,且满足
\begin{align}
    \begin{cases}
        \varphi(0) = \psi(0) = \varphi''(0) = 0 \\
        \varphi(l) = \psi'(l) = \psi'''(l) = 0
    \end{cases}
\end{align}
则由(\ref{equa3-1})和(\ref{equa3-2})给出$u \in \mathcal{C}^2[0,l] \times [0,\infty)$,且满足(\ref{Q3-10-1})。
\end{thm}

\begin{note}
$\forall f(x) \in L_2[0,l]$,可以按照特征函数$\{X_n(x)\}$展开,
\begin{align*}
    f(x) = \sum_{k=1}^\infty C_n X_n(x) \\
    C_n = \frac{\int_0^l f(x)X_n(x) \intd x}{\int_0^l X_n^2(x) \intd x}
\end{align*}
\end{note}

\textbf{(2) 把非齐次边界条件化为齐次}


\begin{enumerate}[(a)]
    \item $u(0,t) = q_1(t), u(l,t) = q_2(t)$。

    引入辅助函数$v(x,t) = u(x,t) + w(x,t)$,s.t.$w(0,t) = -q_1(t),w(l,t) = -q_2(t)$,
    $$
    \Rightarrow \; \; w(x,t) = -q_1(t) + \frac{q_1(t) - q_2(t)}{l}x
    $$

    \item $u(0,t) = q_1(t), u_x(l,t) = q_2(t)$。

    令$v(x,t) = u(x,t) + w(x,t)$,s.t. $w(0,t) = -q_1(t),w_x(l,t) = -q_2(t)$,
    $$
    \Rightarrow \; \; w(x,t) = -q_1(t) - q_2(t)x
    $$

    \item $u_x(0,t) = q_1(t), u_x(l,t) = q_2(t)$

    令$v(x,t) = u(x,t) + w(x,t)$,s.t. $w_x(0,t) = -q_1(t),w_x(l,t) = -q_2(t)$,用二次函数找出,即试图
    $$
    w(x,t) = a(t)x^2 + b(t)x + c(t)
    $$
    $$
    \Rightarrow \; \; a(t) = \frac{q_1(t) - q_2(t)}{2l}, b(t) = -q_1(t), c(t) = 0
    $$

    \item 其他情形化为前3种情况

    $$
    u_x + \alpha u = q(t)  \longrightarrow (u e^{\alpha x})_x = q(t) e^{\alpha x}
    $$

    step 1: 解对应的齐次方程对应的S-L问题,得到特征函数为$\{\mathbb{X}_n(x)\}_{n=1}^{\infty}$(分离变量)

    step 2: 令解为$u(x,t) = \sum_{n=1}^\infty T_n(t) \mathbb{X}_n(x) $,则$u(x,t)$仍旧满足边界条件。

    为使得其满足$\qed u = f(x,t), u(x,0) = \varphi(x), u_t(x,0) = \psi(x)$,将$f,\varphi,\psi$按照$\{\mathbb{X}_n(x)\}_{n=1}^\infty$展开

    $f(x,t) = \sum_{n=1}^\infty f_n(t)\mathbb{X}_n(x), \varphi(x,t) = \sum_{n=1}^\infty \varphi_n(t)\mathbb{X}_n(x), \psi(x,t) = \sum_{n=1}^\infty \psi_n(t)\mathbb{X}_n(x)$,

    代入$\{\mathbb{X}_n(x)\}_{n=1}^{\infty}$,解出$\mathbb{X}_n(x)$。
\end{enumerate}

\begin{eg} 求解初边值问题
\begin{align}
    \begin{cases}
    u_{tt} - u_{xx} + u = A, \quad \quad &0 < x < l, t > 0 \\
    u(0,t) = -A, u_x(l,t) = 0, & t > 0 \\
    u\bigg|_{t = 0} = 0, u_t\bigg|_{t = 0} = Bx, & 0 \leq x \leq l
    \end{cases}
\end{align}

解:令$v(x,t) = u(x,t) + A$,则
\begin{align*}
    \begin{cases}
    v_{tt} - v_{xx} + v = 2A \\
    v(0,t) = v_x(l,t) = 0 \\
    v\bigg|_{t = 0} = A, v_t \bigg|_{t = 0} = Bx
    \end{cases}
\end{align*}
先解齐次方程对应的S-L问题,将$v(x,t) = \mathbb{X}(x)T(t)$代入对应的齐次方程,
\begin{align*}
    \begin{cases}
    T'' \mathbb{X} - \mathbb{X}'' T + \mathbb{X} T = 0 \\
    \mathbb{X}(0) = \mathbb{X}'(l) = 0
    \end{cases}
\end{align*}
$$
\Rightarrow \; \; \frac{T''}{T} = \frac{\mathbb{X}'' - \mathbb{X}}{\mathbb{X}} = - \lambda
$$
由定理(\ref{thm-character}),$\lambda > 0$,$\lambda_n = 1 + \beta_n^2, \beta_n = \frac{(2n+1)\pi}{2l}$。令$v(x,t) = \sum_{n=1}^\infty T_n(t)\mathbb{X}_n(x) $,代入
\begin{align*}
    \begin{cases}
    \sum_{n=1}^\infty \mathbb{X}_n(x) \left(T''_n(t) + \lambda_n T_n\right) = 2A \\
    \sum_{n=1}^\infty \mathbb{X}_n(x) T_n(0) = A \\
    \sum_{n=1}^\infty \mathbb{X}_n(x) T_n'(x) = Bx
    \end{cases}
\end{align*}
\begin{align*}
\therefore \quad \quad
    \begin{cases}
    A = \sum_{n=1}^\infty A_n \mathbb{X}_n(x), \\
    A_n = \frac{2}{l} \int_0^l A \sin \beta_n x = \frac{4A}{(2n+1)} \\
    B = \sum_{n=1}^\infty B_n \mathbb{X}_n(x), \\
    B_n = \frac{2}{l} \int_0^l B_x \sin \beta_n x = \frac{(-1)^n  2 Bl}{((2n+1)\pi)^2}
    \end{cases}
\end{align*}
比较$\mathbb{X}_n(x)$的系数
\begin{align*}
    \begin{cases}
    T''_n + \lambda_n T_n = 2 A_n \\
    T_n(0) = A_n, T_n'(0) = B_n
    \end{cases}
\end{align*}
对应齐次方程特征方程:$\overline{\lambda}^2 + \lambda_n \overline{\lambda} = 0$。所以齐次方程
$$
\overline{T}_n = C_1 \cos \sqrt{\lambda_n}t + C_2 \sin \sqrt{\lambda_n} t
$$
方程的通解为
$$
T_n(x) =  C_1 \cos \sqrt{\lambda_n}t + C_2 \sin \sqrt{\lambda_n} t + \frac{2A_n}{\lambda_n}
$$
再利用初始条件解得
$$
T_n(x) =  (A_n - \frac{2A_n}{\lambda_n}) \cos \sqrt{\lambda_n}t + \frac{B_n}{\lambda_n} \sin \sqrt{\lambda_n} t + \frac{2A_n}{\lambda_n}
$$
因此原问题的解为
$$
u(x,t) = -A + \sum_{n=1}^\infty \mathbb{X}_n(x)T_n(t)
$$
\end{eg}

\begin{note}
如果齐次ODE
\begin{align}
    T''(t) + a_1(t) T'(t) + a_2(t) T(t) = 0
\end{align}
有两个线性无关解$\varphi_1(t), \varphi_2(t)$,则其对应的非齐次方程的通解为
\begin{align}
    T(t) = c_1 \varphi_1(t) + c_2 \varphi_2(t) + \int_0^t \frac{\varphi_1(s)\varphi_2(t) - \varphi_1(t) \varphi_2(s)}{\varphi_1(s)\varphi'_2(t) - \varphi'_1(t)\varphi_2(s)}f(s) \intd s
\end{align}
\end{note}

\textcolor{red}{常微分方程解法见 \emph{高等微积分教程(上), 清华大学出版社. p227-p229}。}

\subsection{初边值问题解的唯一性}

\subsubsection{能量守恒}

\begin{align}
\label{Q3-11-1}
    \begin{cases}
    u_{tt} - a^2 u_{xx} + cu = 0 \\
    u(0,t) = 0, u_x(l,t) = 0 \\
    u(x,0) = \varphi(x), u_t(x,0) = \psi(x)
    \end{cases}
\end{align}
同乘$u_t$,在$[0,l] \times [0,T]$上积分,
$$
\int_0^T \int_0^l (u_{tt}u_t - a^2 u_{xx}u_t + cu u_t) \intd x \intd t = 0
$$

\begin{align*}
\because \quad \int_0^T \int_0^l u_{tt}u_t \intd x \intd t &=\int_0^l \frac{u_t^2(x,T)}{2} \intd x - \int_0^l \frac{u_t^2(x,0)}{2} \intd x \\
\int_0^T \int_0^l u u_t \intd x \intd t &=\int_0^l \frac{u^2(x,T)}{2} \intd x - \int_0^l \frac{u^2(x,0)}{2} \intd x \\
\int_0^T \int_0^l u_{xx}u_t \intd x \intd t &= \int_0^T \left(u_x u_t \bigg|_0^l - \int_0^l u_{xt}u_x \intd x \right) \intd t \\
&= - \int_0^l \frac{u_x^2(x,T)}{2} \intd x + \int_0^l \frac{u_x^2(x,0)}{2} \intd x
\end{align*}
($u_t \bigg|_{x = 0} = 0, u(0,t) = 0$)

令$E(t) = \frac{1}{2} \int_0^l \left(cu^2 + a^2 u_x^2 + u_t^2\right)(x,t) \intd x$ $\Rightarrow$ $E(T) = E(0), \forall T$,能量守恒。

\begin{lemma}
设$Q_T = [0,l] \times [0,T], V \in \mathcal{C}^2(Q_T)$满足
\begin{align}
    V_{tt}  - a^2 V_{xx} + cV = 0, \quad \text{in} Q_T
\end{align}
$V(0,t) = V(l,t) = 0$或者$V_x(0,t) = V_x(l,t) = 0$或者$V_(0,t) = V_x(l,t) = 0$或者周期,则$\forall t \in [0,T]$,均有$E(t) = E(0)$,where
\begin{align} = 0
    E(t) = \frac{1}{2} \int_0^l \left(V_t^2 + a^2 V_x^2 + cV^2\right)(x,t) \intd x
\end{align}
\end{lemma}

\begin{cor}
问题(\ref{Q3-11-1})如果将齐次换为$f, q_1(t), q_2(t)$,其在$\mathcal{C}^2\left([0.l] \times [0,+\infty) \right)$上的解是唯一的。
\end{cor}

\subsubsection{能量方法}

\begin{align}
    \label{Q3-11-2}
    \begin{cases}
    u_{tt} - a^2 \triangle u + \sum b^i(x,t) u_{xi} + c(x) u = f(x,t), \quad \text{in}\; \Omega \times (0,t) \\
    \alpha(x) u + \beta(x) \frac{\partial u}{\partial n} = 0, \quad \partial \Omega \times (0,t) \\
    u \bigg|_{t = 0} = \varphi(x), u_t \bigg|_{t = 0} = \psi(x)
    \end{cases}
\end{align}
其中$\Omega \subset \mathbb{R}^n$有界开,$\partial \Omega$分片属于$\mathcal{C}^1$,$f, b^i, c, \alpha, \beta, \varphi, \psi$是已知函数。

设$u \in \mathcal{C}^1(\overline{\Omega} \times [0,T)) \cap \mathcal{C}^2(\Omega \times (0,T)), \forall 0 < s < T$,在两边同时乘$u_t$,并在$\Omega \times (0, s)$上积分。

$$
\int_0^s \int_\Omega (u_{tt}u_t - a^2 \triangle u u_t + c u u_t) \intd x \intd t = \int_0^s \int_\Omega f u_t - \sum b^i u_{xi} u_t \intd x \intd t
$$
\begin{align*}
    \textcircled{1} &= \int_0^s \int_\Omega u_{tt}u_t \intd x \intd t = \frac{1}{2} \int_\Omega u^2_t(x,s) \intd x - \frac{1}{2} \int_\Omega \psi^2 \intd x \\
    \because \quad &u_{tt} u_t = \frac{1}{2} \frac{\partial}{\partial t} (u_t)^2 \\
    \textcircled{2} &= - \int_0^s \int_\Omega a^2 \triangle u u_t \intd x \intd t = -a^2 \int_0^s \int_{\partial \Omega} \frac{\partial u }{\partial \vec{n}} u_t \intd x \intd t + \frac{a^2}{2} \int_\Omega |\triangledown u(x,s)|^2 \intd x - \frac{a^2}{2} \int_\Omega |\triangledown \varphi|^2 \intd x \\
    \because \quad &\triangle u u_t = \text{div}(\triangledown u) u_t, \; \; \text{div}(u_t \triangledown u) = \text{div}(\triangledown u)u_t + \triangledown u_t \cdot \triangledown u\\
    \textcircled{3} &= \int_0^s \int_\Omega c u u_t \intd x \intd t =  \frac{1}{2} c \int_\Omega u^2(x,s) \intd x - \frac{1}{2} c \int_\Omega \varphi^2 \intd x\\
\end{align*}
因为$\alpha^2 + \beta^2 \neq 0$,设
\begin{align*}
    \Gamma_1 &= \left\{x \in \partial \Omega, \alpha(x) \neq 0\right\} \\
    \Gamma_2 &= \partial \Omega \backslash \Gamma_1
\end{align*}
则$\beta = 0$ on $\Gamma_2$,所以
\begin{align*}
\textcircled{4} = -a^2 \int_0^s \int_{\partial \Omega} \frac{\partial u }{\partial \vec{n}} u_t \intd x \intd t
    &= -a^2 \int_0^s \int_{\Gamma_1} \frac{\partial u }{\partial \vec{n}} \left(- \frac{\beta}{\alpha} \frac{\partial u }{\partial \vec{n}}\right)_t \intd x \intd t + a^2 \int_0^s \int_{\Gamma_2} u_t \left(\frac{\alpha}{\beta}u\right) \intd x \intd t \\
    &= \frac{a^2}{2} \int_{\Gamma_1} \frac{\beta}{\alpha} \left(\frac{\partial u }{\partial \vec{n}}\right)^2 (x,s) \intd x - \frac{a^2}{2} \int_{\Gamma_1} \frac{\beta}{\alpha} \left(\frac{\partial \varphi }{\partial \vec{n}}\right)^2 \intd x \\
    &+ \frac{a^2}{2} \int_{\Gamma_2} u^2(x,s) \frac{\alpha}{\beta} \intd x - \frac{a^2}{2} \int_{\Gamma_2} \frac{\alpha}{\beta} \varphi^2 \intd x
\end{align*}
设$\alpha, \beta \geq 0$,则$\textcircled{4} \geq - c_1(\varphi)$,其中
$$
c_1(\varphi) = \frac{a^2}{2} \left[\int \frac{\beta}{\alpha} \left(\frac{\partial \varphi}{\partial \vec{n}}\right)^2 \intd x + \int_{\Gamma_2} \frac{\alpha}{\beta} \varphi^2 \intd x\right]
$$
再考虑右端
\begin{align*}
    \int_0^s \int_\Omega f u_t - \sum b^i u_{xi} u_t \intd x \intd t
    &\leq \frac{1}{2} \int_0^s \int_\Omega u_t^2 \intd x \intd t + \frac{1}{2} \int_0^s \int_\Omega f^2 \intd x \intd t \\
    &+ \frac{1}{2} \int_0^s \int_\Omega u_t^2 \intd x \intd t + c_2 \int_0^s \int_\Omega \frac{|\triangledown u|^2}{2} \intd x \intd t
\end{align*}
where $c_2 = (\sum_i \max \|b^i\|)^2$。

令$E(t) = \frac{1}{2} \int_0^s \int_\Omega u_t^2 + a^2 |\triangledown u|^2 + cu^2 \intd x$,
$$
\Longrightarrow \quad \frac{d}{d t} E(t) \leq E(0) + c_1(\varphi) + \hat{c}_2E(t) + \frac{1}{2} \int_0^s \int_\Omega f^2 \intd x \intd t
$$
where
\begin{align*}
    \hat{c}_2 =
    \begin{cases}
        2, \quad &c_2 \leq a^2 \\
        \max \{2, a^2\}, \quad & c_2 > a^2
    \end{cases}
\end{align*}
所以
$$
\frac{d}{d t} E(s) \leq \hat{c}_2 E(s) + F(s)
$$
where
$$
F(s) = \frac{1}{2} \int_0^s \int_\Omega f^2 \intd x \intd t + \hat{E}(0), \quad \quad \hat{E}(0) = \frac{1}{2} \int_\Omega \left(|\psi|^2 + c \varphi^2 + |\triangledown \varphi|^2 \right) \intd x
$$
由Gronwell不等式,
\begin{align*}
    &E(s) \leq \hat{c}_2^{-1} \left(e^{\hat{c}_2 s - 1}\right) F(s), \quad \forall 0 < s < T \\
    \Longrightarrow \quad & \frac{d}{d s} E(s) \leq \hat{c}_2^{-1} \left(e^{\hat{c}_2 s - 1}\right) e^{\hat{c}_2 s} F(s)
\end{align*}
\begin{thm}
设$u \in \mathcal{C}^2(\Omega \times (0,T)) \cap \mathcal{C}^1(\overline{\Omega}\times (0,T))$是(\ref{Q3-11-2})的解,$\alpha(x), \beta(x), b^i(x,t)$是有界函数,且$\alpha^2 + \beta^2 \neq 0, \alpha \geq 0, \beta \geq 0$,则$\forall S \in (s,T)$均有
\begin{align}
    \int_\Omega \left[u_t^2(x,s) + a^2 \bigg| \triangledown u(x,s)\bigg|^2 + c(x)u^2(x,t) \right] \intd s
    &\leq e^{\hat{c_2}T} \left(\int_0^s \int_\Omega f^2 \intd x \intd t\right) + c_1(f) \notag \\
    &+ \int_\Omega \left(\psi^2 + a^2 |\triangledown \varphi|^2 + c(x) \varphi^2 \intd x\right)
\end{align}
where
\begin{align}
    c_1(\varphi) = a^2 \left[\int_{\Gamma_2}\frac{\alpha}{\beta}\varphi^2 \intd S_x + \int_{\Gamma_1} \frac{\beta}{\alpha} \left(\frac{\partial \varphi}{\partial \vec{n}}\right)^2 \intd S_x \right]
\end{align}
\end{thm}

\begin{cor}
    问题(\ref{Q3-11-2})在$\mathcal{C}^2(\Omega \times [0,T)) \cap \mathcal{C}^1(\overline{\Omega} \times [0, \infty)$中的解是唯一的。
\end{cor}

\subsection{广义解}

\subsubsection{广义(弱)导数}

$\alpha = (\alpha_1, \alpha_2, ..., \alpha_n)$称为多重指标,如果$\alpha_i$是非负指数,记$|\alpha| = \sum_{i=1}^n \alpha_i$,
\begin{align*}
    D^\alpha u = \frac{\partial^{|\alpha|} u}{\partial x_1^{\alpha_1} \cdots \partial x_n^{\alpha_n}}
\end{align*}
$$
\frac{\partial^2 u}{\partial x_i \partial x_j} = v, \; \text{in} \; \Omega \Leftrightarrow \; \varphi \in \mathcal{C}_0^\infty(\Omega)
$$
$$
\int_\Omega \frac{\partial^2 u}{\partial x_i \partial x_j} \varphi \intd x = \int_\Omega v \varphi \intd x = \int_\Omega u \frac{\partial^2 \varphi}{\partial x_i \partial x_j} \intd x = - \int_\Omega \frac{\partial u}{ \partial x_j} \frac{\partial u}{\partial x_i} \intd x
$$

\begin{definition}
若$u,v \in L^1_\alpha (\Omega)$,有
\begin{align}
    \int_\Omega v \varphi \intd x = (-1)^{|\alpha|} \int_\Omega u D^\alpha \varphi \intd x, \quad \forall \varphi \in \mathcal{C}_0^\infty(\Omega)
\end{align}
则称$v$是$u$的$|\alpha|$阶弱导数,记为$v = D^\alpha u$。
\end{definition}

\textbf{(1) 基本性质}
\begin{enumerate}[(i)]
    \item 若 $D^\alpha u_i = v_i$,$$
    \int_\Omega v \varphi = (-1)^{|\alpha|} \int_\Omega (uv) D^\alpha \varphi
    $$
    \item $$\int v_i \varphi = (-1)^{|\alpha|} \int_\Omega u_i D^\alpha \varphi, \quad \varphi \in \mathcal{C}_0^\infty(\Omega)$$
    \item $v = \psi \in \Omega \Longrightarrow \varphi \psi \in \mathcal{C}_0^\infty$,
    $$
    D(uv) = v D u  + u D v
    $$
\end{enumerate}

\textbf{(2) 求法}

求$D^\alpha u \Leftrightarrow F \intd V$ s.t.
$$
\int_\Omega u D^\alpha \varphi \intd x = (-1)^{|\alpha|} \int_\Omega v \varphi \intd x
$$

\begin{eg}
    求$\mathbb{R}^n$中的函数$u(x) = |x|^{- r}$的弱导数,$\frac{\partial u}{\partial x}, r \leq n-1$。
\end{eg}

\begin{slv}
$\forall \varphi \in \mathcal{C}_0^\infty(\mathbb{R}^n)$,令$\Omega = \{x \in \mathbb{R}^n. \varphi(x) \neq 0\}$,则
\begin{align*}
    \int_{\mathbb{R}^n} |x|^{-r} \frac{\partial \varphi}{\partial x_i} \intd x
    &= \int_\Omega |x|^{-r} \frac{\partial \varphi}{\partial x_i} \intd x = \lim\limits_{\epsilon \rightarrow 0} \int_{\Omega \backslash B_\epsilon(0)} |x|^{-r} \frac{\partial \varphi}{\partial x_i} \intd x \\
    &= \lim\limits_{\epsilon \rightarrow 0} \left(\int_{\partial B_\epsilon(0)}\left(|x|^{-r}\varphi \right) (\vec{n} \vec{e}_i)  - \int_{\Omega \backslash B_\epsilon(0)} \varphi \frac{\partial}{\partial x_i} |x|^{-r}  \right) \; \; \text{(分部积分)} \\
    &= \lim\limits_{\epsilon \rightarrow 0} - \int_{\Omega \backslash B_\epsilon(0)} \varphi \frac{\partial}{\partial x_i} |x|^{-r} \quad \text{(球面对称性)} \\
    &= - \int_{\Omega} \varphi \frac{\partial}{\partial x_i} |x|^{-r}
\end{align*}
所以可以解得
\begin{align}
    V = \begin{cases}
    \frac{\partial }{\partial x_i} |x|^{-r}, \quad &x \neq 0 \\
    \text{any value}, \quad & x = 0
    \end{cases}
\end{align}
\end{slv}

\subsubsection{广义解举例}

\begin{align}
\label{Q3-12-1}
    \begin{cases}
    u_{tt} - a^2 u_{xx} + cu = f(x,t), \quad 0 < x < l, t>0 \\
    u(0,t) = q_1(t), u_x(l,t) = q_2(t) \\
    u(x,0) = \varphi(x), u_t(x,0) = \psi(x)
    \end{cases}
\end{align}
对$\forall T > 0, \forall \xi \in \mathcal{C}^2(\overline{Q}_T)$, 将问题(\ref{Q3-12-1})的两边同时乘$\xi$并在$Q_T$上积分,
\begin{align*}
    \int_{Q_T} [u_{tt} - a^2 u_{xx}] \xi \intd x \intd t + \int_{Q_T} cu \xi \intd x \intd t = \int_{Q_T} f \xi \intd x \intd t
\end{align*}
\begin{align*}
    \textcircled{1} = &\int_{Q_T} [u_{tt} - a^2 u_{xx}] \xi \intd x \intd t \\
    = &\int_0^l u_t \xi \bigg|_{t > 0}^T \intd x - \int_0^l \intd x \int_0^T u_t \xi_t \intd t - a^2 \int_0^T \intd t u_x \xi \bigg|_{x = 0}^l + a^2 \int_0^T \intd t \int_)^l u_x \xi_x \intd x \\
    = &- \int_0^l \psi(x) \xi(x,0) \intd x + a^2 \int_0^T q_2(t) \xi(l,t) \intd t - \int_{Q_T} \left(u_t \xi_t -a^2 u_x \xi_x\right) \intd x \intd t \quad \quad (u \in \mathcal{C}^1(\overline{Q}_T)) \\
    = &- \int_0^l \psi(x) \xi(x,0) \intd x + a^2 \int_0^T q_2(t) \xi(l,t) \intd t - \int_0^l d\intd t u \xi_t \bigg|_{t=0}^T + \int_{Q_T} u \xi_{tt} \intd x \intd t \\
    &+ a^2 \int_0^T \intd t u \xi_x \bigg|_{x=0}^l - a^2 \int_{Q_T} u \xi_{xx} \intd x \intd t \\
    = &- \int_0^l \psi(x) \xi(x,0) \intd x + a^2 \int_0^T q_2(t) \xi(l,t) \intd t - \int_0^T \varphi(x) \xi_t(x,0) \intd x + a^2 \int_0^T q_1(t) \xi_x(0,t) \intd t \\
    &+ \int_{Q_T}  u(\xi_{tt} - a^2 \xi_{xx}) \intd x \intd t
\end{align*}
令$D(Q_T) = \left\{\xi \in \mathcal{C}^2(\overline{Q}_T): \xi(x,T) = \xi_t(x,T)=\xi(0,t)=\xi_x(l,t)=0, \forall (x,t) \in \overline{Q}_T \right\}$,令
$$
I(\xi) = - \int_0^l \psi(x) \xi(x,0) \intd x + a^2 \int_0^T q_2(t) \xi(l,t) \intd t - \int_0^T \varphi(x) \xi_t(x,0) \intd x + a^2 \int_0^T q_1(t) \xi_x(0,t) \intd t
$$

\textbf{(1) 定义}

\begin{definition} \textbf{广义解}.

对$\forall T > 0$,如果$u \in \mathcal{C}(\overline{Q}_T)$满足
\begin{align}
    \int_{Q_T} \xi \left(u_{tt} - a^2 u_{xx} + cu\right) \intd x \intd t + I(\xi) = \int_{Q_T} f \xi \intd x \intd t, \quad \forall \xi \in D(\overline{Q}_T)
\end{align}
则称$u$为问题(\ref{Q3-12-1})的\emph{广义解}。(线性泛函的表示定理)
\end{definition}

\textbf{(2) 古典解是广义解}

\textbf{(3) 广义解是否唯一?}

设(\ref{Q3-12-1})有两个广义解,$u_1, u_2$,令$v = u_1 - u_2$,则
$$
\int_{Q_T} v(\xi_{tt} - a^2 \xi_{xx} + c \xi) \intd x \intd t + I(\xi) = 0, \quad \forall T > 0, \forall \xi \in  D(\overline{Q_T})
$$
因此,$v = 0$。$\forall g \in \mathcal{C}_0^\infty(Q_T)$,只需要找到$\xi \in D(Q_T)$,s.t.
\begin{align*}
    \begin{cases}
    \xi_{tt} - a^2 \xi_{xx} + c \xi = g, \quad \text{in} \; Q_T \\
    \xi(0,t) = \xi_x(l,t) = 0 \\
    \xi(x,T) = \xi_t(x,T) = 0
    \end{cases}
\end{align*}
令$\tau \rightarrow T - t$,
\begin{align*}
  \begin{cases}
    \xi_{tt} - a^2 \xi_{xx} + c \xi = g, \quad \text{in} \; Q_T \\
    \xi(0,t) = \xi_x(l,t) = 0 \\
    \xi(x,0) = \xi_\tau(x,0) = 0, \quad g \in \mathcal{C}_0^\infty(Q_T)
  \end{cases}
\end{align*}
因此,解存在。$\xi \in \mathcal{C}^2(\overline{Q_T})$

\newpage

\section{热方程}

\subsection{Fourier变换}

\subsubsection{定义}

\begin{definition}
  若$f \in L^1(\mathbb{R}^n)$,则称
\begin{align}
    \hat{f}(x) &= \mathcal{F}(f)(x) = \left(\frac{1}{2\pi}\right)^{n/2} \int_{\mathbb{R}^n} f(y) e^{-ixy} \intd y \\
    \breve{f}(x) &= \mathcal{F}^{-1}(f)(x) = \left(\frac{1}{2\pi}\right)^{n/2} \int_{\mathbb{R}^n} f(y) e^{ixy} \intd y
\end{align}
分别为$f$的Fourier变换和Fourier逆变换。
\end{definition}

\begin{note}
\begin{align*}
&e^{i\alpha} = \cos \alpha + i \sin \alpha \\
&|e^{-ixy}| = |e^{ixy}| = 1
\end{align*}
由上述,$\hat{f}, \breve{f}$在$L^1(\mathbb{R}^n)$上有定义。
\end{note}

问题:$f \in L^2(\mathbb{R}^n)$,如何定义$\hat{f}$?

\begin{thm} Plancherel.
\label{thm-Plancherel}
若$f \in L^2(\mathbb{R}^n) \cap L^1(\mathbb{R}^n)$,则有
\begin{align}
    \| f\|_{L^2(\mathbb{R}^n)} = \| \hat{f}\|_{L^2(\mathbb{R}^n)} = \| \breve{f}\|_{L^2(\mathbb{R}^n)} = \left(\int_{\mathbb{R}^n} f^2 \intd x \right)^{1/2}
\end{align}
\end{thm}
\begin{proof}
见Evans.L.C. \S 4.3,(思路见反演公式。)
\end{proof}

\textbf{推荐}: E.M.Stein \& Weiss. Introduction to Fourier Analysis in Euclidean Space. Princeton University Press. 1975.

$\forall f \in L^2(\mathbb{R}^n)$,取$\{f_k\} \subset L^2(\mathbb{R}^n) \cap L^1(\mathbb{R}^n)$,s.t.
\begin{align*}
    &\| f_k - f \|_{L^2(\mathbb{R}^n)} \longrightarrow 0 \; (k \rightarrow 0) \\
\Longrightarrow \quad &\{f_k\} \; \text{是} \; L^2(\mathbb{R}^n) \; \text{中的基本列} \\
\Longrightarrow \quad &\{\hat{f}_k\}, \{\breve{f}_k\} \; \text{是} \; L^2(\mathbb{R}^n) \; \text{中的基本列}
\end{align*}

\begin{definition}
\begin{align}
    \hat{f}(x) = \lim\limits_{k \rightarrow \infty} \hat{f}_k(x) \notag \\
    \breve{f}(x) = \lim\limits_{k \rightarrow \infty} \breve{f}_k(x)
\end{align}
由定理(\ref{thm-Plancherel})知,该极限与$f_k$选取无关。
\end{definition}

\subsubsection{性质}

\begin{enumerate}[(1)]
  \item 线性性质:若$f_1, f_2 \in L^2(\mathbb{R}^n)$(或$L^1(\mathbb{R}^n)$),$\lambda_i \in \mathcal{C}$。则
  \begin{align}
  \widehat{(\lambda_1 f_1 + \lambda_2 f_2)} = \lambda_1 \hat{f}_1 + \lambda_2 \hat{f}_2, \quad \text{逆变换也成立}
  \end{align}
  \item 导数性质:$f, \frac{\partial f}{\partial x_j} \in L^1(\mathbb{R}^n)$(或$L^2(\mathbb{R}^n)$),则
  \begin{align}
      \widehat{\left(\frac{\partial f}{\partial x_j}\right)}(y) = i y_j \hat{f}(y)
  \end{align}
  \item 乘积性质:$f(x), f(x)x_j \in L^1(\mathbb{R}^n)$,则
  \begin{align}
      \frac{\partial }{\partial y_j} \hat{f}(y) = - i \widehat{\left[x_j f(x)\right]}(y)
  \end{align}
  以及推论:
  \begin{align}
      \widehat{(x_j^m f(x))}(y) = i^m \frac{\partial^m }{\partial y^m_j} \hat{f}(y)
  \end{align}
  \item 平移性质:$a \in \mathbb{R}^n$,$f \in L^2(\mathbb{R}^n)$ 或$L^1(\mathbb{R}^n)$,
  \begin{align}
      \widehat{\left[f(x-a)\right]}(y) = e^{-iay} \hat{f}(y)
  \end{align}
  \item 伸缩变换:$k \in \mathbb{R}, k \neq 0$,$f \in L^2(\mathbb{R}^n)$ 或$L^1(\mathbb{R}^n)$
  \begin{align}
    \widehat{\left[f(kx)\right]}(y) = \frac{1}{|k|^n} \widehat{[f(x)]}(\frac{y}{k})
  \end{align}
  \item 对称性质:
  \begin{align}
      (\breve{f}(x))(y) = \hat{f}(y) = (\breve{f}(-x))(y)
  \end{align}
  \item 变量分离:若$f(x) = \prod_{j=1}^n f_j(x_j)$,且$f_j \in L^2(\mathbb{R}^n)$ 或$L^1(\mathbb{R}^n)$,则
  \begin{align}
      \hat{f}(x) = \prod_{j=1}^n \hat{f}_j(x_j)
  \end{align}
  \item 卷积性质:$f,g \in L^1(\mathbb{R}^n)$,定义其卷积
  $$
  (f * g)(x) = \int_{\mathbb{R}^n} f(x - y) g(y) \intd y, \quad \quad x \in \mathbb{R}^n
  $$
  则$f * g \in L^1(\mathbb{R}^n)$,且
  \begin{align}
      \widehat{[f * g]} = \hat{f} \cdot \hat{g} \cdot (2\pi)^{n/2}
  \end{align}
\end{enumerate}

\begin{proof}
(2)

先设$f, \frac{\partial f}{\partial y_j} \in L^1(\mathbb{R}^n)$,则$\exists r_k \rightarrow \infty$,s.t.
$$
\int_{\partial B_{r_k}(0)} \left|\frac{\partial f}{\partial y_j}\right| \rightarrow 0
$$
\begin{align*}
    \widehat{\left(\frac{\partial f}{\partial x_j}\right)}(y)
    &= \left(\frac{1}{2 \pi}\right)^{- n/2} \lim\limits_{r_k \rightarrow \infty} \int_{\partial B_{r_k}(0)} \frac{\partial f}{\partial y_j} (x) e^{-ixy} \intd x \\
    &= \left(\frac{1}{2 \pi}\right)^{n/2} \left(\lim\limits_{r_k \rightarrow \infty} i y_j \int_{\partial B_{r_k}(0)} f(x) e^{-ixy} \intd x\right)
\end{align*}

(3)

\begin{align*}
    \frac{\partial }{\partial y_j} \hat{f}(y) &= \left(\frac{1}{2 \pi}\right)^{n/2} \int_{\mathbb{R}^n} \frac{\partial }{\partial y_j} \left(f(x) e^{-ixy}\right) \intd x  \\
    &= \left(\frac{1}{2 \pi}\right)^{n/2} \int_{\mathbb{R}^n} (-i x_j) f(x) e^{-ixy} \intd x \\
    &= - i \widehat{\left[x_j f(x)\right]}(y)
\end{align*}

(8)

\begin{align*}
    \int_{\mathbb{R}^n} |f * g| \intd x
    &\leq \int_{\mathbb{R}^n} \int_{\mathbb{R}^n} |f(x-y)| |g(y)| \intd y \intd x \\
    &= \int_{\mathbb{R}^n} \intd y \int_{\mathbb{R}^n} |f(x-y)| |g(y)| \intd x \\
    &= \int_{\mathbb{R}^n} |g(y)| \intd y \int_{\mathbb{R}^n} |f(x-y)| \intd x \leq + \infty
\end{align*}
\begin{align*}
    \widehat{[f * g]}(x) &=  \left(\frac{1}{2 \pi}\right)^{n/2} \int_{\mathbb{R}^n} (f * g) e^{-ixy} \intd y \\
    &= \left(\frac{1}{2 \pi}\right)^{n/2} \int_{\mathbb{R}^n} e^{-ixy} \intd y \int_{\mathbb{R}^n} f(y-z) g(z) \intd z \\
    &= \left(\frac{1}{2 \pi}\right)^{n/2} \int_{\mathbb{R}^n} g(z) \intd z \int_{\mathbb{R}^n} f(y-z) e^{-ixy} \intd y \\
    &= \left(\frac{1}{2 \pi}\right)^{n/2} \int_{\mathbb{R}^n} g(z) \intd z \int_{\mathbb{R}^n} f(Y) e^{-ix(Y+z)} \intd Y \\
    &= \left(\frac{1}{2 \pi}\right)^{n/2} \int_{\mathbb{R}^n} g(z)e^{-ixz} \intd z \int_{\mathbb{R}^n} f(Y) e^{-ixY} \intd Y \\
    &= \left(\frac{1}{2 \pi}\right)^{n/2} \hat{f}(x) \hat{g}(x)
\end{align*}
\end{proof}

\begin{thm} 反演公式.
若$u \in L^1(\mathbb{R}^n) \cap \mathcal{C}(\mathbb{R}^n)$,$\hat{u} \in L^1(\mathbb{R}^n)$,则$u(x) = \breve{\hat{u}}(x), \forall x \in \mathbb{R}^n$。
\end{thm}

\begin{eg}
设$A,B > 0$, 求$\widehat{\left(e^{-A|x|^2}\right)}$和$\widehat{\left(e^{-B|x|^2}\right)}$.

\begin{align*}
    \widehat{\left(e^{-A|x|^2}\right)}
    &= \widehat{\left(e^{-|x|^2}\right)} \left(\frac{y}{\sqrt{A}}\right) \left(\frac{1}{A}\right)^{n/2} \; \; \text{(伸缩)} \\
    &= A^{-n/2} \prod_{j=1}^n \widehat{\left(e^{-x_j^2}\right)} \frac{y_j}{\sqrt{A}} \;\; \text{(变量分离)} \\
    &= A^{-n/2} \prod_{j=1}^n \left(\frac{1}{2\pi} \int_{-\infty}^\infty e^{-i\frac{y_j}{\sqrt{A}}x_j} e^{-x_j^2} \intd x_j \right) \\
    &= (2 \pi A)^{-n/2} \prod_{j=1}^n I\left(-\frac{y_j}{\sqrt{A}},1\right) \\
    &= (2A)^{-n/2} e^{-\frac{|y|^2}{4A}}
\end{align*}
where
$$
I(a,b) = \int_{-\infty}^\infty e^{iab - bx^2} \intd x = \sqrt{\frac{\pi}{b}} e^{-\frac{a^2}{4b}}, \; \; (b > 0)
$$
\end{eg}

\subsection{Poisson公式}

\subsubsection{Poisson公式的推导}

求
\begin{align}
\label{Q4-2-1}
    \begin{cases}
        u_t - a^2 \triangle u = f(x,t), \quad x \in \mathbb{R}^n, t > 0 \\
        u|_{t = 0} = \varphi(x), \quad x \in \mathbb{R}^n
    \end{cases}
\end{align}
两边对$x \in \mathbb{R}^n$做Fourier变换,令
$$
\hat{u}(x,t) = \widehat{\left(u(y,t)\right)}(x) = (2\pi)^{-n/2} \int_{\mathbb{R}^n} e^{-ixy} u(y,t) \intd y
$$
\begin{align*}
    \begin{cases}
        \hat{u}_t - a^2 \sum_{j=1}^n (i x_j)^2 \hat{u} = \hat{f}(x,t), \quad x \in \mathbb{R}^n, t > 0 \\
        \hat{u}(x,0) = \hat{\varphi}(x), \quad x \in \mathbb{R}^n
    \end{cases}
\end{align*}

\begin{align*}
    \frac{d}{dt} \left(e^{a^2 |x|^2t} \hat{u}(x,t) \right) &= \hat{f}(x,t) e^{a^2 |x|^2t} \\
    \Longrightarrow \; \; e^{a^2 |x|^2t} \hat{u}(x,t) - \hat{\varphi}(x) &= \int_0^t f(x,s) e^{a^2 |x|^2 s} \intd s \\
    \hat{u}(x,t) - \hat{\varphi}(x)e^{- a^2 |x|^2t}  &= \int_0^t \hat{f}(x,s) e^{- a^2 |x|^2(t - s)} \intd s \\
    \therefore \; \; \hat{u}(x,t) = \hat{\varphi}(x)e^{- a^2 |x|^2t} &+ \int_0^t \hat{f}(x,s) e^{- a^2 |x|^2(t - s)} \intd s
\end{align*}
由反演公式,
$$
u(x,t) = \left[e^{-a^2 |y|^2t} \hat{\varphi}(y)\right]{\check{}}\; (x) + \left[ \int_0^t \hat{f}(y,s) e^{- a^2 |y|^2(t - s)} \intd s \right]\check{} \; (x)
$$
且
\begin{align*}
\left[e^{-a^2 |y|^2t} \hat{\varphi}(y)\right]{\check{}} &= \int_{\mathbb{R}^n} \varphi(\xi) K(x - \xi,t) \intd \xi \\
\left[ \int_0^t \hat{f}(y,s) e^{- a^2 |y|^2(t - s)} \intd s \right]\check{} &= \int_0^t \intd s \int_0^t f(\xi, s) K(x - \xi, t - s) \intd \xi
\end{align*}
其中第二个式子利用了Fubini定理,且$K(x,t)$是Poisson核。
\begin{align}
    K(x,t) = \begin{cases}
    0, \quad &t \leq 0 \\
    \left(\frac{1}{4\pi a^2 t}\right)^{n/2} e^{-\frac{|x|^2}{4a^2t}}, \quad &t > 0
    \end{cases}
\end{align}
因此(\ref{Q4-2-1})的解为
\begin{align}
    \label{sol4-2-1}
    u(x,t) = \int_{\mathbb{R}^n} \varphi(\xi) K(x - \xi,t) \intd \xi + \int_0^t \intd s \int_0^t f(\xi, s) K(x - \xi, t - s) \intd \xi
\end{align}
称(\ref{sol4-2-1})为热方程的Poisson公式。

\begin{note}
\begin{align}
    \left[e^{-a^2 |y|^2t}\right]{\check{}} = \left(\frac{1}{2a^2t}\right)^{n/2} e^{- \frac{|x|^2}{4a^2t}}
\end{align}
\end{note}

\subsubsection{Poisson核函数的性质}

\begin{lemma}
\begin{enumerate}[(i)]
    \item $K \in \mathcal{C}^\infty \left( \mathbb{R}^n \times (0, + \infty) \cup (-\infty,0) \right)$
    \item $K_t - a^2 \triangle K = 0$ in $\mathbb{R}^n \times \{t \neq 0\}$
    \item $\int_{\mathbb{R}^n} K(x,t) \intd x = 1, \quad t > 0$
\end{enumerate}
\end{lemma}

\begin{lemma}
齐次方程解的性质:$f \equiv 0$。
\begin{enumerate}[(i)]
  \item $\varphi(\cdot)$的奇偶性和周期性能够传导到解:

  如果$\varphi(-x) = \pm \varphi(x), \forall x \in \mathbb{R}^n$,那么$u(-x,t) = \pm u(x,t), \forall x \in \mathbb{R}^n$;

  如果$\varphi(x + T) = \varphi(x), \forall x \in \mathbb{R}^n$,那么$u(x+T,t) = u(x,t), \forall x \in \mathbb{R}^n$。
  \item $\varphi(\cdot)$可积,具有指数增长,那么$u \in \mathcal{C}^\infty$无穷光滑;
  \item 若$\varphi(\cdot) > 0 \; \; \text{in} \; \; [t_0 - \epsilon, t_0 + \epsilon]$,$\varphi(\cdot) = 0 \; \; \text{in} \; \; \forall x \in \mathbb{R}^n \backslash [t_0 - \epsilon, t_0 + \epsilon]$,那么$u(x,t) > 0, \forall x \in \mathbb{R}^n, t > 0$
\end{enumerate}
\end{lemma}

\subsubsection{Poisson公式的证明}

\begin{thm}
设$\varphi \in \mathcal{C}(\mathbb{R}^n)$且$\exists M, A, B \geq 0, Q > 0$,s.t.
$$
|\varphi(x)| \leq M e^{A|x|^2 + B|x|^{2 - r}}, \; \; x \in \mathbb{R}, f \in \mathcal{C}^{2.1}(\mathbb{R}^n \times (0,+\infty ))
 $$
 且满足$spt f = \{(x,t): f(x,t) \neq 0\}$是有界集,则由(\ref{sol4-2-1})给出的$u \in \mathcal{C}^{2.1} (\mathbb{R}^n \times (0,T_0))$,且满足
 \begin{enumerate}[(i)]
     \item $u_t - a^2 \triangle u = f$ in $\mathbb{R}^n \times (0,+\infty )$
     \item $\lim_{x \rightarrow x_0, t \rightarrow 0^+} u(x,t) = \varphi(x_0), \; \; \forall x_0 \in \mathbb{R}^n$
 \end{enumerate}
 其中
 \begin{align*}
     T_0  = \begin{cases}
     + \infty, \quad &A = 0 \\
     \frac{1}{4a^2A}, \quad &A > 0
     \end{cases}
 \end{align*}
\end{thm}

\subsection{广义函数}

\subsubsection{广义函数的定义}
$$
\delta(x) = \begin{cases}
+ \infty, \quad &x = 0 \\
0, \quad &x \neq 0
\end{cases}
$$
且
$$
\int_{\mathbb{R}} \delta(x) \intd x = 1
$$

\begin{definition}
$\mathcal{C}_0^\infty(\mathbb{R}^n) = \{\varphi \in \mathcal{C}^\infty(\mathbb{R}^n): \text{support}(\varphi) \; \text{是有界集}\}$,$\mathcal{C}_0^\infty(\mathbb{R}^n)$按照函数的线性运算是一个线性空间,记为$\mathcal{D}(\mathbb{R}^n)$(试验函数空间),$\varphi(\cdot)$称为试验函数。

设$\{\varphi_k\} \subset \mathcal{D}(\mathbb{R}^n)$收敛于$\varphi \in \mathcal{D}(\mathbb{R}^n)$,如果
\begin{enumerate}[(i)]
  \item $\exists r > 0$,s.t. support$(\varphi)$和support$(\varphi_k)$ $\subset B_r(0)$,$\forall k$
  \item $\forall \alpha \in \mathbb{Z}^n$,偏导数$D^\alpha \varphi_k$在$B_r(0)$中一致收敛于$D^\alpha \varphi$
\end{enumerate}
则$\mathcal{D}(\mathbb{R}^n)$是一个线性拓扑空间。
\end{definition}

\begin{definition}
如果$F: \mathcal{D}(\mathbb{R}^n) \rightarrow \mathbb{R}^n$是$\mathcal{D}(\mathbb{R}^n)$上的一个线性连续泛函,则称$F$为广义函数。

称$F$为线性连续泛函,如果
\begin{enumerate}[(i)]
  \item $\forall f, g \in \mathcal{D}(\mathbb{R}^n)$,$\lambda, \mu \in \mathbb{R}$,有$F(\lambda f + \mu g) = \lambda F(f) + \mu F(g)$;
  \item $\forall \varphi \in \mathcal{D}(\mathbb{R}^n)$以及任意的$\{\varphi_k\} \subset \mathcal{D}(\mathbb{R}^n)$,如果$\varphi_k \rightarrow \varphi$ in $\mathcal{D}(\mathbb{R}^n)$,则$F(\varphi_k) \rightarrow F(\varphi)$
\end{enumerate}
我们用对偶积$\langle F, \varphi \rangle$表示$F(\varphi)$,记$\mathcal{D}'(\mathbb{R}^n)$为$\mathcal{D}(\mathbb{R}^n)$上广义函数的全体。
\end{definition}

\subsubsection{广义函数的运算}

\textbf{(1) 线性运算}

$F_i \in \mathcal{D}'(\mathbb{R}^n)$,$\lambda, \mu \in (\mathbb{R})$,则
\begin{align}
\langle \lambda F_1 + \mu F_2, \varphi \rangle = \lambda \langle F_1, \varphi \rangle + \mu \langle F_2, \varphi \rangle, \quad \forall \varphi \in \mathcal{D}(\mathbb{R}^n)
\end{align}

\textbf{(2) 乘积}

$\forall \psi \in \mathcal{C}^\infty(\mathbb{R}^n)$,$F \in \mathcal{D}'(\mathbb{R}^n)$,定义乘积$\psi F$:
\begin{align}
\langle \psi F, \varphi \rangle = \langle F, \psi \varphi \rangle, \quad \forall \varphi \in \mathcal{D}(\mathbb{R}^n)
\end{align}
那么$\psi F \in \mathcal{D}'(\mathbb{R}^n)$。

\textbf{(3) 收敛性}

设$\{F_k\}, F \subset \mathcal{D}'(\mathbb{R}^n$,如果$\forall \varphi \in \mathcal{D}(\mathbb{R}^n)$,
\begin{align}
\lim\limits_{k \rightarrow \infty} \langle F_k, \varphi \rangle = \langle F, \varphi \rangle
\end{align}
则称$\{F_k\}$弱*收敛于$F$,记为$F_k \xrightarrow[]{*} F$。

\begin{definition}
若$F \in \mathcal{D}'(\mathbb{R}^n), a \in \mathbb{R}, a \neq  0, b \in \mathbb{R}^n$,定义$\langle F(ax+b), \varphi \rangle = \langle F, \frac{1}{|a|^2} \varphi(\frac{x-b}{a})$,称为$F$的伸缩平移运算。
\end{definition}

\subsubsection{广义函数的导数}

\begin{definition}
回忆$f \in L^1_{loc}(\mathbb{R}^n), D^\alpha f = g$(广义弱导数)如果:
\begin{align}
\int_{\mathbb{R}^n} f D^\alpha \varphi \intd x = (-1)^{|\alpha|} \int_{\mathbb{R}^n} g \varphi \intd x, \quad \forall \varphi \in \mathcal{D}(\mathbb{R}^n)
\end{align}
对于$F \in \mathcal{D}'(\mathbb{R}^n)$,定义$\alpha-$阶偏导数为
\begin{align}
    \langle D^\alpha F, \varphi \rangle = (-1)^{|\alpha|} \langle F, D^\alpha \varphi \rangle, \quad \forall \varphi \in \mathcal{D}(\mathbb{R}^n)
\end{align}
\end{definition}

\begin{thm}

\begin{enumerate}
  \item $\forall F \in  \mathcal{D}'(\mathbb{R}^n), \forall \alpha \in \mathbb{Z}_+^n$,$D^\alpha F$存在且$\forall \varphi \in  \mathcal{D}(\mathbb{R}^n)$
  \begin{align}
      \langle D^\alpha F, \varphi \rangle = (-1)^{|\alpha|} \langle F, D^\alpha \varphi \rangle
  \end{align}
  \item 广义函数的导数运算与普通函数一样:$\forall F_i \in  \mathcal{D}'(\mathbb{R}^n), \lambda_i \in  \mathcal{C}^\infty(\mathbb{R}^n)$
  \begin{align}
      D^\alpha (\lambda_1 F_1 + \lambda_2 F_2) = \lambda_1 D^\alpha F_1 + \lambda_2 D^\alpha F_2
  \end{align}
  如果$F \in  \mathcal{D}'(\mathbb{R}^n), \psi \in \mathcal{C}^\infty(\mathbb{R}^n), \alpha \in \in \mathbb{Z}_+^n$,则
  \begin{align}
      D^\alpha(\psi F) = \sum_{0 \leq \beta \leq \alpha}  \binom{\alpha}{\beta} D^\beta \psi D^{\alpha - \beta} F
  \end{align}
  当$\alpha = 1$时,
  \begin{align}
      D^\alpha(\psi F) = F D^\alpha \psi + \psi D^{\alpha} F
  \end{align}
  对于$\forall \varphi \in \mathcal{D}(\mathbb{R}^n)$,
  \begin{align}
      \langle D^\alpha(\psi F), \varphi \rangle
      &= - \langle \psi F, D^\alpha \varphi \rangle
      = - \langle F, \psi D^\alpha \varphi \rangle \notag \\
      &= - \langle F, D^\alpha(\psi \varphi) - \varphi D^\alpha \psi \rangle \notag \\
      &= \langle D^\alpha F, \psi \varphi \rangle + \langle F, \varphi D^\alpha \psi \rangle
  \end{align}
\end{enumerate}
\end{thm}

\subsection{热方程的基本解}

考虑
\begin{align}
\label{Q4-5-1}
    \begin{cases}
    u_t - a^2\triangle u = 0, \quad &\text{in} \; Q \\
    u\bigg|_{t = 0} = \varphi(x), \quad &\text{in} \; \mathbb{R}^n
    \end{cases}
\end{align}
\begin{align}
\label{Q4-5-2}
    \begin{cases}
    u_t - a^2\triangle u = f(x,t), \quad &\text{in} \; Q \\
    u\bigg|_{t = 0} = \varphi(x), \quad &\text{in} \; \mathbb{R}^n
    \end{cases}
\end{align}
解为
\begin{align*}
    u(x,t) = \int_{\mathbb{R}^n} \varphi(\xi) \Gamma(x - \xi,t) \intd \xi + \int_0^t \intd s \int_0^t f(\xi, s) \Gamma(x - \xi, t - s) \intd \xi
\end{align*}
其中
\begin{align*}
    \Gamma(x,t) = \begin{cases}
    0, \quad &t \leq 0 \\
    \left(\frac{1}{4\pi a^2 t}\right)^{n/2} e^{-\frac{|x|^2}{4a^2t}}, \quad &t > 0
    \end{cases}
\end{align*}
$$
\Gamma(x,t) \xrightarrow[]{t \rightarrow 0^+} \delta(x), \quad \Gamma(x - \xi,t) \xrightarrow[]{t \rightarrow 0^+} \delta(x - \xi)
$$

令$\Gamma(x,\xi, t) = K(x - \xi, t)$,则$\forall \xi \in \mathbb{R}^n$,$\Gamma \in \mathcal{C}^\infty(Q)$,且$\Gamma_t - a^2 \triangle_x \Gamma = 0$ in $Q$,且广义满足$\Gamma(x,\xi,t) \xrightarrow[]{t \rightarrow 0^+} \delta(x - \xi)$。

\begin{definition}
$\forall \xi \in \mathbb{R}^n$,$u(x,t) = u(x,\xi,t) \in \mathcal{L}^1_{loc}(Q)$且在广义函数意义下满足
\begin{align}
    \begin{cases}
      u_t - a^2 \triangle u = 0, \quad \text{in} \; Q \\
      u \xrightarrow[]{t \rightarrow 0^+} \delta(x - \xi) \\
    \end{cases}
\end{align}
则称$u(x,\xi,t)$是齐次方程(\ref{Q4-5-1})的基本解。
\end{definition}

\begin{note}
\begin{enumerate}
  \item 基本解 $\Longrightarrow$ (\ref{Q4-5-1})解的表达式;
  \item 基本解不一定唯一
  \begin{align}
      &\eta(t) = \begin{cases}
      e^{-\frac{1}{t^2}}, \quad & t \neq 0 \\
      0, \quad &t = 0
      \end{cases} \\
      &u(x,t) = \begin{cases}
      \sum\limits_{k=0}^\infty \eta^{(k)}(t) \frac{x^{2k}}{(2k)!}, \quad &t \neq 0 \\
      0, \quad &t = 0
      \end{cases}
  \end{align}
  可以验证$\lim\limits_{t \rightarrow 0^+}u(x,t) = 0$,$u_t - \triangle u = 0$ in $\mathbb{R} \times (0, \infty)$,因此基本解不唯一。
  \item 令$\overline{\Gamma}(x,\xi,t, \tau) = K(x-\xi, t - \tau)$ in $\mathbb{R}^n \times \mathbb{R}^n \times (0,\infty) \times (0,\infty)$,满足$\overline{\Gamma} \in \mathcal{C}^\infty(\{t \neq \tau\})$,$\Gamma_t = - \Gamma_\tau, t \neq \tau$,则
  \begin{align*}
        &\triangle_x \overline{\Gamma} = \triangle_\xi \overline{\Gamma}, \; t \neq \tau  \\
        \Longrightarrow \quad &\overline{\Gamma}_t - a^2 \triangle_x \overline{\Gamma} = 0, \; \forall t \neq \tau \\
        &\overline{\Gamma}_t - a^2 \triangle_x \overline{\Gamma} = \delta(x-\xi, t - \tau), \quad \text{in} \; \mathbb{R}^n \times (0,\infty), \forall (\xi,\tau) \in Q
  \end{align*}
\end{enumerate}
\end{note}

\begin{cor}
$\forall (\xi, \tau) \in Q$,$\Gamma(x,\xi,t,\tau)$,且在广义函数意义下满足
\begin{align}
    \begin{cases}
      \overline{\Gamma}_t - a^2 \triangle_x \overline{\Gamma} = \delta(x-\xi, t - \tau),\quad \text{in} \; Q, \forall (\xi,\tau)\\
      \lim\limits_{t \rightarrow 0^+} \overline{\Gamma} = 0
    \end{cases}
\end{align}
\textcolor{red}{此处在广义函数意义下$\overline{\Gamma}_t - a^2 \triangle_x \overline{\Gamma} = \delta(x-\xi, t - \tau)$是指,对于$\forall \varphi$有
\begin{align}
    \langle \overline{\Gamma}_t - a^2 \triangle_x \overline{\Gamma}, \varphi \rangle = \langle \delta(x - \xi, t - \tau), \varphi \rangle = \varphi
\end{align}
}
\end{cor}
\begin{proof}
对于$\forall \tau > 0$,当$0<t<\tau, \overline{\Gamma} \equiv 0$,我们可以推出$\lim\limits_{t \rightarrow 0^+} \overline{\Gamma} = 0$,因此只需要证明$\forall \varphi \in \mathcal{C}_0^\infty(Q)$均有$\langle \overline{\Gamma}_t - a^2 \triangle_x \overline{\Gamma}, \varphi \rangle = \varphi(x,t)$。

注意到$\langle \overline{\Gamma}_t - a^2 \triangle_x \overline{\Gamma}, \varphi \rangle = \langle \overline{\Gamma}, - \varphi_t - a^2 \triangle_x \varphi \rangle$,因此
$$
-\int_Q \overline{\Gamma}(x,\xi,t,\tau) \varphi_t(x,t) \intd x \intd t - a^2 \int_Q \overline{\Gamma}(x,\xi,t,\tau) \triangle_x \varphi(x,t) \intd x \intd t = \varphi(\xi,\tau)
$$
即
$$
\int_\tau^\infty \int_{\mathbb{R}^n}  \overline{\Gamma}(x,\xi,t,\tau) \left[- \varphi_t(x,t) - a^2 \triangle_x \varphi(x,t)\right] \intd x \intd t = \varphi(\xi, \tau)
$$
左边式子为
\begin{align*}
  &\int_\tau^\infty \int_{\mathbb{R}^n}  \overline{\Gamma}(x,\xi,t,\tau) \left[- \varphi_t(x,t) - a^2 \triangle_x \varphi(x,t)\right] \intd x \intd t \\
  = &\lim\limits_{\epsilon \rightarrow 0^+} - \int_{\mathbb{R}^n}  \overline{\Gamma}(x,\xi,t,\tau) \varphi(x,t) \bigg|_{t = \tau + \epsilon}^{t = \infty} \intd x + \int_{\tau + \epsilon}^\infty \intd t\int_{\mathbb{R}^n} \varphi(x,t) \left[\overline{\Gamma}_t - a^2 \triangle_x \overline{\Gamma}\right] \intd x \\
  = &\lim\limits_{\epsilon \rightarrow 0^+} - \int_{\mathbb{R}^n}  \overline{\Gamma}(x,\xi,t,\tau) \varphi(x,t) \bigg|_{t = \tau + \epsilon}^{t = \infty} \intd x  \\
  = &\lim\limits_{\epsilon \rightarrow 0^+} \int_{\mathbb{R}^n}  \overline{\Gamma}(x,\xi,\tau+\epsilon,\tau) \varphi(x,\tau+\epsilon) \intd x \\
  = &\varphi(\xi, \tau)
\end{align*}
\end{proof}

\begin{definition}
一个定义在$Q \times Q$上的函数$u(x,\xi; t, \tau)$满足
\begin{align}
    \begin{cases}
    u_t - a^2 \triangle_x u = \delta(x - \xi, t - \tau), \quad &\text{in}\; Q \\
    \lim\limits_{t \rightarrow 0^+} u = 0, \quad &\text{in} \; \mathbb{R}^n
    \end{cases}
\end{align}
称$u$为(\ref{Q4-5-2})的一个基本解。
\end{definition}

\subsection{半空间的解}

\textbf{(1) 齐次Neuman边值问题.}$\mathbb{R}_+^n = \{(x_1, ..., x_{n-1}, x_n = (x',x_n): x_n > 0\}$
\begin{align}
\label{Q4-6-1}
    \begin{cases}
    u_t - a^2 \triangle u = f, \quad &\text{in} \; \mathbb{R}_+^n \times (0,\infty) \\
    u\bigg|_{t = 0} = \varphi(x), \quad &\text{in} \; \mathbb{R}_+^n \\
    \frac{\partial u}{\partial x_n} \bigg|_{x_n =0} = 0
    \end{cases}
\end{align}
方法:做偶延拓
\begin{align}
    \tilde{f}(x,t) =
    \begin{cases}
        f(x,t), \quad x_n > 0 \\
        f(x^s,t), \quad x_n < 0
    \end{cases}, \quad     \tilde{\varphi}(x,t) =
    \begin{cases}
        \varphi(x,t), \quad x_n > 0 \\
        \varphi(x^s,t), \quad x_n < 0
    \end{cases}
\end{align}
其中$x^s = (x',-x_n)$。因此(\ref{Q4-6-1})在全空间的解为
$$
    \tilde{u}(x,t) = \int_{\mathbb{R}^n} K(x-\xi, t) \tilde{\varphi}(\xi) \intd \xi + \int_0^t \int_{\mathbb{R}^n} K(x-\xi, t-\tau) \tilde{f}(\xi,\tau) \intd \xi \intd \tau
$$
将其限制在$\mathbb{R}_+^n$上就得到(\ref{Q4-6-1})的解:
\begin{align}
u(x,t)
&= \int_{\mathbb{R}^n_+} \left[K(x-\xi, t) - K(x-\xi^s,t)\right]\tilde{\varphi}(\xi) \intd \xi \notag\\
&+ \int_0^t \int_{\mathbb{R}^n_+} \left[K(x-\xi, t-\tau) - K(x-\xi^s,t-\tau)\right] \tilde{f}(\xi,\tau) \intd \xi \intd \tau
\end{align}

\textbf{(2) 齐次Dirichlet边值问题.} 对$f,\varphi$作奇延拓。

\textbf{(3) Robin齐次边界条件}

\subsection{初边值问题的解法}

\subsubsection{一维情况}

考虑方程
\begin{align}
    \label{Q4-6-1}
    \begin{cases}
      u_t - a^2 u_{xx} = f(x,t), \quad &0 < x < l, t > 0 \\
      u(0,t) = q_1(t), \; u_x(l,t) = q_2(t), \quad &t > 0 \\
      u(x,0) = \varphi(x), \quad &x \in [0,l]
    \end{cases}
\end{align}

我们使用Fourier方法(分离变量)

\begin{enumerate}[Step 1.]
  \item 化为齐次边界条件:

  作$u(x,t) = v(x,t) + w$, 解得$w(x,t) = q_1(t) + xq_2(t)$,因此可以设$q_1(t) = q_2(t) = 0$。

  \item 令$u(x,t)= T(t)\mathbb{X}(x)$,分离变量并代入原方程:

  $$
  \frac{T'}{a^2 T} = \frac{\mathbb{X}''}{\mathbb{X}} = - \lambda
  $$

  根据定理(\ref{thm-character}),$\lambda > 0$,设$\lambda = \beta^2, \beta > 0$,因此得到
  $$
  \mathbb{X}(x) = c_1 \cos \beta x + c_2 \sin \beta x
  $$
  根据边界条件:
  \begin{align*}
      \begin{cases}
      \mathbb{X}(0) = 0 \\
      \mathbb{X}_x(l) = 0
      \end{cases} \; \; \Longrightarrow
      \begin{cases}
        c_1 = 0 \\
        \cos \beta l = 0
      \end{cases}
  \end{align*}
  因此$\beta_k = \frac{(2k+1)\pi}{2l}, k = 0,1,2,...$, $\lambda_k = \beta_k^2$。对应的特征函数为$\mathbb{X}_k(x) = \cos \beta_k x$

  \item 叠加原理(待定系数)

  $$
  u(x,t) = \sum_{k=0}^\infty T_k(t) \cos \beta_k x
  $$
  代入原问题:
  \begin{align}
      \begin{cases}
        \sum_{k=0}^\infty \left(T_k' + a^2 \beta_k^2 T_k(t)\right) \cos \beta_k x = f(x,t) \\
        \sum_{k = 0}^\infty T_k(0) \cos \beta_k x = \varphi(x)
      \end{cases}
  \end{align}
  将$f, \varphi$关于$\{\cos \beta_k x\}$展开,
  \begin{align*}
      f(x,t) = \sum_{k=0}^\infty f_k(t) \cos \beta_k x,& \quad f_k(t) = \frac{2}{l} \int_0^l f(x,t) \cos \beta_k x\intd x \\
      \varphi(x) = \sum_{k=0}^\infty \varphi_k \cos \beta_k x,& \quad \varphi_k = \frac{2}{l} \int_0^l \varphi(x) \cos \beta_k x\intd x
  \end{align*}
  因此问题化为求解常微分方程:
  \begin{align}
      \begin{cases}
        T'_k + a^2 \beta_k^2 T_k = f_k(t) \\
        T_k(0) = \varphi_k
      \end{cases}
  \end{align}
  可以解得
  \begin{align}
      T_k(t) &= \left[\varphi_k + \int_0^t f_k(s)e^{(a \beta_k)^2 s} \intd s \right] e^{-(a\beta_k)^2t} \notag \\
      &= e^{-(a\beta_k)^2t}\varphi_k + \int_0^t e^{- (a \beta_k)^2 (t-s)} f_k(s) \intd s
  \end{align}
\end{enumerate}

\begin{note}
\begin{enumerate}
  \item Poisson公式:设$f \equiv 0$,$u(x,t) = \int_{\mathbb{R}}K(x - \xi)\varphi(\xi) \intd \xi$,Poisson公式按照$\{\cos \beta_k x\}$ Fourier展开得到结果同上式。

  Poisson公式 $\Longleftrightarrow$ 周期初边值问题的解。

  \item
  \begin{align}
  \label{sol4-6}
      u(x,t) &= \sum_{k = 0}^\infty e^{-(a\beta_k)^2 t} \left(\frac{2}{l}\int_0^l \varphi(\xi)\cos \beta_k \xi \intd \xi \right) \cos \beta_k x \\
(f \equiv 0) \quad &= \frac{2}{l} \int_0^l \varphi(\xi) \left[\sum_{k=0}^\infty \cos \beta_k \xi \cos \beta_k x e^{-(a\beta_k)^2 t} \right] \intd \xi
  \end{align}
  经过计算:
  $$
  \sum_{k=0}^\infty \cos \beta_k \xi \cos \beta_k x e^{-(a\beta_k)^2 t} = K(x - \xi, t)
  $$
\end{enumerate}
\end{note}

\subsubsection{高维情况}

仍旧是$u(x,t) = T(t) \mathbb{X}(x)$,
\begin{align}
    \begin{cases}
      \triangle \mathbb{X} + \lambda \mathbb{X} = 0, \quad \mathbb{X} \in \Omega \\
      \mathbb{X} \bigg|_{\partial \Omega} = 0, \; or \; \mathbb{X} + \frac{\partial \mathbb{X}}{\partial \vec{n}} \bigg|_{\partial \Omega} = 0
    \end{cases}
\end{align}
化为Laplace方程特征值问题。方法:求变分。

若$\Omega = B_\rho(0)$,可以解得方程:$\mathbb{X}(x) = \gamma(\rho)\Theta(\theta)$,再分离变量。

若$\Omega = [0,a] \times [0,b] \times \cdots$,可以得到许多ODE。

\subsubsection{无穷衰减性质}

当$f \equiv 0, q_i \equiv 0$,
\begin{align*}
    u(x,t) &= \frac{2}{l} \int_0^l \sum_{k = 0}^\infty e^{-(a\beta_k)^2 t} \varphi(\xi)\cos \beta_k \xi \cos \beta_k x \intd \xi, \\
    \beta_0 &= \frac{\pi}{2 l}, \beta_k > \beta_0, \forall k > 0 \\\
\end{align*}
$\forall \beta \in (0, a \beta_0), \; \Rightarrow \; \lim\limits_{t \rightarrow \infty} e^{-\beta^2 t} u(x,t) = 0$,因此$u(x,t)$关于$t$是指数衰减的,且关于$x$是一致的。

\subsubsection{验证定律}

令$Q = (0,l) \times (0,\infty), \overline{Q} = [0,l) \times [0,\infty)$,
\begin{align}
    \begin{cases}
      u_t - a^2 u_{xx} = f(x,t) \\
      u(0,t) = q_1(t), \; u_x(l,t) = q_2(t) \\
      u(x,0) = \varphi(x)
    \end{cases}
\end{align}

\begin{enumerate}[(i)]
  \item 要求$u \in \mathcal{C}^{2.1}(\overline{Q})$,必须
  \begin{align}
      \begin{cases}
        q_1(0) = \varphi(0), \varphi'(t) = q_2(0) \\
        q_1'(0) - a^2 \varphi''(0) = f(0,0) \\
        q_2'(l) - a^2 \varphi'''(l) = f_x(l,0), \; \text{此条件波方程需要热方程不需要}
      \end{cases}
  \end{align}
  \item 只要求 $u \in \mathcal{C}^{2.1}(Q) \cap \mathcal{C}^{1}(\overline{Q})$,必须
  \begin{align*}
      q_1(0) = \varphi(0), \varphi'(l) = q_2(0)
  \end{align*}
  \item 广义解。(\ref{sol4-6})级数需要收敛($\overline{Q}$)。
\end{enumerate}

\begin{thm}
\begin{enumerate}[(i)]
  \item 若$f \equiv 0, q_i \equiv 0, \varphi \in \mathcal{C}^1([0,l])$,则(\ref{Q4-6-1})的解$u \in \mathcal{C}^\infty((0,l) \times (0,\infty))$,且$\forall \beta \in (0, (\frac{a\pi}{2l})^2)$,均有
  $$
  \lim\limits_{t \rightarrow \infty} e^{\beta t} u(x,t) = 0
  $$
  $\forall x \in [0,l]$一致成立。
  \item $\forall T > 0, f \in \mathcal{C}^2(\overline{Q}_T), \varphi \in \mathcal{C}^1([0,l]), q_i \in \mathcal{C}^1([0,l])$,且$\varphi'(0) = q_2(0), q_1(0) = \varphi(0)$,则用分离变量法求出的解$u$满足
  $$
  u(x,t) \in \mathcal{C}^{2.1} (Q_T) \cap \mathcal{C}^1(\overline{Q}_T)
  $$
  在$\overline{Q}_T$上满足(\ref{Q4-6-1}),其中$Q_T = (0,l) \times (0,T)$。
\end{enumerate}
\end{thm}

\subsection{极值原理和最大模估计}

\subsubsection{初边值问题}

考虑$\Omega \in \mathbb{R}^n$为有界开集,$T>0$,证:$\Omega_T = \Omega \times [0,T]$,$\Gamma_T = \overline{\Omega}_T \backslash \Omega_T$\textcolor{red}{(抛物界面,即下底面+柱面)},
$$
u_t - a^2 \triangle u = f, \; \text{in} \; \Omega_T; \; u\bigg|_{\Gamma_T} = \varphi(x)
$$
设$u(x,t) \in \mathcal{C}^{2} (\Omega_T) \cap \mathcal{C}^1(\overline{\Omega}_T)$,

若在$(x_0, t_0) \in \Omega_T$取得极大值,$\Longrightarrow$ $u(x,t_0)$在$x_0$处取得极大值,因此
\begin{align*}
    \triangledown u(x_0,t_0) = 0, \quad D^2(u(x_0,t_0)) = \left[u_{x_i x_j}(x_0,t_0) \right]_{n \times 1} \leq 0
\end{align*}
因此$\triangle u(x_0,t_0) \leq 0$。同理,$u(x_0,t)$在$t_0$处取得极大值,因此
$$
\partial_t u(x_0,t_0) = 0, \;  (t_0 < T), \quad \partial_t u(x_0,t_0) \geq 0, \;  (t_0 = T)
$$
$$
\Longrightarrow \; u_t - a^2 \triangle u \geq 0
$$
换言之,若$u_t - a^2 \triangle u < 0, \; \text{in} \; \Omega_T$,则$u$不可能在$\Omega_T$中取得极大值(包含最大值)。特别地
\begin{align}
\label{equa4-7-1}
    \max\limits_{\overline{\Omega}_T} u = \max\limits_{\Gamma_T} u
\end{align}

\begin{thm}
\label{thm-4-7}
设$u(x,t) \in \mathcal{C}^{2.1} (Q_T) \cap \mathcal{C}^1(\overline{Q}_T)$,满足
\begin{align}
    u_t - a^2 \triangle u \leq 0, \; \text{in} \; \Omega_T
\end{align}
则有:
\begin{enumerate}[(1)]
  \item
  \begin{align}
      \max\limits_{\overline{\Omega}_T} u = \max\limits_{\Gamma_T} u
  \end{align}
  \item \textbf{强极值原理.} 若存在$(x_0,t_0) \in \Omega_T$,s.t. $u(x_0,t_0) = \max\limits_{\overline{\Omega}_T}$,则
  \begin{align}
      u(x,t) \equiv u(x_0,t_0), \quad \forall (x,t) \in \overline{\Omega}_{t_0}
  \end{align}

\end{enumerate}
\end{thm}

\begin{proof}
(1)的证明:令$V_\epsilon(x,t) = u(x,t) - \epsilon t, \epsilon > 0$,则
$$
V_{\epsilon t} - a^2 \triangle V_\epsilon = u_t - a^2 \triangle u - \epsilon < 0
$$
由(\ref{equa4-7-1}),
$$
\max\limits_{\overline{\Omega}_T} V_\epsilon = \max\limits_{\Gamma_T} V_\epsilon
$$
令$\epsilon \rightarrow 0^+$,得证。

(2)的证明:主要使用平均值公式。

\end{proof}

\subsubsection{热球}

取$(x,t) \in \mathbb{R}^n_+$,称 $E(x,t; r) = \left\{(y,s) \in \mathbb{R}^n, s \leq t, \; \text{and} \; P(x,t,y,s) \geq \frac{1}{r^n} \right\}$ 为以$(x,t)$为中心的(抛物)热球,其中:
\begin{align}
    P(x,t,y,s) =
    \begin{cases}
      \left(\frac{1}{4\pi a^2(t-s)}\right)^{n/2} e^{-\frac{|x - y|^2}{4a^2(t-s)}}, \quad &s < t \\
      0, \quad &s \geq t
    \end{cases}
\end{align}
因此由$P(x,t,y,s) \geq \frac{1}{r^n}$可以得到:
$$
t > s, \; |x - y|^2 \leq 4a^2(t-s) \left\{n \ln r - \frac{n}{2} \ln [4 \pi a^2(t-s)]\right\}
$$
当$s \rightarrow t^-$时,$y \rightarrow x$,必$\exists s_0$,s.t. $s = s_0$时,$y = x$。

设$E(r) = E(0,0; r), E(1) = E(0,0;1)$,体积:
\begin{enumerate}[(i)]
  \item
  \begin{align}
      \iint_{E(1)} \frac{|y|^2}{s^2} \intd y \intd s = \int_{-\frac{1}{4\pi a^2}}^0 \intd s \frac{1}{s^2} \int_{B_{r(s)}(0)} y^2 \intd y = 4a^2
  \end{align}
  \item 令
  \begin{align}
      \psi(y,s) = n \ln r - \frac{n}{2} \ln 4 \pi a^2(-s) + \frac{|y|^2}{4 a^2 s}
  \end{align}
  则$\psi = 0$ on $\partial E(r)$;
  \item $\forall (x,t) \in \Omega_T, \exists r > 0$,s.t. $E(x,t;r) \subset \Omega_T$;
  \item 令$y' = \frac{y - x}{r}, s' = \frac{s-t}{r^2}$,则
  \begin{align}
      E(X,t;r) \longrightarrow E(0,0;r)
  \end{align}
  且
  \begin{align}
      \iint_{E(x,t;r)} \frac{|y - x|^2}{(t-s)^2} \intd y \intd s = r^n \iint_{E(1)} \frac{|y|^2}{s^2} \intd y \intd s = 4a^2r^n
  \end{align}
\end{enumerate}

\subsubsection{平均值定理}

\begin{thm}
\label{thm-average}
设$u \in \mathcal{C}^{2.1}(\Omega)$,在$\Omega_T$中满足$u_t - a^2 \triangle u \leq 0$,则$\forall E(x,t;r) \in \Omega_T$有:
\begin{align}
\label{equa4-8-1}
    u(x,t) \leq \frac{1}{4a^2 r^n} \iint_{E(x,t;r)} u(y,s) \frac{|y - x|^2}{(s-t)^2} \intd y \intd s
\end{align}
\end{thm}

\begin{proof}
  令$\phi(r) \triangleq \frac{1}{4a^2 r^n} \iint_{E(x,t;r)} u(y,s) \frac{|y - x|^2}{(s-t)^2} \intd y \intd s$,则$\lim\limits_{r \rightarrow 0}\phi(r) = u(x,t)$,因此只需要证明$\phi'(r) \geq 0$。

  不妨设$x = 0, t = 0$,则
  \begin{align*}
      \phi'(r)
      &= \frac{d}{d r} \left[\iint_{E(1)} u(r y, r^2 s) \frac{|y|^2}{s^2} \intd y \intd s \right] \frac{1}{4a^2} \\
      &= \frac{1}{4a^2} \iint_{E(1)} \sum_{i=1}^n u_{x_i}(r y, r^2 s) y_i \frac{|y|^2}{s^2} \intd y \intd s + \frac{1}{4a^2} \iint_{E(1)} u_t(ry, r^2s) 2r \frac{|y|^2}{s} \intd y \intd s \\
      &= \frac{1}{4a^2 r^{n+1}} \iint_{E(r)} \sum_{i=1}^n u_{x_i}(y,s) y_i \frac{|y|^2}{s^2} \intd y \intd s + \frac{1}{4a^2 r^{n+1}} \iint_{E(r)} u_t(y,s) 2 \frac{|y|^2}{s} \intd y \intd s \\
      :&= A + B
  \end{align*}
因为
\begin{align}
    \psi_{y} &= \frac{y}{2a^2s} \\
    \psi_s &= - \frac{n}{2s} - \frac{|y|^2}{4a^2s^2}
\end{align}
并且$\sum_{i=1}^n \psi_{y_i} y_i = \frac{|y|^2}{2a^2s}$,所以我们可以得到
\begin{align*}
    B &= \frac{1}{r^{n+1}} \iint_{E(r)} u_s \sum_{i=1}^n y_i \psi_{y_i} \intd y \intd s \\
      &= - \frac{1}{r^{n+1}} \iint_{E(r)} \left[\sum_{i=1}^n u_{s y_i} y_i + n u_s \right] \psi \intd y \intd s, \quad \text{固定$s$对$y$用分部积分(边界上$\psi$为0)} \\
 &= - \frac{1}{r^{n+1}} \iint_{E(r)} \left[n u_s \psi + \left(\frac{|y|^2}{4s^2a^2} + \frac{n}{2s}\right)\sum_{i=1}^n u_{y_i} y_i\right] \intd y \intd s , \quad \text{对$s$用分部积分}\\
      &= - \frac{1}{r^{n+1}} \iint_{E(r)} \left(n u_s \psi + \frac{n}{2s}\sum_{i=1}^n u_{y_i} y_i\right) \intd y \intd s - A \\
      &\geq - \frac{1}{r^{n+1}} \iint_{E(r)} \left(n a^2 \triangle_y u \psi + \frac{n}{2s}\sum_{i=1}^n u_{y_i} y_i\right) \intd y \intd s - A, \quad (u_t - a^2 \triangle_x u \leq 0) \\
\end{align*}
因此可以得到
\begin{align}
    \phi'(r) = A + B &\geq - \frac{n}{r^{n+1}} \iint_{E(r)} \left(a^2 \triangle_y u \psi + \frac{1}{2s}\sum_{i=1}^n u_{y_i} y_i\right) \intd y \intd s \notag \\
    &= 0, \quad \text{后一部分对$y$用分部积分}
\end{align}
\end{proof}

\begin{note}
如果$u_t - a^2 \triangle u = 0$,那么
$$
u(x,t) = \frac{1}{4a^2 r^n} \iint_{E(x,t;r)} u(y,s) \frac{|y - x|^2}{(s-t)^2} \intd y \intd s
$$
\end{note}

\subsubsection{极值原理 revisit}

利用定理(\ref{thm-average})可以证明定理(\ref{thm-4-7})的第二部分,
\begin{proof}
取$r > 0$, s.t. $E(x_0,t_0; r) \subset \Omega_T$。
\begin{enumerate}[Step I.]
  \item 根据定理(\ref{thm-average}),
  $$
  u(x_0,t_0) \leq \frac{1}{4a^2 r^n} \iint_{E(x_0,t_0;r)} u(y,s) \frac{|y - x|^2}{(s-t)^2} \intd y \intd s
  $$
  若$\forall (y_0, s_0) \in E(x_0,t_0; r)$,$u(y_0, s_0) \leq u(x_0,t_0)$,
  那么我们能够得到,
  \begin{align}
      \frac{1}{4a^2 r^n} \iint_{E(x_0,t_0;r)} u(y,s) \frac{|y - x|^2}{(s-t)^2} \intd y \intd s \leq \frac{1}{4a^2 r^n} \iint_{E(x_0,t_0;r)} u(x_0,s_0) \frac{|y - x_0|^2}{(s-t_0)^2} \intd y \intd s = u(x_0,t_0)
  \end{align}
  因此$u(y,s) = u(x_0,t_0)$ in $E(x_0,t_0;r)$。

  \item 因为$\Omega$连通,所以$\forall (y_0,s_0) \in \Omega_{t_0}, s_0 < t_0$,

  利用折线段L将$(x_0,t_0)$和$(y_0,s_0)$连接起来,再用有限热球将L覆盖,后一个的热球一定是前一个热球内部,对每一个热球仍用Step I可以得到
  $$
  u(y_0, s_0) = u(x_0,t_0)
  $$
\end{enumerate}
\end{proof}

\begin{cor}
设$u(x,t) \in \mathcal{C}^{2.1} (Q_T) \cap \mathcal{C}^1(\overline{Q}_T)$,满足
\begin{align}
    u_t - a^2 \triangle u \geq 0, \; \text{in} \; \Omega_T
\end{align}
则有:
\begin{enumerate}[(1)]
  \item
  \begin{align}
      \min\limits_{\overline{\Omega}_T} u = \min\limits_{\Gamma_T} u
  \end{align}
  \item \textbf{强极值原理.} 若存在$(x_0,t_0) \in \Omega_T$,s.t. $u(x_0,t_0) = \min\limits_{\overline{\Omega}_T}$,则
  \begin{align}
      u(x,t) \equiv u(x_0,t_0), \quad \forall (x,t) \in \overline{\Omega}_{t_0}
  \end{align}
\end{enumerate}
\end{cor}

\begin{cor} \textbf{比较原理.} 设$u(x,t) \in \mathcal{C}^{2.1} (Q_T) \cap \mathcal{C}^1(\overline{Q}_T)$,满足
\begin{align}
    u_{it} - a^2 \triangle u_i = f_i, \; (i = 1,2) \; \; \text{in} \; \Omega_T
\end{align}
如果$f_1 \geq f_2$ in $\Omega_T$,$u_1 \bigg|_{\Gamma_T} \geq u_2 \bigg|_{\Gamma_T}$,则$u_1 \geq u_2$ in $\overline{\Omega_T}$。
\end{cor}

\begin{proof}
令$V = u_1 - u_2$,则
$$
V_t - a^2 \triangle V = f_1 - f_2 \geq 0 \; \text{in} \; \Omega_T
$$
又因为$V\bigg|_{\Gamma_T} \geq 0$,因此由定理(\ref{thm-4-7}),
$$
\min\limits_{\overline{\Omega}_T} V = \min\limits_{\Gamma_T} V
$$
因此$V \geq 0$ in $\overline{\Omega_T}$。
\end{proof}

\subsubsection{最大模估计}

设$L u = u_t - a^2 \triangle u = f$ in $\Omega_T$,令$V_k = u - kt$,则
$$
L V_k = L u - k = f - k, \quad k_M = \sup\limits_{\Omega_T} f^+, k_m = \inf\limits_{\Omega_T} f^-
$$
那么$L V_{k_M} \leq 0$,
$$
V_{k_M}(x,t) \leq \max\limits_{\Gamma_T} V\bigg|_{\Gamma_T} \leq \max\limits_{\Gamma_T} u, \;\; \forall (x,t) \in \overline{\Omega_T}
$$
同理,$L V_{k_m} \geq 0$,
$$
V_{k_m}(x,t) \geq \min\limits_{\Gamma_T} V\bigg|_{\Gamma_T} \geq \min\limits_{\Gamma_T} u, \;\; \forall (x,t) \in \overline{\Omega_T}
$$
所以
\begin{align}
\label{equa4-7-3}
    T \inf\limits_{\Omega_T} f^- + \min\limits_{\Gamma_T} u \leq u(x,t) \leq \max\limits_{\Gamma_T} u + T \sup\limits_{\Omega_T} f^+, \; \; \forall (x,t) \in \Omega_T
\end{align}

\begin{thm} \textbf{最大模估计.} 设$u(x,t) \in \mathcal{C}^{2.1} (Q_T) \cap \mathcal{C}^1(\overline{Q}_T)$,且$u_t - a^2 \triangle u = f$ in $\Omega_T$,则有(\ref{equa4-7-3})。特别地,有
\begin{align}
    \max\limits_{\Omega_T} |u| \leq \max\limits_{\Gamma_T} |u| + T \sup\limits_{\Omega_T} |f|
\end{align}
\end{thm}

\subsection{柯西问题}

$Q_T = \mathbb{R}^n \times(0,T], \overline{Q}_T = \mathbb{R}^n \times[0,T]$,考虑问题
\begin{align}
    \label{Q4-8-1}
    L u := u_t - a^2 \triangle u \leq 0, \quad \text{in} \; Q_T
\end{align}

\textbf{(1) 极值原理}

\begin{thm}
设$u(x,t) \in \mathcal{C}^{2.1} (Q_T) \cap \mathcal{C}^1(\overline{Q}_T)$,满足(\ref{Q4-8-1}),则有
\begin{enumerate}[(i)]
  \item 如果$\exists (x_0,t_0) \in Q_T$ s.t. $u(x_0,t_0) = \sup\limits_{Q_T} u$,则
  \begin{align}
      u \equiv u(x_0,t_0) \; \; \text{in} \; \overline{Q}_{t_0}
  \end{align}
  \item 如果$u$满足条件$E(A,B)$:$\exists A,B > 0$ s.t.
  $$
  u(x,t) \leq A E^{B|x|^2}, \quad \forall (x,t) \in Q_T
  $$
  则有
  \begin{align}
      \sup\limits_{Q_T} u = \sup\limits_{\mathbb{R}^n} u(x,0)
  \end{align}
\end{enumerate}
\end{thm}

\begin{proof}
(1) $\forall R > 0$,对于$u$在$B(0,R) \times (0,T]$中,使用定理(\ref{thm-4-7})的第二部分,可以得到
$$
u \equiv u(x_0,t_0) \; \; \text{in} \; B(0,R) \times (0,T]
$$
再将$R \rightarrow \infty$,得证。

(2) 令$B_R = B(0,R)$,令
$$
v(x,t) = \left(\frac{1}{4\pi a^2(T + \mu -t)}\right)^{n/2} e^{\frac{|x|^2}{4a^2(T + \mu -t)}}
$$
$\mu$非常小,则易证$L v = 0$ in $Q_{\epsilon T}, \forall \epsilon > 0$。设$w(x,t) = u(x,t) - \epsilon v(x,t)$,则$L w \leq 0$ in $Q_T$。
\begin{align}
    w\bigg|_{\partial B_R \times (0,T)} \leq A e^{BR^2} - \epsilon \left(\frac{1}{4\pi a^2 (T + \mu)}\right)^{n/2} e^{\frac{R^2}{4a^2 (T + \mu)}}
\end{align}
\begin{enumerate}[Step 1.]
  \item 取$T_0 + \mu = \frac{1}{8a^2B}$,则当$T \leq T_0$,有
  $$
  w\bigg|_{\partial B_R \times (0,T)} \leq A e^{BR^2} - \epsilon \left(\frac{1}{4\pi a^2 T_0}\right)^{n/2} e^{2BR^2}
  $$
  因此当$R$充分大的时候,显然有$w\bigg|_{\partial B_R \times (0,T)} \to - \infty$。

  在$\partial B_R \times (0,T)$,对$w$使用定理(\ref{thm-4-7})的第一部分,可以得到$w$的极值在抛物界面中的下底面达到。
  \begin{align}
      \max\limits_{B_R \times (0,T)} w = \max\limits_{\partial B_R \times (0,T) \cup (\mathbb{R}^n \times \{0\})} w \leq \sup\limits_{\mathbb{R}^n} u(x,0)
  \end{align}
  将$R \rightarrow \infty$,得到$\sup\limits_{Q_T} w \leq \sup\limits_{\mathbb{R}^n} u(x,0)$,即
  $$
  u(x,t) - \epsilon v(x,t) \leq \sup\limits_{\mathbb{R}^n} u(x,0), \quad \forall (x,t) \in Q_T
  $$
  令$\epsilon \rightarrow 0^+$,得到
  \begin{align}
      u(x,t) \leq \sup\limits_{\mathbb{R}^n} u(x,0), \quad \forall (x,t) \in Q_T
  \end{align}
  \item (分层法.) 当$T > T_0$时,将$[0,T]$分成$k$个区间:$[0,T_0], [T_0,2T_0], ..., [(k-1)T_0,T]$,s.t. $T - (k-1)T_0 < T_0$,在$\mathbb{R}^n \times (0,T_0), \mathbb{R}^n \times (T_0,2T_0), ..., \mathbb{R}^n \times ((k-1)T_0,T)$上依次使用Step 1的结论。
\end{enumerate}
\end{proof}

\begin{cor}
设$u(x,t) \in \mathcal{C}^{2.1} (Q_T) \cap \mathcal{C}^1(\overline{Q}_T)$,满足
$$
L u := u_t - a^2 \triangle u \geq 0, \quad \text{in} \; Q_T
$$
,则有
\begin{enumerate}[(i)]
  \item 如果$\exists (x_0,t_0) \in Q_T$ s.t. $u(x_0,t_0) = \inf\limits_{Q_T} u$,则
  \begin{align}
      u \equiv u(x_0,t_0) \; \; \text{in} \; \overline{Q}_{t_0}
  \end{align}
  \item 如果$(-u)$满足条件$E(A,B)$:$\exists A,B > 0$ s.t.
  $$
  u(x,t) \geq - A E^{B|x|^2}, \quad \forall (x,t) \in Q_T
  $$
  则有
  \begin{align}
      \inf\limits_{Q_T} u = \inf\limits_{\mathbb{R}^n} u(x,0)
  \end{align}
\end{enumerate}
\end{cor}

\begin{cor} \textbf{唯一性.} 如果
\begin{align}
    u_t - a^2 \triangle u = f, \quad \text{in} \; Q_T, \; u\bigg|_{t = 0} = \varphi \; \text{in} \; \mathbb{R}^n
\end{align}
在空间$\left\{u \in \mathcal{C}^{2.1} (Q_T) \cap \mathcal{C}^1(\overline{Q}_T) \bigg| |u| \; \text{满足} \; E(A,B)\right\}$中的解是唯一的。
\end{cor}

\subsection{能量估计及其推论}

\subsubsection{初边值问题:$\Omega \in \mathbb{R}^n$为有界开集}

\begin{align}
\label{Q4-9-1}
  \begin{cases}
    u_t - a^2 \triangle u = f, \; &\text{in} \; \Omega_T \\
    u = g, \; &\text{on} \; \partial \Omega \times [0,T) \\
    u \bigg|_{t = 0} = \varphi, \; &\text{in} \; \overline{\Omega}
  \end{cases}
\end{align}
where $u \in \mathcal{C}^{2.1}(\Omega_T) \cap \mathcal{C}(\overline{\Omega}), g \in \mathcal{C}^{2.1}(\overline{\Omega})$,古典解中极值原理与最大模估计,唯一。

令$v = u - g$,则,
\begin{align}
\label{Q4-9-2}
  \begin{cases}
    v_t - a^2 \triangle v = f - g_t + a^2 \triangle g := F, \; &\text{in} \; \Omega_T \\
    v = 0, \; &\text{on} \; \partial \Omega \times [0,T) \\
    v \bigg|_{t = 0} = \varphi - g := \phi, \; &\text{in} \; \overline{\Omega}
  \end{cases}
\end{align}
两边同乘$v$,在$\Omega_T$上积分,$\left(\forall t \in [0,T]\right)$,则
\begin{align*}
    \int_{\Omega_T} \left[\frac{d}{\intd t}\left(\frac{v^2}{2}\right) - a^2 \triangle v \cdot v\right] \intd x \intd s = \int_{\Omega_T} F \cdot v \intd x \intd s
\end{align*}
又因为
\begin{align*}
    \int_{\Omega_T} \frac{d}{\intd t}\left(\frac{v^2}{2}\right) \intd x \intd s
    &= \int_0^t \frac{d}{\intd t} \int_\Omega \frac{v^2}{2} \intd x, \quad \text{求导和积分交换} \\
    &= \int_\Omega \frac{v^2(x \cdot t)}{2} \intd x - \frac{1}{2} \int_\Omega \phi^2 \intd x
\end{align*}
且
\begin{align*}
    - a^2 \int_{\Omega_t} \triangle v \cdot v \intd x \intd s
    &= -a^2 \int_{\Omega_t} \left(\mathrm{div}(\triangledown v \cdot v) - |\triangledown v|^2\right) \intd x \intd s \\
    &= -a^2 \int_0^t \int_{\partial \Omega} (\triangledown v \cdot v) \vec{n} \intd S_x \intd s + a^2 \int_0^t \int_\Omega |\triangledown v|^2 \intd x \intd s \\
    &= a^2 \int_0^t \int_\Omega |\triangledown v|^2 \intd x \intd s, \quad (v = 0 \; \text{on} \; \partial \Omega)
\end{align*}
因此
\begin{align*}
    \int_\Omega v^2(x,t) \intd x + 2 a^2 \int_0^t \int_\Omega |\triangledown v|^2 \intd x \intd s \leq \int_\Omega \phi^2 \intd x + \int_0^t \int_\Omega F^2 + v^2 \intd x \intd s
\end{align*}
由Cronwall,
\begin{align*}
    \int_\Omega v^2(x,t) \intd x &\leq \int_0^t \int_\Omega v^2 \intd x \intd s + \int_{\Omega_t} F^2 \intd x \intd s + \int_\Omega \phi^2 \intd x \\
    \therefore \; \int_0^t \int_\Omega v^2(x,t) \intd x \intd s &\leq (e^t - 1) \left[\int_{\Omega_t} F^2 \intd x \intd s + \int_{\Omega_t} \phi^2 \intd x \intd s \right]
\end{align*}
代入(\ref{Q4-9-2}),由于$\forall t \in [0,T)$,
\begin{align}
    \int_\Omega v^2(x,t) \intd x + 2 a^2 \int_{\Omega_t} |\triangledown v|^2 \intd x \intd s \leq e^2 \left[\int_{\Omega_t} F^2 \intd x \intd s + \int_\Omega \phi^2 \intd x \right]
\end{align}
又因为$v = u - g$,
\begin{align}
\label{equa4-9-3}
    \int_\Omega u^2(x,t) \intd x + 2 a^2 \int_{\Omega_t} |\triangledown u|^2 \intd x \intd s \leq 2 \left[\int_{\Omega_t} F^2 \intd x \intd s + \int_\Omega \phi^2 \intd x + \int_\Omega g^2(x,t) \intd x + \int_{\Omega_t} |\triangledown g|^2 \intd x \intd s \right]
\end{align}

\begin{thm}
    设 $u \in \mathcal{C}^{2.1}(\Omega_T) \cap \mathcal{C}(\overline{\Omega})$,则我们有能量估计(\ref{equa4-9-3})。
\end{thm}

\begin{note}
  \begin{enumerate}
    \item 边界条件 $u = g$ on $\partial \Omega \times [0,T]$可以换成 $\frac{\partial u}{\partial \vec{n}} = g$ on $\partial \Omega \times [0,T]$;
    \item 对广义解也成立;
    \item 能量估计 $\Longrightarrow$ 解的唯一性;
    \item 稳定性:设$\{u_k\}_{k=0}^\infty \in \mathcal{C}^{2.1}(\Omega_T) \cap \mathcal{C}(\overline{\Omega})$满足,
    \begin{align*}
        \begin{cases}
            u_{kt} - a^2 \triangledown u_k = f_k, \; &\text{in} \; \Omega_T \\
            u_k = g_k, \; &\text{on} \; \partial \Omega \times (0,T) \\
            u_k \bigg|_{t = 0} = \varphi_k, \; &\text{in} \; \overline{\Omega}
        \end{cases}\quad \quad k = 0,1,2, ...
    \end{align*}
    如果 $f_k \rightarrow f_0$, $g_k \rightarrow g_0$, $\triangledown g_k \rightarrow \triangledown g_0$ in $\mathcal{L}^2(\Omega)$, $\varphi_k \rightarrow \varphi_0$ in $\mathcal{L}^2(\Omega)$,则有
    \begin{align}
        \lim\limits_{k \rightarrow \infty} \int_{\Omega_T} \left[|\triangledown u_k - \triangledown u_0|^2 + |u_k - u_0|^2\right] \intd x \intd t = 0
    \end{align}
  \end{enumerate}
\end{note}

\subsubsection{Cauchy问题:$Q = \mathbb{R}^n \times (0,T]$}
\begin{align}
\label{Q4-9-3}
  \begin{cases}
    u_t - a^2 \triangle u = f, \; &\text{in} \; Q_T \\
    u \bigg|_{t = 0} = \varphi
  \end{cases}
\end{align}
$u(x,t) \in \mathcal{L}^2(\mathbb{R}^n), \triangledown u \in \mathcal{L}^2(Q_T)$。记$\mathcal{L}^2(0,T; H^1(\mathbb{R}^n)) = \left\{u \in \mathcal{L}^2(Q_T): u_{x_i} \in \mathcal{L}^2(Q_T), \; i = 1,...,n \right\}$,两边同乘$u$在$Q_T$上积分,
$$
\int_0^t \int_{\mathbb{R}^n} \frac{d}{d t} \left(\frac{u^2}{2}\right) - a^2 \int_0^t \int_{\mathbb{R}^n} \triangle u \cdot u = \int_{Q_t} f \cdot u \intd x \intd t \leq \frac{1}{2} \int_{Q_t} (f^2 + u^2) \intd x \intd t
$$
$$
A := \int_0^t \int_{\mathbb{R}^n} \frac{d}{d t} \left(\frac{u^2}{2}\right) = \frac{1}{2} \left(\int_{\mathbb{R}^n} u^2(x,t) \intd x - \int_{\mathbb{R}^n} \varphi^2 \intd x\right)
$$
\begin{align*}
B :&= - a^2 \int_0^t \int_{\mathbb{R}^n} \triangle u \cdot u \intd x \intd s \\
&= -a^2 \int_0^t \intd s \lim\limits_{k \rightarrow \infty} \int_{B_k(0)} \triangle u \cdot u \intd x \\
&= -a^2 \lim\limits_{k \rightarrow \infty} \int_0^t \intd s \int_{\partial B_k(0)} \frac{\partial u}{\partial \vec{n}} u \intd x + a^2 \lim\limits_{k \rightarrow \infty} \int_0^t \intd s \int_{B_k(0)} |\triangledown u|^2 \intd x
\end{align*}
因为,
\begin{align*}
    &\int_{\mathbb{R}^n} u^2 = \int_0^\infty \intd r \int_{\partial B_r(0)} u^2 \intd s \\
    & r \rightarrow \infty, \quad \int_{\partial B_{r}(0)} u^2 \intd S_x \rightarrow 0 \\
  \therefore \; \; &-a^2 \lim\limits_{k \rightarrow \infty} \int_0^t \intd s \int_{\partial B_k(0)} \frac{\partial u}{\partial \vec{n}} u \intd x = 0
\end{align*}
因此,
\begin{align}
    B = a^2 \lim\limits_{k \rightarrow \infty} \int_0^t \intd s \int_{B_k(0)} |\triangledown u|^2 \intd x
\end{align}

\begin{thm}
  设 $u \in \mathcal{C}^{2.1}(\Omega_T) \cap \mathcal{C}(\overline{\Omega}) \cap \mathcal{L}^2(0,T; H^1(\mathbb{R}^n))$,满足(\ref{Q4-9-3}),则$\forall t \in [0,t]$,有
  \begin{align}
      \int_{\mathbb{R}^n} u^2(x,t) \intd x + 2 a^2 \int_0^t \int_{\mathbb{R}^n} |\triangledown u|^2 \intd x \intd s \leq e^t \left[\int_0^t \int_{\mathbb{R}^n}f^2 \intd x \intd s + \int_{\mathbb{R}^n}\varphi^2 \intd x \right]
  \end{align}
\end{thm}

\begin{note}
    \begin{enumerate}
      \item 问题(\ref{Q4-9-3})在$\mathcal{C}^{2.1}(\Omega_T) \cap \mathcal{C}(\overline{\Omega}) \cap \mathcal{L}^2(0,T; H^1(\mathbb{R}^n))$中的解是唯一的;
      \item 稳定性;
      \item 弱解
      令$E = \{u \in \mathcal{L}^2(0,T; H^1(\mathbb{R}^n)): u_t \in \mathcal{L}^2(Q_T)\}$,称$u \in E$为(\ref{Q4-9-3})的弱解,如果$\forall \varphi \in E$,
      $$
      \int_{\Omega_t} \left(u_t \varphi + a^2 \triangledown u \cdot \triangledown \varphi\right) \intd x = \int_{Q_t} f \varphi \intd x \intd s
      $$
      且
      $$
      \lim_{t \rightarrow 0^+} \int_{\mathbb{R}^n} |u(x,t) - \varphi(x)|^2 \intd x = 0
      $$
    \end{enumerate}
\end{note}

\begin{note}
  能量不等式可能会用到Young不等式,
  \begin{align*}
  |ab| &= |a \epsilon \cdot \frac{b}{\epsilon}|^2 \leq \frac{\epsilon^2 a^2 + \frac{b^2}{\epsilon^2}}{2}, \forall \epsilon > 0 \\
  |ab| &\leq \frac{a^p}{p} + \frac{b^q}{q}, \quad \frac{1}{p} + \frac{1}{q} = 1
  \end{align*}
\end{note}

\newpage

\section{Poisson方程}

Poisson方程的应用:
\begin{enumerate}[(1)]
  \item 热方程(波方程)的稳态:$t \rightarrow \infty, u_t \rightarrow 0, u_{tt} \rightarrow 0$
  $$
  - \triangle u = f
  $$
  \item Dirichlet 问题
  $$
  \min_{u} \int_\Omega \left(\frac{|\triangledown u|^2}{2} + f(x) u\right) \intd x
  $$
  \item 热电(磁)现象模拟,调和函数,
  $$
  - \triangle u = 0
  $$
\end{enumerate}

\subsection{基本解}

\subsubsection{定义}

\begin{definition}
    如果$\forall \xi \in \mathbb{R}^n$,$\Gamma(x,\xi)$作为$x$的函数,满足
    $$
    - \triangle \Gamma(x,\xi) = \delta(x - \xi), \forall x \in \mathbb{R}^n
    $$
    则称$\Gamma(x,\xi)$为Poisson方程的一个基本解。
\end{definition}

\subsubsection{形式推导}

$\delta$函数:单位点热源的温度分布
\begin{align*}
    1
    &= \int_{\partial B_\epsilon(0)} q \, \mathrm{d} \vec{s} \\
    &= - \int_{\partial B_\epsilon(0)} \triangledown w_\epsilon \, \mathrm{d} \vec{s} \\
    &= - \int_{\partial B_\epsilon(0)} \frac{\partial w_\epsilon}{\partial \vec{n}} \intd s \\
    &= - \frac{\partial w_\epsilon}{\partial \vec{n}} \bigg|_{\partial B_\epsilon(0)} \cdot \left| \partial B_\epsilon(0)\right| \\
    &= - \epsilon^{n-1} w_n \cdot \frac{\partial w_\epsilon}{\partial \vec{n}} \bigg|_{\partial B_\epsilon(0)}
\end{align*}
其中
\begin{align*}
w_n &= |\partial B_1(0)| = \frac{2 \pi^{\frac{n}{2}}}{\Gamma(\frac{n}{2})} \\
\text{where} \; \; \Gamma(t) &= \int_0^\infty e^{-x} x^{t-1} \intd x
\end{align*}
其中$w_n$为$n$维单位球球面面积。并且$\Gamma(s+1) = s \Gamma(s)$,

\textcolor{red}{注:$n$维球球面面积为$r^{n-1} w_n$,球的体积为
\begin{align}
    V = \int_0^r t^{n-1} w_n \intd t = \frac{r^n}{n} w_n
\end{align}
}

因此可以得到
\begin{align}
    \label{Q5-1-1}
    \begin{cases}
        - \triangle w_\epsilon = 0, \; \text{in} \; \mathbb{R}^n \backslash B_\epsilon(0) \\
        \frac{\partial w_\epsilon}{\partial \vec{n}} \bigg|_{\partial B_\epsilon(0)} = \frac{-1}{\epsilon^{n-1}w_n}
    \end{cases}
\end{align}
要找到$w_\epsilon(x) = w_\epsilon(|x|)$,(关于球面对称)

令$w_\epsilon(x) = F(r), r = |x| = \sqrt{x_1^2 + \cdots + x_n^2}$,则$w_{\epsilon x_i} = F'(r) \frac{x_i}{r}$,$w_{\epsilon x_i x_i} = F''(r) \frac{x_i^2}{r^2} + F'(r)\frac{r - \frac{x^2_i}{r}}{r^2}$,因此
$$
\triangle w_\epsilon = F''(r) + F'(r) \frac{(n-1)r^2}{r^3} = F''(r) + F'(r)\frac{n-1}{r}
$$
因此,
\begin{align}
    \label{Q5-1-2}
    \begin{cases}
        f'(r) + \frac{n-1}{r} f(r) = 0, \quad r > \epsilon \\
        f(r) \bigg|_{r = \epsilon} = - \frac{1}{w_n \epsilon^{n-1}}
    \end{cases}
\end{align}
令$f(r) = F'(r)$,$\frac{f'(r)}{f(r)} = -\frac{n-1}{r}$,因此可以得到
\begin{align*}
\ln f(r) &= \ln r^{1-n} + c \\
\Longrightarrow \; f(r) &= c r^{1-n} = F'(r)
\end{align*}
积分,得到
\begin{align*}
    F(1) = \begin{cases}
      c_2 \ln r + c, \quad &n = 2 \\
      c_n r^{2-n} + c, \quad &n > 2
    \end{cases}
\end{align*}
令$c = 0$,得到
\begin{align*}
    \begin{cases}
        c_2 = - \frac{1}{2\pi} \\
        c_n = \frac{1}{(n-2)w_n}
    \end{cases}
\end{align*}
因此
\begin{align}
    w_\epsilon(x) = \begin{cases}
      - \frac{1}{2 \pi} \ln |x|, \; &n = 2\\
      \frac{|x|^{2-n}}{(n-2)w_n}, \; &n \geq 3
    \end{cases}\quad \xrightarrow{\epsilon \rightarrow 0} w(x)
\end{align}
因此我们可以解得
\begin{align}
    \Gamma(x,\xi) = w(x - \xi) = \begin{cases}
    - \frac{1}{2 \pi} \ln |x - \xi|, \; &n = 2\\
    \frac{1}{(n-2)w_n|x - \xi|^{n-2}}, \; &n \geq 3
    \end{cases}
 \end{align}

 \subsubsection{基本解的严格证明}

 \begin{lemma}
 \label{lemma5-1-1}
 \begin{enumerate}[(i)]
   \item $\forall \xi \in \mathbb{R}^n$,$\Gamma(x,\xi)$作为$x$的函数$\in \mathcal{L}_{loc}^1(\mathbb{R}^n)$;
   \item $- \triangle_x \Gamma(x,\xi) = 0$,$x \neq \xi$。
 \end{enumerate}
 \end{lemma}

 不妨设$\xi = 0$,则
 $$
 - \triangle \Gamma(x) = \delta(x), \Gamma(x) = \Gamma(x,0)
 $$
 i.e.
 \begin{align}
     \label{equa5-1-1}
     - \int_{\mathbb{R}^n} \triangle \varphi \cdot \Gamma \intd x = \varphi(0), \forall \varphi \in \mathcal{C}_0^\infty(\mathbb{R}^n)
 \end{align}

 \begin{note}
    Green公式:由高斯公式
    \begin{align}
        \int_\Omega \mathrm{div} \vec{F} \intd x = \int_{\partial \Omega} \vec{F} \vec{n} \intd s
    \end{align}
    令$\vec{F} = \triangledown v \cdot u$,因此得到Green公式
    \begin{align}
        \int_\Omega \left(\triangle v \cdot u + \triangledown v \cdot \triangledown u \right) \intd x = \int_{\partial \Omega} u \cdot \frac{\partial v}{\partial \vec{n}} \intd s
    \end{align}
 \end{note}

 \begin{lemma}
  \label{lemma5-1-2}
    设$\Omega \subset \mathbb{R}^n$有界,$\partial \Omega$分片属于$\mathcal{C}^1$,$u,v \in \mathcal{C}^2(\Omega) \cap \mathcal{C}^1(\overline{\Omega})$,则
    \begin{align}
        \int_\Omega (u \triangle v - v \triangle u) \intd x = \int_{\partial \Omega} \left(u \frac{\partial v}{\partial \vec{n}} - v \frac{\partial u}{\partial \vec{n}}\right) \intd s
    \end{align}
 \end{lemma}

 取$R >0$,s.t. $\mathrm{spt}(\varphi) \in B_R(0)$,(\ref{equa5-1-1})的左端:
 $$
 - \int_{\mathbb{R}^n} \triangle \varphi \cdot \Gamma \intd x = - \int_{B_R(0)} \triangle \varphi \cdot \Gamma \intd x = \lim\limits_{\epsilon \rightarrow 0^+} \int_{B_R(0) \backslash B_\epsilon(0)} \triangle \varphi \cdot \Gamma(x) \intd x
 $$
 而
 \begin{align*}
     - \int_{B_R(0) \backslash B_\epsilon(0)} \triangle \varphi \cdot \Gamma(x) \intd x
     &\xlongequal[]{\mathrm{Lemma} (\ref{lemma5-1-1})} \int_{B_R(0) \backslash B_\epsilon(0)} \left(\varphi \cdot \triangle \Gamma - \triangle \varphi \cdot \Gamma\right) \intd x \\
     &\xlongequal[]{\mathrm{Lemma} (\ref{lemma5-1-2})} \int_{\partial B_R(0)} \left(\varphi \frac{\partial \Gamma}{\partial \vec{n}} - \Gamma \frac{\partial \varphi}{\partial \vec{n}}\right) \intd s + \int_{\partial B_\epsilon(0)} \left(\varphi \frac{\partial \Gamma}{\partial \vec{n}_\Gamma} - \Gamma \frac{\partial \varphi}{\partial \vec{n}_\Gamma}\right) \intd s
 \end{align*}
其中大球里挖掉小球的边界,外球的外法向和内球的内法向;$\vec{n}_\Gamma$指向原点。又因为,
\begin{align*}
    - \int_{\partial B_\epsilon(0)} \Gamma \cdot \frac{\partial \varphi}{\partial \vec{n}_\Gamma} \intd s = \begin{cases}
      \displaystyle\frac{\ln\epsilon}{2\pi} \int_{\partial B_\epsilon(0)} \frac{\partial \varphi}{\partial \vec{n}_\Gamma} \intd s, \quad & n = 2 \\
       \\
      \displaystyle - \frac{1}{(n-2)w_n \epsilon^{n-2}} \int_{\partial B_\epsilon(0)} \frac{\partial \varphi}{\partial \vec{n}_\Gamma} \intd s, \quad & n \geq 3 \\
    \end{cases}
\end{align*}
当$\epsilon \rightarrow 0^+$时,
\begin{align*}
    - \int_{\partial B_\epsilon(0)} \Gamma \cdot \frac{\partial \varphi}{\partial \vec{n}_\Gamma} \intd s = \begin{cases}
      \displaystyle\frac{\ln\epsilon}{2\pi} \epsilon w_n \varphi_x(0) \longrightarrow 0,  \quad & n = 2 \\
       \\
      \displaystyle - \frac{1}{(n-2)w_n \epsilon^{n-2}} w_n \epsilon^{n-1} \varphi_x(0) \longrightarrow 0, \quad & n \geq 3 \\
    \end{cases}
\end{align*}
再考虑到,
\begin{align*}
    \frac{\partial \Gamma}{\partial \vec{n}_\Gamma}  \bigg|_{\partial B_\epsilon(0)} =
    \begin{cases}
        \displaystyle \frac{1}{2\pi \epsilon} = \frac{1}{|\partial B_\epsilon|}, \quad &n = 2 \\
          \\
        \displaystyle\frac{1}{(n-2)w_n} \frac{(2-n)}{|x|^{n-1}} \sum_i \frac{x_i^2}{|x_i|^2} = \frac{1}{\epsilon^{n-1}w_n} = \frac{1}{|\partial B_\epsilon|}, \quad &n \geq 3
    \end{cases}
\end{align*}
因此,令$R \rightarrow \infty, \Gamma \rightarrow 0$,则
\begin{align}
    - \int_{\mathbb{R}^n} \triangle \varphi \cdot \Gamma \intd x
    &= \lim\limits_{\epsilon \rightarrow 0^+}\int_{\partial B_\epsilon(0)} \varphi \frac{\partial \Gamma}{\partial \vec{n}_\Gamma} \notag \\
    &= \lim\limits_{\epsilon \rightarrow 0^+} \frac{1}{|\partial B_\epsilon|} \int_{\partial B_\epsilon(0)} \varphi(x) \intd S_x \notag \\
    &= \varphi(0)
\end{align}

\begin{thm}
    $\Gamma(x,\xi)$在广义函数的定义下,满足
    \begin{align}
        - \triangle_x \Gamma(x,\xi) = \delta(x - \xi), \quad \forall x - \xi \in \mathbb{R}^n
    \end{align}
\end{thm}

$\forall \xi \in \Omega$,$\forall u \in \mathcal{C}^2(\Omega) \cap \mathcal{C}^1(\overline{\Omega})$。用$B_\epsilon(\xi)$代替$B_\epsilon(0)$,$\Omega$代替$B_k(0)$,$u$代替$\varphi$,$\Longrightarrow$

\begin{thm}
\label{thm5-1-4}
    设$\Omega \in \mathbb{R}^n$有界开集,$\partial \Omega$分片属于$\mathcal{C}^1$,$u \in \mathcal{C}^2(\Omega) \cap \mathcal{C}^1(\overline{\Omega})$,则$\forall \xi \in \Omega$,有
    \begin{align}
        u(\xi) = \int_{\partial \Omega} \left(\Gamma \frac{\partial u}{\partial \vec{n}} - u \frac{\partial \Gamma}{\partial \vec{n}}\right) \intd S_x - \int_\Omega \triangle u \cdot \Gamma(x, \xi) \intd x
    \end{align}
\end{thm}

\subsubsection{Newman问题有解的必要条件}

\begin{align}
\label{Q5-1-5}
    \begin{cases}
        - \triangle u = f, &\; \text{in} \; \Omega \\
        \frac{\partial u}{\partial \vec{n}} = \varphi, &\; \text{on} \; \partial \Omega
    \end{cases}
\end{align}
根据定理(\ref{thm5-1-4}),该问题的解为
\begin{align}
    u(\xi) = \int_{\partial \Omega} \left[\Gamma \varphi - u(x) \frac{\partial \Gamma}{\partial \vec{n}}\right] \intd S_x + \int_\Omega f\Gamma \intd x
\end{align}

再考虑
\begin{align}
\label{Q5-1-6}
    \begin{cases}
        - \triangle u + \lambda u = f, &\; \text{in} \; \Omega \\
        \frac{\partial u}{\partial \vec{n}} = \varphi, &\; \text{on} \; \partial \Omega
    \end{cases}
\end{align}
其中$f,\varphi, r \in \mathbb{R}$已知。

\begin{definition}
    当$f = 0, \varphi = 0$时,式(\ref{Q5-1-6})的解$u_\lambda$满足$- \triangle u + \lambda u = 0$ in $\Omega$, $u \bigg|_{\partial \Omega} = 0$。如果$u_\lambda \neq 0$,则称$u_\lambda$为与(\ref{Q5-1-5})对应齐次问题的特征函数,$\lambda$为特征值。
\end{definition}
由Green公式,
\begin{align*}
    &\Longrightarrow \; \int_\Omega \left(u \triangle u_\lambda - u_\lambda \triangle u\right) \intd x = \int_{\partial \Omega} \left(u \frac{\partial u_\lambda}{\partial \vec{n}} - u_\lambda \frac{\partial u}{\partial \vec{n}}\right) \intd S_x \\
    &\Longrightarrow \; \int_\Omega f u_\lambda \intd x + \int_{\partial \Omega} u_\lambda \varphi \intd S_x = 0
\end{align*}

\begin{thm}
    问题(\ref{Q5-1-6})在$\mathcal{C}^2(\Omega) \cap \mathcal{C}^1(\overline{\Omega})$中有解的必要条件是:对于任意满足$- \triangle u + \lambda u = 0$ in $\Omega$, $u \bigg|_{\partial \Omega} = 0$的$u_\lambda \in \mathcal{C}^2(\Omega) \cap \mathcal{C}^1(\overline{\Omega})$,均有
    \begin{align}
        \int_\Omega f u_\lambda \intd x + \int_{\partial \Omega} u_\lambda \varphi \intd S_x = 0
    \end{align}
    \textbf{该条件也是充分的。}
\end{thm}

\subsection{Green函数}

\subsubsection{Dirichlet问题}

\begin{align}
    \label{Q5-2-1}
    \begin{cases}
        - \triangle u = f, \; &\text{in} \; \Omega \\
        u = \varphi, \; &\text{on} \; \partial \Omega
    \end{cases}
\end{align}
取$g(x,\xi) \in \mathcal{C}(\overline{\Omega} \times \overline{\Omega})$ s.t. $g \bigg|_{\partial \Omega} = - \Gamma(x, \xi), \; \forall \xi \in \Omega$,且$- \triangle_x g = 0$ in $\Omega \times \Omega$,对于$u,g$使用Green公式,得到
\begin{align}
    \int_\Omega (g \triangle u - u \triangle_x g ) \intd x = \int_{\partial \Omega} \left(\frac{\partial u}{\partial \vec{n}} g - \frac{\partial g}{\vec{n}} u\right) \intd x
\end{align}
所以结合定理(\ref{thm5-1-4})
\begin{align}
    0 &= \int_{\partial \Omega} \left(\frac{\partial u}{\partial \vec{n}} g - \frac{\partial g}{\vec{n}} u\right) \intd x - \int_\Omega g \triangle u \intd x \notag \\
    u(\xi) &= \int_{\partial \Omega} \left(\Gamma \frac{\partial u}{\partial \vec{n}} - u \frac{\partial \Gamma}{\partial \vec{n}}\right) \intd S_x - \int_\Omega \triangle u \cdot \Gamma(x, \xi) \intd x
\end{align}
再利用·$g \bigg|_{\partial \Omega} = - \Gamma(x, \xi)$
\begin{align}
\label{equ5-2-2}
    u(\xi) = - \int_\Omega \triangle u G(x,\xi) \intd x - \int_{\partial \Omega} u(x) \frac{\partial G}{\partial n_x} (x,\xi) \intd S_x
\end{align}
其中
$$
G(x,\xi) = g(x,\xi) + \Gamma(x,\xi)
$$

\begin{definition}
    如果$g \in \overline{\Omega} \times \overline{\Omega}`'$满足
    \begin{align}
        \begin{cases}
            - \triangle_x g = 0, \; &\forall x,\xi \in \Omega \\
            g \bigg|_{x \in \partial \Omega} = - \Gamma(x,\xi), \; &\text{on} \; \partial \Omega
        \end{cases}
    \end{align}
    则称$G(x,\xi) = g(x,\xi) + \Gamma(x,\xi)$为问题(\ref{Q5-2-1})的Green函数。

    \textbf{注:根据调和函数的性质(最大值和最小值在边界达到),Green函数是唯一的。}
\end{definition}

\begin{thm}
    问题(\ref{Q5-2-1})在$\mathcal{C}^2(\Omega) \cap \mathcal{C}^1(\overline{\Omega})$中如果有解,则解的表达式一定由式(\ref{equ5-2-2})给出,其中$G$为问题(\ref{Q5-2-1})的Green函数。
\end{thm}

\subsubsection{Newman问题}

考虑问题(\ref{Q5-1-5})
\begin{align*}
    \begin{cases}
        - \triangle u = f, &\; \text{in} \; \Omega \\
        \frac{\partial u}{\partial \vec{n}} = \varphi, &\; \text{on} \; \partial \Omega
    \end{cases}
\end{align*}
解为
\begin{align}
    u(\xi) = \int_\Omega f(x) G(x,\xi) \intd x - \int_{\partial \Omega} \varphi(x) G(x,\xi) \intd S_x
\end{align}

\begin{definition}
    如果$g \in \overline{\Omega} \times \overline{\Omega}$满足
    \begin{align}
        \begin{cases}
            - \triangle_x g = 0, \; &\forall x,\xi \in \Omega \\
            \frac{\partial g}{\partial \vec{n}} \bigg|_{x \in \Omega} = - \Gamma(x,\xi), \; &\text{on} \; \partial \Omega
        \end{cases}
    \end{align}
    则称$G(x,\xi) = g(x,\xi) + \Gamma(x,\xi)$为问题(\ref{Q5-1-5})的Green函数。
\end{definition}

\subsubsection{混合问题}

\begin{align}
    \label{Q5-2-3}
    \begin{cases}
        - \triangle u = f \\
        \frac{\partial u}{\partial \vec{n}} = \varphi_1, \; &\text{on} \; \Gamma_1 \\
        u = \varphi_2, \; &\text{on} \; \Gamma_2
    \end{cases}
\end{align}

\begin{definition}
    如果$g \in \overline{\Omega} \times \overline{\Omega}$满足
    \begin{align}
        \begin{cases}
            - \triangle_x g &= 0, \; \forall x,\xi \in \Omega \\
            \frac{\partial g}{\partial \vec{n}} \bigg|_{\Gamma_1} &= - \frac{\partial \Gamma}{\partial \vec{n}} \\
            g \bigg|_{\Gamma_2} &= - \Gamma
        \end{cases}
    \end{align}
    则称$G(x,\xi) = g(x,\xi) + \Gamma(x,\xi)$为问题(\ref{Q5-2-3})的Green函数。
\end{definition}

\subsubsection{Green函数的性质}

我们以Dirichlet为例,介绍Green函数的性质。

\begin{thm}
    Poisson方程的Dirichlet问题的Green函数满足
    \begin{enumerate}[(i)]
        \item $\triangle_x G(x, \xi) = 0, \forall x \neq \xi \in \Omega$,$- \triangle_x G = \delta(x-\xi), \forall x,\xi \in \Omega$;
        \item $\int_{\partial \Omega} \frac{\partial G}{\partial \vec{n}}(x,\xi) \intd S_x = -1$, $\forall \xi \in \Omega$,并且$\lim\limits_{x \rightarrow \xi} \frac{G(x,\xi)}{\Gamma(x,\xi)} = 1$,$\forall \xi \in \Omega$;
        \item $\forall \eta, \xi \in \Omega$,$\eta \neq \xi$,均有$G(\xi,\eta) = G(\eta,\xi)$;
        \item
        \begin{align}
            0 < G(x,\xi) <
            \begin{cases}
            \displaystyle\frac{1}{2\pi} \ln \frac{\text{diam}(\Omega)}{|x - n|}, \quad &n = 2 \\
            \\
            \displaystyle\frac{1}{(n-2) w_n|x - \xi|^{n-2}}, \quad &n \geq 3
            \end{cases}
        \end{align}
    \end{enumerate}
\end{thm}

\begin{proof}
\begin{enumerate}[(i)]
    \item 因为$G(x,\xi) = g(x,\xi) + \Gamma(x,\xi)$,并且$\triangle_x g = 0$ in $\Omega$。
    \begin{align*}
        - \triangle_x \Gamma = \delta(x - \xi), \;\; \triangle_x \Gamma = 0, x \neq \xi
    \end{align*}
    \item $u \equiv 1$,则
    $$
    \begin{cases}
        - \triangle u = 0 \\
        u \bigg|_{\partial \Omega=1} = \varphi
    \end{cases}
    $$
    当$u \in \mathcal{C}^1(\overline{\Omega}) \cap \mathcal{C}^2(\Omega)$,
    \begin{align*}
        1 = - \int_{\partial \Omega} \varphi \frac{\partial G}{\partial \vec{n}} \intd S_x
    \end{align*}
    因为$\forall \xi \in \Omega$,
    \begin{align}
        |g(x,\xi)| \leq \max\limits_{x \in \partial \Omega} |\Gamma(x,\xi)| \leq C(\xi), \; \forall x \in \Omega
    \end{align}
    \item 取$u(x) = G(x,\eta)$,$\Gamma(x) = G(\xi,x)$,令$\Omega_\epsilon = \Omega \backslash \left\{B_\epsilon(\xi) \cup B_\epsilon(\eta)\right\}$,则
    \begin{align}
        \int_{\Omega_\epsilon} ( v \triangle u - u \triangle v) \intd x = \int_{\partial \Omega_\epsilon} \left(\frac{\partial u}{\partial \vec{n}} v - \frac{\partial v}{\partial \vec{n}} u\right) \intd S_x
    \end{align}
    令$\epsilon \rightarrow 0^+$即证。
    \item 任取$\xi \in \Omega$(固定,考虑$x$的函数),因为
    \begin{align*}
        \begin{cases}
            - \triangle_x g = 0, \; &\forall x,\xi \in \Omega \\
            g \bigg|_{x \in \partial \Omega} = - \Gamma(x,\xi), \; &\text{on} \; \partial \Omega
        \end{cases}
    \end{align*}
    所以可以得到
    \begin{align}
        g(x,\xi) < - \Gamma(x,\xi) \bigg|_{x \in \partial \Omega} \leq
        \begin{cases}
            \frac{\ln \text{diam}(\Omega)}{2 \pi}, \quad & n = 2 \\
            0, & n \geq 3
        \end{cases}
    \end{align}
    由此可以推出上界。

    因为调和函数$G(x,\xi)$在外边界恒为0,在内边界,(调和函数一定在边界达到极值,如果在内部达到极值,那么这个调和函数一定是常数。)
    $$
    g(x,\xi)\bigg|_{x \in \partial B_\epsilon(\xi)} + \Gamma(x,\xi)\bigg|_{x \in \partial B_\epsilon(\xi)} > 0
    $$
    当小球半径缩小的时候,趋向于奇点,因此在内边界值很大,所以最小值在外边界取得,为0。
\end{enumerate}

\end{proof}

考虑问题:
\begin{align}
\label{Q5-1-9}
    - \triangle u &= f, \; \text{in} \; \Omega \notag \\
    u \bigg|_{\partial \Omega} &= \varphi
\end{align}
的解为
\begin{align}
    u(\xi) = \int_\Omega G(x,\xi) f(x) \intd x - \int_{\partial \Omega} \frac{\partial G}{\partial n_x}(x,\xi) \varphi(x) \intd S_x
\end{align}
where
\begin{align}
\label{eq5-1-9}
    G(x,\xi) &= \Gamma(x,\xi) + g(x,\xi) \notag \\
    \Gamma(x,\xi) &= \begin{cases}
    - \frac{1}{2\pi} \ln |x - \xi|, \quad &n = 2 \\
    \frac{1}{(n-2) w_n|x - \xi|^{n-2}}, \quad &n \geq 3
    \end{cases}
\end{align}
并且
\begin{align*}
    g &\in \mathcal{C}(\overline{\Omega} \times \overline{\Omega}), \; \text{s.t.} \; \\
    &\begin{cases}
    \triangle_x g(x,\xi) = 0, \; &\forall x,\xi \in \Omega \\
    g \bigg|_{x \in \partial \Omega} = - \Gamma(x,\xi), \; &\forall \xi \in \Omega
    \end{cases}
\end{align*}

\subsubsection{半空间中的Green函数}

令$\overline{\xi}$是$\xi$关于$x_n = 0$的对称点,即若$\xi = (\xi_1, ..., \xi_n)$,则$\overline{\xi} = (\xi_1, ..., \xi_{n-1}, - \xi_n)$。则$g(x,\xi) = - \Gamma(x,\overline{\xi})$为$\mathbb{R}_n^+$的Green函数。

下面验证是解。

\begin{thm}
    设$f \in \mathcal{C}_0^\infty(\mathbb{R}_n^+)$。对于$\forall \alpha > 0$,$\rho \in \mathcal{C}(\mathbb{R}_0^+) \cap \mathcal{L}^\infty(\mathbb{R}_n^+)$,则
    \begin{align}
        u(\xi) = \int_{\mathbb{R}_n^+} G(x,\xi) f(x) \intd x - \int_{\partial \mathbb{R}_n^+} \frac{\partial G}{\partial n_x} (x,\xi) \varphi(x) \intd x, \quad \forall \xi \in \mathbb{R}_n^+
    \end{align}
    是问题
    \begin{align}
        - \triangle u(\xi) &= f(\xi), \; \text{in} \; \mathbb{R}_n^+ \notag \\
        u \bigg|_{\partial \mathbb{R}_n^+} &= \varphi
    \end{align}
    的古典解。其中$G$由(\ref{eq5-1-9})给出。
\end{thm}

\begin{proof}

\textbf{(i)} 证明$- \triangle u(\xi) = f(\xi)$,

\begin{align*}
\triangle_\xi I_1 &= \int_{\mathbb{R}_n^+} \triangle_\xi G(x,\xi) f(x) \intd x, \quad (G(x,\xi) = - \delta(x - \xi)) \\
&= - f(\xi) \\
\triangle_\xi I_2 &= - \int_{\partial \mathbb{R}_n^+} \frac{\partial }{\partial n_x} (\triangle_\xi G) \intd x = 0
\end{align*}

\textbf{(ii)} 证明:$\forall \rho_0 \in \partial \mathbb{R}_n^+ = \mathbb{R}_{n-1}$,$\lim\limits_{\xi \rightarrow \rho_0} I_1 = 0$,趋向于边界时为0。

$\because G(x,\xi) \xrightarrow{\xi \rightarrow \rho_0} 0$,$\therefore \lim\limits_{\xi \rightarrow \rho_0} I_1 = 0$。

再证明
$$
- \lim\limits_{\xi \rightarrow \rho_0} \int_{\partial \mathbb{R}_n^+} \frac{\partial G}{\partial n_x} (x,\xi) \varphi(x) \intd x = \varphi(\rho_0)
$$
以$n \geq 3$为例子,
\begin{align*}
    \frac{\partial G}{\partial n_x} &= \frac{1}{(n-2)w_n} \left[\frac{(-n+2)\frac{x_n - \xi_n}{|x - \xi|}}{|x - \xi|^{n-1}} - \frac{(-n+2)(x_n + \xi_n)}{|x - \overline{\xi}|^n}\right] \\
    &= - \frac{2 \xi_n}{w_n |x - \xi|^n}, \; \text{when} \; x \in \partial \mathbb{R}_n^+, x_n = 0
\end{align*}
因此,
$$
I_2(\xi) = \frac{2 \xi_n}{w_n} \int_{\mathbb{R}_{n-1}} \frac{\varphi(x)}{|x - \xi|^n} \intd x
$$
并且
$$
\frac{2 \xi_n}{w_n} \int_{\mathbb{R}_{n-1}} \frac{1}{|x - \xi|^n} \intd x = 1
$$
那么
\begin{align*}
    \left|I_2(\xi) - \varphi(\rho_0)\right|
    &\leq \frac{2 \xi_n}{w_n} \int_{\mathbb{R}_{n-1}} \frac{|\varphi(x) - \varphi(\rho_0)|}{|x - \xi|^n} \intd x, \quad \xi \in \mathbb{R}_n^+ \\
    &= \frac{2 \xi_n}{w_n} \int\limits_{|x - \rho_0| \geq \frac{\delta}{2}} \frac{|\varphi(x) - \varphi(\rho_0)|}{|x - \xi|^n} \intd x + \int\limits_{|x - \rho_0| \leq \frac{\delta}{2}} \frac{\xi_n|\varphi(x) - \varphi(\rho_0)|}{|x - \xi|^n} \intd x
\end{align*}
其中
\begin{align*}
    B &= \int\limits_{|x - \rho_0| \leq \frac{\delta}{2}} \frac{\xi_n|\varphi(x) - \varphi(\rho_0)|}{|x - \xi|^n} \intd x \\
    &\leq w_{n-1} \int_0^\delta \frac{\xi_n r^{n-1}}{(x - \xi')^2 + \xi_n^2 \sqrt{\frac{\alpha}{2}}} \longrightarrow 0 \\
    A &= \frac{2 \xi_n}{w_n} \int\limits_{|x - \rho_0| \geq \frac{\delta}{2}} \frac{|\varphi(x) - \varphi(\rho_0)|}{|x - \xi|^n} \intd x \\
    &= \frac{2}{w_n} \int\limits_{|x - \rho_0| \geq \frac{\delta}{2}} \frac{\xi |\varphi(x) - \varphi(\rho_0)|}{|x - \xi|^n} \intd x \\
    &\leq \frac{4 \|\varphi\|}{w_n} \xi_n \int_{\frac{\delta}{2}}^\infty \frac{r^{n-2}w_{n-1}}{r^n} \intd r \longrightarrow 0, \quad \text{最大模估计}
\end{align*}
\end{proof}

\begin{note}
在$\mathbb{R}^2$中,保角变换能够通过变换变成半平面或者单位圆的区域如Green函数构造出来。
\end{note}

\subsubsection{球形区域的Green函数}

$B_a (x_0) = \left\{x \in \mathbb{R}^n, |x - x_0| < a^2\right\}, y = \frac{x - x_0}{a}$,$\Longrightarrow B_1(0)$单位球。

\textbf{(1) 单位球}

我们希望:$g \bigg|_{\partial B_1(0)} = - \Gamma(x,\xi)$,即$\Gamma(x,\overline{\xi}) = \Gamma(x,\xi)$,$\forall x \in \partial B_1(0)$。其中$\overline{\xi}$为$\xi$关于单位球面的共轭点。

以上需要$|x - \xi| = |x - \overline{\xi}|$,距离。考虑
\begin{align*}
    g(x,\xi) = - \Gamma(cx, c \overline{\xi}), \forall |x| = 1
\end{align*}
我们只需要寻找$c$。
$$
\Gamma(cx, c \overline{\xi}) = \Gamma(x, \overline{\xi}) \; \Leftrightarrow |x - \xi| = |cx - c \overline{\xi}|
$$
又因为$\overline{\xi} = \frac{\xi}{|\xi|} \frac{1}{|\xi|}$,因此
\begin{align*}
    |x - \xi| &= c \left|x - \frac{\xi}{|\xi|^2}\right|, \; \forall |x| = 1 \\
    \Leftrightarrow \; |x - \xi|^2 &= c^2 \left|x - \frac{\xi}{|\xi|^2}\right|^2 \\
    \Leftrightarrow \; |x|^2 - 2|x\xi| + |\xi|^2 &= c^2 \left(|x|^2 - \frac{2 |x \xi|}{|\xi|^2} + \frac{1}{|\xi|^2}\right)
\end{align*}
其中,$|x| = 1$在单位球面上。因此,$B_1(0)$的格林函数为
\begin{align}
    G(x,\xi) = \Gamma(x,\xi) - \Gamma\left(x|\xi|, \frac{\xi}{|\xi|}\right)
\end{align}

\textbf{(2) $\Omega = B_a(x_0)$}

接下来我们再考虑$\Omega = B_a(x_0)$,
\begin{align*}
    \begin{cases}
        - \triangle u(x) = f(x), \; \text{in} \; B_a(x_0) \\
        u \bigg|_{\partial B_a(x_0)} = \varphi(x)
    \end{cases}
\end{align*}
作$y = \frac{x - x_0}{a}$,因此$B_a(x_0) \rightarrow B_1(0)$,令$v(y) = u(x) = u(a y + x_0)$,可以得到
\begin{align*}
    \begin{cases}
        - \triangle v(y) = a^2 f(a y + x_0) \\
        v \bigg|_{\partial B_1(0)} = \varphi(a y + x_0)
    \end{cases}
\end{align*}
由Poisson公式,$\Rightarrow$
$$
v(\xi) = a^2 \int_{B(0)} G(y, \xi)  f(ay + x_0) \intd y - \int_{\partial B_1(0)} \frac{\partial G(y,\xi)}{\partial \vec{n}_y} \varphi(ay + x_0) \intd S_y
$$
对于$\forall y \in B_a(x_0)$,有
\begin{align}
\label{equ5-2-11}
    u(y) = &a^{2-n} \int_{B_a(x_0)} \Gamma\left(\frac{x - x_0}{a}, \frac{y - x_0}{a}\right) - \Gamma\left(\left|\frac{y - x_0}{a}\right| \frac{x - x_0}{a}, \frac{y - x_0}{|y - x_0|}\right) f(x) \intd x \notag \\
    &- a^{2-n} \int_{\partial B_1(0)} \frac{\partial }{\partial n_x} \varphi(x) \intd x
\end{align}
\begin{thm}
\label{thm5-2-9}
    设$f \in \mathcal{C}^\alpha(\overline{B_a(x_0)})$,则对于某个$\alpha > 0$,$\varphi \in \mathcal{C}(\partial B_a(x_0))$,则由式(\ref{equ5-2-11})给出的$u(y) \in \mathcal{C}^2(B_\alpha(x_0)) \cap \mathcal{C}^1(\overline{B_\alpha(x_0))}$,并且满足(\ref{eq5-1-9})。
\end{thm}

\textbf{(3) 半球$B^+_a(x_0) \cap \{x_n > 0\}$}

\begin{align*}
    \begin{cases}
        - \triangle u(x) = f(x), \; \text{in} \; B_a(x_0) \\
        u\bigg|_{x_n = 0} = \varphi_1(x), \quad \text{or} \; \frac{\partial u}{\partial \vec{n}} = \varphi_1(x)\\
        u \bigg|_{\partial B_a(x_0) \cap \{x_n > 0\}} = \varphi_0(x)
    \end{cases}
\end{align*}
Green函数:构造标准区域,其他区域向该区域靠。

\begin{enumerate}[Step 1:]
    \item 作变换将$\varphi_1$化为零(齐次)
    \item 作奇(偶)延拓 变成$B_a(0)$上的问题
    \item 用定理(\ref{thm5-2-9})再将表达式限制在半球上
\end{enumerate}

\textbf{(4) 圆盘上的Poisson公式,将(\ref{equ5-2-11})写成球坐标形式}

以$n = 2$为例,令$x_0 = 0$,$x = (r \cos \alpha, r \sin \alpha), y = (\rho \cos \theta, \rho \sin \theta)$,则
$$
\Gamma(x,y) = - \frac{1}{4 \pi} \ln (r^2 + \rho^2 - 2 r \rho \cos(\alpha - \theta))
$$
并且
\begin{align*}
    \Gamma\left(\frac{x}{a}, \frac{y}{a}\right) - \Gamma\left(|y|\frac{x}{a^2}, \frac{y}{y}\right) = \frac{1}{4 \pi} \ln \left(\frac{\rho^2 r^2 + a^4 -  a^2 \rho r \cos(\alpha - \theta)}{a^2 [r^2 + \rho^2 - 2 r \rho \cos(\alpha - \theta)]}\right)
\end{align*}
那么
\begin{align*}
    \frac{\partial}{\partial r} \left\{\Gamma\left(\frac{x}{a}, \frac{y}{a}\right) - \Gamma\left(|y|\frac{x}{a^2}, \frac{y}{y}\right)\right\} = \frac{1}{2 \pi a} \frac{\rho^2 - a^2}{\rho^2 + a^2 - 2 a \rho \cos(\alpha - \theta)}
\end{align*}

考虑$\intd x = r \intd r \intd \alpha, \intd S_x = r \intd \alpha$,那么
\begin{align}
    \label{eq5-2-12}
    u(y) = u(\rho, \theta) &= \frac{1}{4 \pi} \int_0^a r \intd r \int_0^{2 \pi} \ln \left[\frac{\rho^2 r^2 + a^4 - 2 \rho a^2\cos(\alpha - \theta)}{a^2 [\rho^2 + a^2 - 2 a \rho \cos(\alpha - \theta)]}\right] f(r,\alpha) \intd r \intd \alpha \notag \\
    &+ \frac{1}{2 \pi} \int_0^{2 \pi} \frac{a^2 - \rho^2}{\rho^2 + a^2 - 2 a \rho \cos(\alpha - \theta)} \varphi(\alpha) \intd \alpha
\end{align}

\subsection{极值原理与最大模估计(E. Hopf)}

\subsubsection{极值原理}

\begin{align}
    \label{Q5-3-1}
    L u = - \triangle u + \sum_{i=1}^n b^i(x) u_{x_i} + c(x) u
\end{align}
其中$\Omega \in \mathbb{R}^n$是开集,$b^i(x), c(x)$是定义在$\Omega$上的已知函数。设$x_0 \in \Omega$, $u(x_0)$取到极大值,$u \in \mathcal{C}^2(\Omega)$,由此
\begin{align*}
    &\Rightarrow \triangledown u (x_0) = 0, \;  - \triangle u (x_0) \geq 0 \\
    &\Rightarrow L u(x_0) \geq c(x_0) u_0
\end{align*}

\begin{claim}
    如果$L u < 0$ in $\Omega$,且$c(x) = 0$,则$u$在$\Omega$中不可能取到正的极大值。由此可以知道
    $$
    \max\limits_{\overline{\Omega}} u \leq \max\limits_{\partial \Omega} u^+
    $$
    其中$u^+(x) = \max\{u(x), 0\}$, $u^-(x) = \min \{u(x), 0\}$
\end{claim}

\begin{thm} \textbf{E. Hopf (1927)弱极值定理. } 设$\Omega \subset \mathbb{R}^n$是有界开集,并且$\{b^i\}_{i=1}^n$中至少有一个函数在$\Omega$中有界,且$c(x) \geq 0$,$u \in \mathcal{C}^1(\overline{\Omega}) \cap \mathcal{C}^2(\Omega)$,则
\begin{enumerate}[(i)]
    \item 如果$L u \leq 0$ in $\Omega$,则
    $$
    \max\limits_{\overline{\Omega}} u \leq \max\limits_{\partial \Omega} u^+
    $$
    \item 如果$L u \geq 0$ in $\Omega$,则
    $$
    \min\limits_{\overline{\Omega}} u \geq \min\limits_{\partial \Omega} u^-
    $$
\end{enumerate}
\label{weak-max}
\end{thm}

\begin{proof}
令$v(x) = u(x) + \epsilon e^{a x_n}$,不妨设$b_n(x)$有界,则
$$
L v = L u + \epsilon e^{a x_n} \leq \epsilon e^{a x_n} (- a^2 + a b_n(x) + c(x))
$$
因为$b_n(x), c(x)$在$\Omega$中有界,因此取$a >> 1$时,有
$$
- a^2 + a b_n(x) + c(x) < 0, \; \text{in} \; \Omega
$$
因此$\forall a > 0, L v < 0$ in $\Omega$。
由声明知:
$$
\max\limits_{\overline{\Omega}}  v\leq \max\limits_{\partial \Omega} v^+
$$
令$\epsilon \rightarrow 0^+$,
$$
\max\limits_{\overline{\Omega}} u \leq \max\limits_{\partial \Omega} u^+
$$
再令$u = - v$,即可以证明(ii)。
\end{proof}

\begin{lemma} \textbf{Hopf边界引理. } 设$b^i, c$在$\Omega$中有界($i = 1,...n$),且$c(x) \geq 0$ in $\Omega$,如果$u \in \mathcal{C}^1(\overline{B_a(y)}) \cap \mathcal{C}^2(B_a(y))$满足
\begin{enumerate}[(i)]
    \item $L u \leq 0$ in $\Omega$
    \item $\exists x_0 \in \partial B_a(y)$ s.t. $u(x_0) \geq 0, u(x) \leq u(x_0)$ in $B_a(y)$
\end{enumerate}
则对任意单位向量$\vec{v}$,只要它与$\partial B_a(y)$在$x_0$处的外单位法向量夹角$ < \frac{\pi}{2}$,则一定有$\frac{\partial u}{\partial \vec{v}}(x_0) > 0$。
\end{lemma}

\begin{proof}
    idea: $v(x) = u(x) + w(x)$,我们希望$v(x) \leq v(x_0)$,并且$\frac{\partial w}{\partial \vec{v}}(x_0) < 0$,则
    $$
    \frac{\partial v}{\partial \vec{v}}(x_0) \geq 0 \; \Longrightarrow \; \frac{\partial u}{\partial \vec{v}}(x_0) \geq \frac{- \partial w}{\partial \vec{v}} > 0
    $$
    不妨设$y = 0$,$w = e^{- \delta |x|^2} - e^{-\delta a^2}$,满足$\frac{\partial w}{\partial \vec{v}} < 0$。再令$v(x) = u(x) + \epsilon w(x) - u(x_0)$,其中$\epsilon, \delta$待定。

    于是我们得到,$v \bigg|_{|x| = a} = u(x) \bigg|_{|x| = a} - u(x_0) \leq 0$,
    $$
    \max\limits_{|x| = \frac{a}{2}} (u(x) - u(x_0)) = A < 0
    $$
    因此可以取$\epsilon_0 > 0$ s.t. $v(x) \bigg|_{x = \frac{a}{2}} < 0$,$\forall 0 < \epsilon < \epsilon_0$。并且,
    \begin{align*}
        L v
        &= L u + \epsilon L w \\
        &\leq \epsilon \left(\sum_{i=1}^n b^i(x)x_i(2 \delta) \right) - \left(- 2\delta x_i e^{- \delta |x|^2}\right) \\
        &\leq \epsilon \left(\sum_{i=1}^n b^i(x)x_i(2 \delta) + 2\delta n - 2 \delta^2 |x|^2 + c(x)\right)e^{- \delta |x|^2} - \epsilon c(x) e^{- \delta |x|^2} \\
        &< 0, \; \text{if} \; \delta > > 1
    \end{align*}
    根据定理(\ref{weak-max}),$v(x) \leq \max\limits_{\Omega} v^+ = 0$,$v(x_0) = 0$ $\Rightarrow$ $\frac{\partial v}{\partial \vec{v}} \geq 0$。
    又因为,如果夹角小于$\frac{\pi}{2}$
    $$
    - \frac{\partial w}{\partial \vec{v}} = - \frac{\partial w}{\partial \vec{n}} \cos(\vec{n}, \vec{v}) = 2 a r e^{-ar^2} \cos(\vec{n}, \vec{v}) > 0
    $$
    则$\frac{\partial u}{\partial \vec{v}}(x_0) > 0$。
\end{proof}

\begin{thm} \textbf{强极值定理. }
\label{thm-strong-max}

设$b^i, c$都在$\Omega$中有界,$c(x) \geq 0$,$u \in \mathcal{C}^1(\overline{\Omega}) \cap \mathcal{C}^2(\Omega)$,则

\begin{enumerate}[(i)]
  \item 如果$L u \leq 0$ in $\Omega$,且$\exists x_0 \in \Omega$,s.t. $u(x_0) = \max\limits_{\overline{\Omega}} u \geq 0$,则
  $$
  u \equiv u(x_0)
  $$
  \item 如果$L u \geq 0$ in $\Omega$,且$\exists x_0 \in \Omega$,s.t. $u(x_0) = \min\limits_{\overline{\Omega}} u \leq 0$,则
  $$
  u \equiv u(x_0)
  $$
\end{enumerate}
\end{thm}

\begin{proof}
只需要证明(i)。令$\{x \in \Omega: u(x) = u(x_0)\}$,则显然$E$不为空。

只需要证明$E$是$\Omega$的相对闭集和开集,(则$E$只能为$\Omega$或者$\emptyset$,因为$E$非空,所以$E = \Omega$。)

用连续函数的定义,$E$只能为$\Omega$或者$\emptyset$,因为$E$非空,相对闭集成立。只要证$E$是$\Omega$中的开集,因为$\Omega$是开集,所以只要证明对于$x \in E$, $\exists r >0$ s.t. $B_r(x) \subset \Omega$。

若不存在,即存在$\hat{x} \in B_r(x)$ s.t. $\hat{x} \in (\Omega \backslash E) \cap B_r(x)$,设$d = |x - \hat{x}|$,则存在$x_0 \in E$,s.t. $d = |\hat{x} - x_0|$,考虑$B_d(\hat{x})$,$u(x_0) > u(x'), \forall x' \in B_d(\hat{x})$,由Hopf边界引理,$\frac{\partial u(x_0)}{\vec{v}} > 0$。

但是另一方面,$\triangledown u(x_0) = 0$,因为$x_0 \in E$,所以矛盾。
\end{proof}

\subsubsection{极值原理之推论}

\begin{cor} \textbf{极值原理推论.}
\begin{enumerate}
  \item Dirichlet问题
  \begin{align}
      \begin{cases}
        L u = f, \quad &\text{in} \; \Omega \\
        u = \varphi, \quad &\text{on} \; \partial \Omega
      \end{cases}
  \end{align}
  在$\mathcal{C}^1(\overline{\Omega}) \cap \mathcal{C}^2(\Omega)$上的解是唯一的(弱极值原理)。
  \item 比较原理
  \begin{align}
      \begin{cases}
        L u \leq L v, \quad &\text{in} \; \Omega \\
        u \leq v, \quad &\text{on} \; \partial \Omega
      \end{cases}
  \end{align}
  则$u \leq v$ in $\overline{\Omega}$。
  \item $b^i, c$都在$\Omega$中有界,$c(x) \geq 0$,$u \in \mathcal{C}^1(\overline{\Omega}) \cap \mathcal{C}^2(\Omega)$,并且$\Omega$满足内球条件,则Newman问题
    \begin{align}
      \begin{cases}
        L u = f, \quad &\text{in} \; \Omega \\
        \frac{\partial u}{\partial \vec{n}} = \varphi, \quad &\text{on} \; \partial \Omega
      \end{cases}
  \end{align}
  在空间$\mathcal{C}^1(\overline{\Omega}) \cap \mathcal{C}^2(\Omega)$上的任意两个解最多只相差一个常数。
\end{enumerate}
\end{cor}

\begin{proof}
  只证明(3)。

  对任意两个解,$u_1, u_2$,则$\max\limits_{\overline{\Omega}} (u_1 - u_2)$和$\max\limits_{\overline{\Omega}} (u_2 - u_1)$至少有一个$\geq 0$,不妨设$\max\limits_{\overline{\Omega}} (u_1 - u_2) \geq 0$。

  令$v(x) = u_1(x) - u_2(x)$,若存在$x_0 \in \Omega$,s.t. $v(x_0) = \max\limits_{\overline{\Omega}} v$,则$v \equiv v(x_0)$为常数。

  否则,$\exists x_0 \in \partial \Omega$,s.t. $v(x) < v(x_0), \forall x \in \Omega$。取$B_r(\hat{x}) \subset \Omega$,s.t. 该球与边界$\partial \Omega$相切于$x_0$。由Hopf引理,$\frac{\partial v}{\partial \vec{v}}(x_0) > 0$,但是,
  $$
  \frac{\partial v}{\partial \vec{v}}(x_0) = \varphi(x_0) - \varphi(x_0) = 0
  $$
  矛盾。
\end{proof}

\subsubsection{最大模估计}

记$K u = - \triangle u + c(x) u$

\textbf{(1) Dirichlet问题}

\begin{align}
    \label{Q5-3-6}
    \begin{cases}
      K u = f, \quad &\text{in} \; \Omega \\
      u = \varphi, &\text{on} \; \Omega
    \end{cases}
\end{align}
目标:找到$Z(x) = c(f,\varphi, \Omega)$ s.t. $|u(x)| \leq Z(x), \forall x \in \overline{\Omega}$。即$Z(x) \pm u(x) \geq 0$,这可以由定理(\ref{weak-max})推出,即$K(Z \pm u) \geq 0, Z \pm u \bigg|_{\partial \Omega} \geq 0$。

我们来构造$Z(x)$,
\begin{align}
    Z(x) = \sup\limits_{\partial \Omega} |\varphi| + C_1 \sup\limits_{\Omega} |f| (d^2 - |x - x_0|^2)
\end{align}
其中$x_0$取定,$d = \text{diam}(\Omega)$,$C_1 \geq 0$待定。
\begin{align*}
    K Z &= C_1 \sup\limits_{\Omega} |f| ( + 2 n) + c(x) Z \\
    &\geq 2 n C_1 \sup\limits_{\Omega} |f| \\
    &\geq \sup\limits_{\Omega} |f|
\end{align*}
如果我们取$C_1 = \frac{1}{2n}$。因此,$K(Z \pm u) \geq \sup\limits_{\Omega} |f| \pm f(x) \geq 0, \forall x \in \Omega$,又因为$Z \pm u \bigg|_{\partial \Omega} \geq 0$,所以$Z \pm u \geq 0$ in $\overline{\Omega}$。

\begin{thm}
设$\Omega \subset \mathbb{R}^n$有界,$c(x) \geq 0$有界函数,$u \in \mathcal{C}^1(\overline{\Omega}) \cap \mathcal{C}^2(\Omega)$是问题(\ref{Q5-3-6})的解,则
\begin{align}
    |u(x)| \leq \sup\limits_{\partial \Omega} |\varphi| + \frac{d^2}{2 n} \sup\limits_{\Omega} |f|, \quad \forall x \in \overline{\Omega}
\end{align}
其中$d = \text{diam}(\Omega)$。
\end{thm}

\textbf{(2) 混合问题}

\begin{align}
    \label{Q5-3-7}
    \begin{cases}
      K u = f, \; &\text{in} \; \Omega \\
      \frac{\partial u}{\partial \vec{n}} + \alpha(x) u = \varphi, &\text{on} \; \Omega
    \end{cases}
\end{align}
其中$\alpha(x) \geq \alpha_0 > 0$
为了简化,不妨设原点在$\Omega$中。
\begin{align}
    Z(x) =  \frac{1}{2n} \sup\limits_{\Omega} |f| (d^2 - |x|^2) + C_1
\end{align}
$C_1$给定,则$K Z(x) = \sup\limits_{\Omega} |f|$,同上知$K (Z \pm u) \geq \sup\limits_{\Omega} |f| \pm f(x) \geq 0$。只要找到$C_1$,使得$Z \pm u \bigg|_{\overline{\Omega}} \geq 0$。

令$w(x) = Z(x) \pm u(x)$,根据弱极值定理知,$w(x)$的负最小值在边界上取得,设为$x_0$,$w(x_0) < 0$。则根据强极值原理,$\frac{\partial w}{\partial \vec{n}} <0$,则
$$
\frac{\partial w}{\partial \vec{n}}(x_0) + \alpha(x_0) w(x_0) \leq \alpha_0 w(x_0) < 0
$$
另一方面,我们需要
$$
\frac{\partial w}{\partial \vec{n}} + \alpha(x) w(x) \bigg|_{\partial \Omega} = \frac{\partial Z}{\partial \vec{n}} + \alpha(x) Z(x) \pm \varphi(x) \geq 0
$$
因此只需要$C_1$满足$\frac{\partial Z}{\partial \vec{n}} + \alpha(x) Z(x) \geq \sup\limits_{\Omega} |\varphi|$ on $\partial \Omega$。
\begin{align*}
    \frac{\partial Z}{\partial \vec{n}} + \alpha(x) Z(x)
    &= \frac{1}{2n} \sup\limits_{\Omega} |f| (-2 \vec{x} \cdot \vec{n}) + \frac{\alpha}{2n} \sup\limits_{\Omega} |f| (d^2 - |x|^2) + \alpha(x) C_1 \\
    &\geq \frac{1}{2n} \sup\limits_{\Omega} |f| (- d^2 - 1) + \alpha_0 C_1 \geq \sup\limits_{\Omega} |\varphi|
\end{align*}
因此可以取
\begin{align}
    C_1 = \frac{1}{\alpha_0} \left[\sup\limits_{\Omega} |\varphi| + \frac{d^2 + 1}{2n} \sup\limits_{\Omega} |f| \right]
\end{align}
因此我们可以推出在$x_0$处的矛盾,这样$w(x)$就不存在负最小值。因此,$Z \pm u \bigg|_{\overline{\Omega}} \geq 0$。

\begin{thm}
设$\Omega \subset \mathbb{R}^n$有界,$c(x) \geq 0$有界函数,$\alpha(x) \geq \alpha_0 > 0$, $u \in \mathcal{C}^1(\overline{\Omega}) \cap \mathcal{C}^2(\Omega)$是问题(\ref{Q5-3-7})的解,则
\begin{align}
    |u(x)|
    &\leq \frac{1}{\alpha_0} \left[\sup\limits_{\Omega} |\varphi| + \frac{d^2 + 1}{2n} \sup\limits_{\Omega} |f| \right] + \frac{d^2}{2n} \sup\limits_{\Omega} |f| \notag \\
    &\leq C \left[\sup\limits_{\Omega} |\varphi| + \sup\limits_{\Omega} |f|\right]
\end{align}
其中$d = \text{diam}(\Omega)$,
\begin{align}
    C = \max \left\{\frac{1}{\alpha_0}, \frac{1}{2n} \left(\frac{d^2 + 1}{\alpha_0} + d^2\right)\right\}
\end{align}
\end{thm}

\subsection{能量估计}

\begin{align}
    \label{Q5-4-1}
    \begin{cases}
      - \triangle u + c(x) u = f, \; &\text{in} \; \Omega \\
      u = 0, &\text{on} \; \Omega
    \end{cases}
\end{align}

\begin{lemma} \textbf{Friedrichs不等式. }
设$\Omega \subset \mathbb{R}^n$有界,$u \in \mathcal{C}^1(\overline{\Omega})$,且$u = 0$ on $\partial \Omega$,则
\begin{align}
    \int_\Omega u^2 \intd x \leq d^2 \int_\Omega |\triangledown u|^2 \intd x
\end{align}
其中$d = \text{diam}(\Omega)$。
\end{lemma}

\begin{proof}
  不妨设$\Omega \subset \{0 \leq x_i \leq d\}$,则$u \equiv 0$ in $\mathbb{R}^n \backslash \overline{\Omega}$。

  $\forall y = (y_n, y')$有
  $$
  u(y) = \int_0^{y_n} \frac{\partial u}{\partial y_n} (t, y') \intd t
  $$
  则
  \begin{align*}
      \int_\Omega u^2 \intd y \leq
  \end{align*}
\end{proof}

\newpage

\section{Homework}

\begin{hw} \textbf{第一周}.
3月11日交:教材Chapter 1, 1 \& 2.
\end{hw}

\begin{hw} \textbf{第二周}.
3月11日交:教材Chapter 1, 6,9,13,14,15,16.
\end{hw}

\begin{hw} \textbf{第二周}.
3月11日交:教材Chapter 2, 3(1)(4)。
\end{hw}

\begin{hw} \textbf{第三周}.
3月18日交:教材Chapter 2, 1,4,8.
\end{hw}


\begin{hw} \textbf{第三周}.
3月18日交:教材Chapter 2, 9 \& 13.
\end{hw}

\begin{hw} \textbf{第四周}.
3月25日交:教材Chapter 2, 10 \& 12.
\end{hw}

\begin{hw} \textbf{第四周}.
3月25日交:教材Chapter 2, 19 \& 21.
\end{hw}

\begin{hw} \textbf{第五周}.
4月1日交:教材Chapter 2, 22(2)(4).
\end{hw}

\begin{hw} \textbf{第五周}.
4月1日交:教材Chapter 2, 23(3)(4), 25, 26(3)(4).
\end{hw}

\begin{hw} \textbf{第六周}.
4月8日交?:教材Chapter 2, 28

补充习题:求$\mathbb{R}^n$中的函数$u(x) = |x|^{- r}$的弱导数,$\frac{\partial^2 u}{\partial x_1 \partial x_n}, r < n$。
\end{hw}

\begin{hw} \textbf{第七周}.
4月15日交:教材Chapter 2, 29
\end{hw}

\begin{hw} \textbf{第七周}.
4月15日交:教材Chapter 3, 1(1)(4), 2(2)(6), 3(1)
\end{hw}

\begin{hw} \textbf{第八周}.
4月22日交:教材Chapter 3, 4
\end{hw}

\begin{hw} \textbf{第八周}.
4月22日交:教材Chapter 3, 5(3,5), 6(1,3), 7(3)
\end{hw}

\begin{hw} \textbf{第九周}.
4月29日交:教材Chapter 3, 8(1,3)
\end{hw}

\begin{hw} \textbf{第十周}.
5月6日交:教材Chapter 3, 9(1,5), 10(2)
\end{hw}

\begin{hw} \textbf{第十一周}.
5月13日交:教材Chapter 3, 13,18,19,20,21
\end{hw}

\begin{hw} \textbf{第十二周}.
5月20日交:教材Chapter 3, 23
\end{hw}

\begin{hw} \textbf{第十二周}.
5月20日交:证明Lemma(\ref{lemma5-1-1})(ii)
\end{hw}

\begin{hw} \textbf{第十三周}.
5月27日交:教材Chapter 4, 1, 3, 4, 5, 6, 8
\end{hw}

\begin{hw} \textbf{第十四周}.
6月3日交:教材Chapter 4, 14, 15, 19, 20, 21
\end{hw}

\begin{hw} \textbf{第十五周}.
6月10日交:教材Chapter 4, 24, 25, 32, 33
\end{hw}

\begin{hw} \textbf{第十六周}.
6月17日交:教材Chapter 4, 31, 34, 36
\end{hw}


\end{document}
